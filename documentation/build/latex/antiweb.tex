% Generated by Sphinx.
\def\sphinxdocclass{report}
\newif\ifsphinxKeepOldNames \sphinxKeepOldNamestrue
\documentclass[letterpaper,10pt,english]{sphinxmanual}
\usepackage{iftex}

\ifPDFTeX
  \usepackage[utf8]{inputenc}
\fi
\ifdefined\DeclareUnicodeCharacter
  \DeclareUnicodeCharacter{00A0}{\nobreakspace}
\fi
\usepackage{cmap}
\usepackage[T1]{fontenc}
\usepackage{amsmath,amssymb,amstext}
\usepackage{babel}
\usepackage{times}
\usepackage[Bjarne]{fncychap}
\usepackage{longtable}
\usepackage{sphinx}
\usepackage{multirow}
\usepackage{eqparbox}


\addto\captionsenglish{\renewcommand{\figurename}{Fig.\@ }}
\addto\captionsenglish{\renewcommand{\tablename}{Table }}
\SetupFloatingEnvironment{literal-block}{name=Listing }

\addto\extrasenglish{\def\pageautorefname{page}}

\setcounter{tocdepth}{1}


\title{antiweb Documentation}
\date{Aug 08, 2016}
\release{2.0}
\author{Philipp Rathmanner}
\newcommand{\sphinxlogo}{}
\renewcommand{\releasename}{Release}
\makeindex

\makeatletter
\def\PYG@reset{\let\PYG@it=\relax \let\PYG@bf=\relax%
    \let\PYG@ul=\relax \let\PYG@tc=\relax%
    \let\PYG@bc=\relax \let\PYG@ff=\relax}
\def\PYG@tok#1{\csname PYG@tok@#1\endcsname}
\def\PYG@toks#1+{\ifx\relax#1\empty\else%
    \PYG@tok{#1}\expandafter\PYG@toks\fi}
\def\PYG@do#1{\PYG@bc{\PYG@tc{\PYG@ul{%
    \PYG@it{\PYG@bf{\PYG@ff{#1}}}}}}}
\def\PYG#1#2{\PYG@reset\PYG@toks#1+\relax+\PYG@do{#2}}

\expandafter\def\csname PYG@tok@nd\endcsname{\let\PYG@bf=\textbf\def\PYG@tc##1{\textcolor[rgb]{0.33,0.33,0.33}{##1}}}
\expandafter\def\csname PYG@tok@kd\endcsname{\let\PYG@bf=\textbf\def\PYG@tc##1{\textcolor[rgb]{0.00,0.44,0.13}{##1}}}
\expandafter\def\csname PYG@tok@ch\endcsname{\let\PYG@it=\textit\def\PYG@tc##1{\textcolor[rgb]{0.25,0.50,0.56}{##1}}}
\expandafter\def\csname PYG@tok@nc\endcsname{\let\PYG@bf=\textbf\def\PYG@tc##1{\textcolor[rgb]{0.05,0.52,0.71}{##1}}}
\expandafter\def\csname PYG@tok@gt\endcsname{\def\PYG@tc##1{\textcolor[rgb]{0.00,0.27,0.87}{##1}}}
\expandafter\def\csname PYG@tok@kr\endcsname{\let\PYG@bf=\textbf\def\PYG@tc##1{\textcolor[rgb]{0.00,0.44,0.13}{##1}}}
\expandafter\def\csname PYG@tok@il\endcsname{\def\PYG@tc##1{\textcolor[rgb]{0.13,0.50,0.31}{##1}}}
\expandafter\def\csname PYG@tok@gp\endcsname{\let\PYG@bf=\textbf\def\PYG@tc##1{\textcolor[rgb]{0.78,0.36,0.04}{##1}}}
\expandafter\def\csname PYG@tok@nb\endcsname{\def\PYG@tc##1{\textcolor[rgb]{0.00,0.44,0.13}{##1}}}
\expandafter\def\csname PYG@tok@sr\endcsname{\def\PYG@tc##1{\textcolor[rgb]{0.14,0.33,0.53}{##1}}}
\expandafter\def\csname PYG@tok@s\endcsname{\def\PYG@tc##1{\textcolor[rgb]{0.25,0.44,0.63}{##1}}}
\expandafter\def\csname PYG@tok@nv\endcsname{\def\PYG@tc##1{\textcolor[rgb]{0.73,0.38,0.84}{##1}}}
\expandafter\def\csname PYG@tok@cp\endcsname{\def\PYG@tc##1{\textcolor[rgb]{0.00,0.44,0.13}{##1}}}
\expandafter\def\csname PYG@tok@mi\endcsname{\def\PYG@tc##1{\textcolor[rgb]{0.13,0.50,0.31}{##1}}}
\expandafter\def\csname PYG@tok@ss\endcsname{\def\PYG@tc##1{\textcolor[rgb]{0.32,0.47,0.09}{##1}}}
\expandafter\def\csname PYG@tok@nl\endcsname{\let\PYG@bf=\textbf\def\PYG@tc##1{\textcolor[rgb]{0.00,0.13,0.44}{##1}}}
\expandafter\def\csname PYG@tok@m\endcsname{\def\PYG@tc##1{\textcolor[rgb]{0.13,0.50,0.31}{##1}}}
\expandafter\def\csname PYG@tok@ni\endcsname{\let\PYG@bf=\textbf\def\PYG@tc##1{\textcolor[rgb]{0.84,0.33,0.22}{##1}}}
\expandafter\def\csname PYG@tok@c1\endcsname{\let\PYG@it=\textit\def\PYG@tc##1{\textcolor[rgb]{0.25,0.50,0.56}{##1}}}
\expandafter\def\csname PYG@tok@gs\endcsname{\let\PYG@bf=\textbf}
\expandafter\def\csname PYG@tok@mf\endcsname{\def\PYG@tc##1{\textcolor[rgb]{0.13,0.50,0.31}{##1}}}
\expandafter\def\csname PYG@tok@s1\endcsname{\def\PYG@tc##1{\textcolor[rgb]{0.25,0.44,0.63}{##1}}}
\expandafter\def\csname PYG@tok@gi\endcsname{\def\PYG@tc##1{\textcolor[rgb]{0.00,0.63,0.00}{##1}}}
\expandafter\def\csname PYG@tok@s2\endcsname{\def\PYG@tc##1{\textcolor[rgb]{0.25,0.44,0.63}{##1}}}
\expandafter\def\csname PYG@tok@cs\endcsname{\def\PYG@tc##1{\textcolor[rgb]{0.25,0.50,0.56}{##1}}\def\PYG@bc##1{\setlength{\fboxsep}{0pt}\colorbox[rgb]{1.00,0.94,0.94}{\strut ##1}}}
\expandafter\def\csname PYG@tok@sb\endcsname{\def\PYG@tc##1{\textcolor[rgb]{0.25,0.44,0.63}{##1}}}
\expandafter\def\csname PYG@tok@cm\endcsname{\let\PYG@it=\textit\def\PYG@tc##1{\textcolor[rgb]{0.25,0.50,0.56}{##1}}}
\expandafter\def\csname PYG@tok@gh\endcsname{\let\PYG@bf=\textbf\def\PYG@tc##1{\textcolor[rgb]{0.00,0.00,0.50}{##1}}}
\expandafter\def\csname PYG@tok@ne\endcsname{\def\PYG@tc##1{\textcolor[rgb]{0.00,0.44,0.13}{##1}}}
\expandafter\def\csname PYG@tok@vg\endcsname{\def\PYG@tc##1{\textcolor[rgb]{0.73,0.38,0.84}{##1}}}
\expandafter\def\csname PYG@tok@kn\endcsname{\let\PYG@bf=\textbf\def\PYG@tc##1{\textcolor[rgb]{0.00,0.44,0.13}{##1}}}
\expandafter\def\csname PYG@tok@na\endcsname{\def\PYG@tc##1{\textcolor[rgb]{0.25,0.44,0.63}{##1}}}
\expandafter\def\csname PYG@tok@c\endcsname{\let\PYG@it=\textit\def\PYG@tc##1{\textcolor[rgb]{0.25,0.50,0.56}{##1}}}
\expandafter\def\csname PYG@tok@err\endcsname{\def\PYG@bc##1{\setlength{\fboxsep}{0pt}\fcolorbox[rgb]{1.00,0.00,0.00}{1,1,1}{\strut ##1}}}
\expandafter\def\csname PYG@tok@sx\endcsname{\def\PYG@tc##1{\textcolor[rgb]{0.78,0.36,0.04}{##1}}}
\expandafter\def\csname PYG@tok@gu\endcsname{\let\PYG@bf=\textbf\def\PYG@tc##1{\textcolor[rgb]{0.50,0.00,0.50}{##1}}}
\expandafter\def\csname PYG@tok@sd\endcsname{\let\PYG@it=\textit\def\PYG@tc##1{\textcolor[rgb]{0.25,0.44,0.63}{##1}}}
\expandafter\def\csname PYG@tok@kc\endcsname{\let\PYG@bf=\textbf\def\PYG@tc##1{\textcolor[rgb]{0.00,0.44,0.13}{##1}}}
\expandafter\def\csname PYG@tok@w\endcsname{\def\PYG@tc##1{\textcolor[rgb]{0.73,0.73,0.73}{##1}}}
\expandafter\def\csname PYG@tok@kp\endcsname{\def\PYG@tc##1{\textcolor[rgb]{0.00,0.44,0.13}{##1}}}
\expandafter\def\csname PYG@tok@go\endcsname{\def\PYG@tc##1{\textcolor[rgb]{0.20,0.20,0.20}{##1}}}
\expandafter\def\csname PYG@tok@mh\endcsname{\def\PYG@tc##1{\textcolor[rgb]{0.13,0.50,0.31}{##1}}}
\expandafter\def\csname PYG@tok@ge\endcsname{\let\PYG@it=\textit}
\expandafter\def\csname PYG@tok@gr\endcsname{\def\PYG@tc##1{\textcolor[rgb]{1.00,0.00,0.00}{##1}}}
\expandafter\def\csname PYG@tok@se\endcsname{\let\PYG@bf=\textbf\def\PYG@tc##1{\textcolor[rgb]{0.25,0.44,0.63}{##1}}}
\expandafter\def\csname PYG@tok@nt\endcsname{\let\PYG@bf=\textbf\def\PYG@tc##1{\textcolor[rgb]{0.02,0.16,0.45}{##1}}}
\expandafter\def\csname PYG@tok@vc\endcsname{\def\PYG@tc##1{\textcolor[rgb]{0.73,0.38,0.84}{##1}}}
\expandafter\def\csname PYG@tok@sh\endcsname{\def\PYG@tc##1{\textcolor[rgb]{0.25,0.44,0.63}{##1}}}
\expandafter\def\csname PYG@tok@vi\endcsname{\def\PYG@tc##1{\textcolor[rgb]{0.73,0.38,0.84}{##1}}}
\expandafter\def\csname PYG@tok@kt\endcsname{\def\PYG@tc##1{\textcolor[rgb]{0.56,0.13,0.00}{##1}}}
\expandafter\def\csname PYG@tok@o\endcsname{\def\PYG@tc##1{\textcolor[rgb]{0.40,0.40,0.40}{##1}}}
\expandafter\def\csname PYG@tok@mb\endcsname{\def\PYG@tc##1{\textcolor[rgb]{0.13,0.50,0.31}{##1}}}
\expandafter\def\csname PYG@tok@ow\endcsname{\let\PYG@bf=\textbf\def\PYG@tc##1{\textcolor[rgb]{0.00,0.44,0.13}{##1}}}
\expandafter\def\csname PYG@tok@nf\endcsname{\def\PYG@tc##1{\textcolor[rgb]{0.02,0.16,0.49}{##1}}}
\expandafter\def\csname PYG@tok@nn\endcsname{\let\PYG@bf=\textbf\def\PYG@tc##1{\textcolor[rgb]{0.05,0.52,0.71}{##1}}}
\expandafter\def\csname PYG@tok@gd\endcsname{\def\PYG@tc##1{\textcolor[rgb]{0.63,0.00,0.00}{##1}}}
\expandafter\def\csname PYG@tok@no\endcsname{\def\PYG@tc##1{\textcolor[rgb]{0.38,0.68,0.84}{##1}}}
\expandafter\def\csname PYG@tok@sc\endcsname{\def\PYG@tc##1{\textcolor[rgb]{0.25,0.44,0.63}{##1}}}
\expandafter\def\csname PYG@tok@bp\endcsname{\def\PYG@tc##1{\textcolor[rgb]{0.00,0.44,0.13}{##1}}}
\expandafter\def\csname PYG@tok@cpf\endcsname{\let\PYG@it=\textit\def\PYG@tc##1{\textcolor[rgb]{0.25,0.50,0.56}{##1}}}
\expandafter\def\csname PYG@tok@mo\endcsname{\def\PYG@tc##1{\textcolor[rgb]{0.13,0.50,0.31}{##1}}}
\expandafter\def\csname PYG@tok@si\endcsname{\let\PYG@it=\textit\def\PYG@tc##1{\textcolor[rgb]{0.44,0.63,0.82}{##1}}}
\expandafter\def\csname PYG@tok@k\endcsname{\let\PYG@bf=\textbf\def\PYG@tc##1{\textcolor[rgb]{0.00,0.44,0.13}{##1}}}

\def\PYGZbs{\char`\\}
\def\PYGZus{\char`\_}
\def\PYGZob{\char`\{}
\def\PYGZcb{\char`\}}
\def\PYGZca{\char`\^}
\def\PYGZam{\char`\&}
\def\PYGZlt{\char`\<}
\def\PYGZgt{\char`\>}
\def\PYGZsh{\char`\#}
\def\PYGZpc{\char`\%}
\def\PYGZdl{\char`\$}
\def\PYGZhy{\char`\-}
\def\PYGZsq{\char`\'}
\def\PYGZdq{\char`\"}
\def\PYGZti{\char`\~}
% for compatibility with earlier versions
\def\PYGZat{@}
\def\PYGZlb{[}
\def\PYGZrb{]}
\makeatother

\renewcommand\PYGZsq{\textquotesingle}

\begin{document}

\maketitle
\tableofcontents
\phantomsection\label{index::doc}


Contents:


\chapter{Installation}
\label{installation:installation}\label{installation::doc}\label{installation:documentation}
The most important parts of the system are Python 3, Sphinx and antiweb. Because of incompatibility with Python 2 you can't create
documentaries of Python 2 programs.
\begin{quote}
\begin{itemize}
\item {} 
Install Python 3.4

\item {} 
After installing Python, run below commands in cmd (you can add the directories to the PATH or navigate to the directories {[}I would prefer the first option{]})

\end{itemize}

\begin{Verbatim}[commandchars=\\\{\}]
\PYG{n}{pip} \PYG{n}{install} \PYG{n}{sphinx}
\end{Verbatim}
\begin{itemize}
\item {} 
\sphinxcode{IMPORTANT}: If you get a connection error the reason is most likely your proxy. You then have to use a tool like ``cntlm'' and add a proxy flag when running pip, example:

\end{itemize}

\begin{Verbatim}[commandchars=\\\{\}]
\PYG{n}{pip} \PYG{n}{install} \PYG{n}{sphinx} \PYG{o}{\PYGZhy{}}\PYG{o}{\PYGZhy{}}\PYG{n}{proxy} \PYG{n}{http}\PYG{p}{:}\PYG{o}{/}\PYG{o}{/}\PYG{n}{localhost}\PYG{p}{:}\PYG{l+m+mi}{8123}
\end{Verbatim}
\begin{itemize}
\item {} 
The pre-installed version of babel doesn't work with Python 3, so we need to install an updated version from GitHub. Start the Git Shell and enter this command:

\end{itemize}

\begin{Verbatim}[commandchars=\\\{\}]
\PYG{n}{pip} \PYG{n}{install} \PYG{n}{git}\PYG{o}{+}\PYG{n}{https}\PYG{p}{:}\PYG{o}{/}\PYG{o}{/}\PYG{n}{github}\PYG{o}{.}\PYG{n}{com}\PYG{o}{/}\PYG{n}{jun66j5}\PYG{o}{/}\PYG{n}{babel}\PYG{o}{.}\PYG{n}{git} \PYG{p}{[}\PYG{n}{options}\PYG{p}{]}
\end{Verbatim}
\begin{itemize}
\item {} 
Run sphinx-quickstart.exe and follow the steps to configure sphinx as you like it (however, say yes to all extensions suggested during the process). The sphinx quickstart created a project folder for you, open the conf.py which is in that folder

\item {} 
You should find something like this:
\begin{quote}

\begin{Verbatim}[commandchars=\\\{\},numbers=left,firstnumber=1,stepnumber=1]
\PYG{n}{extensions} \PYG{o}{=} \PYG{p}{[}
\PYG{l+s+s1}{\PYGZsq{}}\PYG{l+s+s1}{sphinx.ext.autodoc}\PYG{l+s+s1}{\PYGZsq{}}\PYG{p}{,}
\PYG{l+s+s1}{\PYGZsq{}}\PYG{l+s+s1}{sphinx.ext.doctest}\PYG{l+s+s1}{\PYGZsq{}}\PYG{p}{,}
\PYG{l+s+s1}{\PYGZsq{}}\PYG{l+s+s1}{sphinx.ext.intersphinx}\PYG{l+s+s1}{\PYGZsq{}}\PYG{p}{,}
\PYG{l+s+s1}{\PYGZsq{}}\PYG{l+s+s1}{sphinx.ext.todo}\PYG{l+s+s1}{\PYGZsq{}}\PYG{p}{,}
\PYG{l+s+s1}{\PYGZsq{}}\PYG{l+s+s1}{sphinx.ext.coverage}\PYG{l+s+s1}{\PYGZsq{}}\PYG{p}{,}
\PYG{l+s+s1}{\PYGZsq{}}\PYG{l+s+s1}{sphinx.ext.mathjax}\PYG{l+s+s1}{\PYGZsq{}}\PYG{p}{,}
\PYG{l+s+s1}{\PYGZsq{}}\PYG{l+s+s1}{sphinx.ext.ifconfig}\PYG{l+s+s1}{\PYGZsq{}}\PYG{p}{,}
\PYG{l+s+s1}{\PYGZsq{}}\PYG{l+s+s1}{sphinx.ext.viewcode}\PYG{l+s+s1}{\PYGZsq{}}\PYG{p}{,}
\PYG{p}{]}
\end{Verbatim}
\end{quote}

\end{itemize}
\begin{itemize}
\item {} 
Add `sphinx.ext.graphviz' at the end and it will look like this:
\begin{quote}

\begin{Verbatim}[commandchars=\\\{\},numbers=left,firstnumber=1,stepnumber=1]
\PYG{n}{extensions} \PYG{o}{=} \PYG{p}{[}
\PYG{l+s+s1}{\PYGZsq{}}\PYG{l+s+s1}{sphinx.ext.autodoc}\PYG{l+s+s1}{\PYGZsq{}}\PYG{p}{,}
\PYG{l+s+s1}{\PYGZsq{}}\PYG{l+s+s1}{sphinx.ext.doctest}\PYG{l+s+s1}{\PYGZsq{}}\PYG{p}{,}
\PYG{l+s+s1}{\PYGZsq{}}\PYG{l+s+s1}{sphinx.ext.intersphinx}\PYG{l+s+s1}{\PYGZsq{}}\PYG{p}{,}
\PYG{l+s+s1}{\PYGZsq{}}\PYG{l+s+s1}{sphinx.ext.todo}\PYG{l+s+s1}{\PYGZsq{}}\PYG{p}{,}
\PYG{l+s+s1}{\PYGZsq{}}\PYG{l+s+s1}{sphinx.ext.coverage}\PYG{l+s+s1}{\PYGZsq{}}\PYG{p}{,}
\PYG{l+s+s1}{\PYGZsq{}}\PYG{l+s+s1}{sphinx.ext.mathjax}\PYG{l+s+s1}{\PYGZsq{}}\PYG{p}{,}
\PYG{l+s+s1}{\PYGZsq{}}\PYG{l+s+s1}{sphinx.ext.ifconfig}\PYG{l+s+s1}{\PYGZsq{}}\PYG{p}{,}
\PYG{l+s+s1}{\PYGZsq{}}\PYG{l+s+s1}{sphinx.ext.viewcode}\PYG{l+s+s1}{\PYGZsq{}}\PYG{p}{,}
\PYG{l+s+s1}{\PYGZsq{}}\PYG{l+s+s1}{sphinx.ext.graphviz}\PYG{l+s+s1}{\PYGZsq{}}\PYG{p}{,}
\PYG{p}{]}
\end{Verbatim}
\end{quote}

\end{itemize}
\begin{itemize}
\item {} 
Download the Graphviz msi installer from here: \url{http://www.graphviz.org/Download\_windows.php}

\item {} 
Add the bin folder inside the Graphviz directory to the PATH

\end{itemize}
\end{quote}


\section{Preparing the .rst files}
\label{installation:preparing-the-rst-files}\begin{quote}
\begin{itemize}
\item {} 
Copy the \sphinxcode{antiweb.py} file from my GitHub repository into the \sphinxcode{Python34} folder

\end{itemize}
\begin{itemize}
\item {} 
You can now begin creating a .rst file out of a C, C++, C\# and py file. To do that, simply use following command:

\end{itemize}

\begin{Verbatim}[commandchars=\\\{\}]
\PYG{n}{python} \PYG{n}{antiweb}\PYG{o}{.}\PYG{n}{py} \PYG{l+s+s2}{\PYGZdq{}}\PYG{l+s+s2}{PATH TO THE FILE/DIRECTORY}\PYG{l+s+s2}{\PYGZdq{}} \PYG{p}{[}\PYG{n}{options}\PYG{p}{]}
\end{Verbatim}
\begin{itemize}
\item {} 
You will then find a new file which is called \sphinxcode{Filename.rst} -\textgreater{} This file will be used in Sphinx to generate the documentation

\item {} 
antiweb can also create/edit a file called ``index.rst'' if you add the -i option when executing antiweb. In that index all processed files for documentation with sphinx are included

\item {} 
When you don't use the -i option you have to edit the file manually (sphinx-quickstart created it):

\end{itemize}

\begin{Verbatim}[commandchars=\\\{\},numbers=left,firstnumber=1,stepnumber=1]
\PYG{n}{Welcome} \PYG{n}{to} \PYG{n}{Antiweb}\PYG{l+s+s1}{\PYGZsq{}}\PYG{l+s+s1}{s documentation!}
\PYG{o}{==}\PYG{o}{==}\PYG{o}{==}\PYG{o}{==}\PYG{o}{==}\PYG{o}{==}\PYG{o}{==}\PYG{o}{==}\PYG{o}{==}\PYG{o}{==}\PYG{o}{==}\PYG{o}{==}\PYG{o}{==}\PYG{o}{==}\PYG{o}{==}\PYG{o}{==}\PYG{o}{==}\PYG{o}{=}

\PYG{n}{Contents}\PYG{p}{:}

   \PYG{o}{.}\PYG{o}{.} \PYG{n}{toctree}\PYG{p}{:}\PYG{p}{:}
      \PYG{p}{:}\PYG{n}{maxdepth}\PYG{p}{:} \PYG{l+m+mi}{2}

      \PYG{n}{filename} \PYG{c+c1}{\PYGZsh{}without file extension!}
\end{Verbatim}
\begin{itemize}
\item {} 
You can add multiple files, they will then be listed in the generated index of your html project

\item {} 
It is also possible to use Graphviz for graph visualizatin. A proper graph should look like this:

\end{itemize}

\begin{Verbatim}[commandchars=\\\{\}]
\PYG{o}{.}\PYG{o}{.} \PYG{n}{digraph}\PYG{p}{:}\PYG{p}{:} \PYG{n}{name}

 \PYG{l+s+s2}{\PYGZdq{}}\PYG{l+s+s2}{bubble 1}\PYG{l+s+s2}{\PYGZdq{}} \PYG{o}{\PYGZhy{}}\PYG{o}{\PYGZgt{}} \PYG{l+s+s2}{\PYGZdq{}}\PYG{l+s+s2}{bubble 2}\PYG{l+s+s2}{\PYGZdq{}} \PYG{o}{\PYGZhy{}}\PYG{o}{\PYGZgt{}} \PYG{l+s+s2}{\PYGZdq{}}\PYG{l+s+s2}{bubble 3}\PYG{l+s+s2}{\PYGZdq{}} \PYG{o}{\PYGZhy{}}\PYG{o}{\PYGZgt{}} \PYG{l+s+s2}{\PYGZdq{}}\PYG{l+s+s2}{bubble 1}\PYG{l+s+s2}{\PYGZdq{}}\PYG{p}{;}
\end{Verbatim}
\begin{itemize}
\item {} 
The output from above code would look like this:

\end{itemize}
\begin{itemize}
\item {} 
For more informatin on Graphviz visit \url{http://www.graphviz.org/}

\item {} 
When you have included the rst file in the index file, you can run Sphinx to finally create your documentation, here is an example:

\end{itemize}

\begin{Verbatim}[commandchars=\\\{\}]
\PYG{n}{sphinx}\PYG{o}{\PYGZhy{}}\PYG{n}{build}\PYG{o}{.}\PYG{n}{exe} \PYG{o}{\PYGZhy{}}\PYG{n}{b} \PYG{n}{html} \PYG{n}{sphinx}\PYGZbs{}\PYG{n}{source} \PYG{n}{sphinx}\PYGZbs{}\PYG{n}{source} \PYG{o}{\PYGZhy{}}\PYG{n}{D} \PYG{n}{graphviz\PYGZus{}dot}\PYG{o}{=}\PYG{n}{dot}\PYG{o}{.}\PYG{n}{exe}
\end{Verbatim}
\begin{itemize}
\item {} 
The \sphinxcode{-b} flag indicates the builder to use

\item {} 
\sphinxcode{sphinx\textbackslash{}source} indicates the path to the index.rst

\item {} 
\sphinxcode{sphinx\textbackslash{}source} indicates the output path (you can change your output path to every path where you want the final documentation)

\item {} 
\sphinxcode{-D graphviz\_dot=dot.exe} indicates the path for the graphviz virtualizer dot.exe (which you already copied to the Scripts folder)

\item {} 
After sphinx has finished you will find some .html files in the output path. This is your finished documentation.

\end{itemize}
\end{quote}


\chapter{Getting Started}
\label{getting_started::doc}\label{getting_started:getting-started}\begin{quote}

\begin{Verbatim}[commandchars=\\\{\}]
\PYGZgt{} python antiweb.py [options] SOURCEFILE
\end{Verbatim}

Tangles a source code file to a rst file.

Options
\begin{quote}
\begin{optionlist}{3cm}
\item [-{-}version]  
show program's version number and exit
\item [-h, -{-}help]  
show this help message and exit
\item [-o OUTPUT, -{-}output=OUTPUT]  
The output file name
\item [-t TOKEN, -{-}token=TOKEN]  
defines a token, usable by @if directives
\item [-w, -{-}warnings]  
suppresses warnings
\item [-r, -{-}recursive]  
Process every file in the given directory
\item [-i, -{-}index]  
Automatically write file(s) to Sphinx' index.rst
\end{optionlist}
\end{quote}

IMPORTANT: When combining certain flags, their behaviour may change. When combining the \sphinxcode{-o} with the \sphinxcode{-r} flag, \sphinxcode{-o} defines the output folder, not the output file.

Every @ directive in antiweb has to be a comment in order to be accepted by antiweb. However, antiweb will still recognize but not accept directives which aren't comments.
\end{quote}


\section{@start}
\label{getting_started:start}\begin{quote}

The \sphinxcode{@start} directive defines the beginning of
a text block. It is called with an argument defining
the name of the text block. There are two special text
blocks:
\begin{itemize}
\item {} 
\sphinxcode{()} The empty one defining the main text block

\item {} 
\sphinxcode{(\_\_macro\_\_)} defining a text block for implementing macros.

\end{itemize}

There are several possibilities to end a text block.
\begin{enumerate}
\item {} 
The end of the file

\item {} 
A line with a smaller indentation as the \sphinxcode{@start} directive.

\item {} 
Another start directive with same indentation.

\item {} 
An unnamed end (\sphinxcode{@}) directive with the same indentation as
the \sphinxcode{@start} directive.

\item {} 
A named end directive closing this block or an outer block.

\end{enumerate}

Text blocks defined by \sphinxcode{@start} can be nested.
\end{quote}


\section{@rstart(text block name)}
\label{getting_started:rstart-text-block-name}\begin{quote}

The \sphinxcode{@rstart} directive works like the \sphinxcode{@start} directive. While \sphinxcode{@start} removes its block completely from the containing block,
\sphinxcode{@rstart} replaces the lines with a \sphinxcode{\textless{}\textless{}name\textgreater{}\textgreater{}} - Sentinel.
\end{quote}


\section{@cstart(text block name)}
\label{getting_started:cstart-text-block-name}\begin{quote}

The \sphinxcode{@cstart} directive can be used as a shortcut for:

\begin{Verbatim}[commandchars=\\\{\}]
\PYG{n+nd}{@start}\PYG{p}{(}\PYG{n}{block}\PYG{p}{)}
\PYG{n+nd}{@code}
\end{Verbatim}
\end{quote}


\section{@{[}(text block name){]}}
\label{getting_started:text-block-name}\begin{quote}

The end(\sphinxcode{@}) directive ends a text block. Optional a specific text block can also be closed by defining its \sphinxcode{text block name}.
\end{quote}


\section{@include(text block name {[}, file{]})}
\label{getting_started:include-text-block-name-file}\begin{quote}

Once you have created a block  you can include it with the \sphinxcode{@include} directive. The order in which your blocks will appear in the documentation is defined by the order of the \sphinxcode{@include} directives

\begin{Verbatim}[commandchars=\\\{\}]
\PYG{n+nd}{@include}\PYG{p}{(}\PYG{n}{Blockname}\PYG{p}{)}
\end{Verbatim}

The directive can have a second file argument. If given, the directive inserts the text block of the specified file.
\end{quote}


\section{@RInclude(text block name)}
\label{getting_started:rinclude-text-block-name}\begin{quote}

The \sphinxcode{@rinclude(text block name)} directive is a is a replacement for:

\begin{Verbatim}[commandchars=\\\{\}]
\PYG{o}{.}\PYG{o}{.} \PYG{n}{\PYGZus{}text} \PYG{n}{block} \PYG{n}{name}\PYG{p}{:}

\PYG{o}{*}\PYG{o}{*}\PYG{o}{\PYGZlt{}\PYGZlt{}}\PYG{n}{text} \PYG{n}{block} \PYG{n}{name}\PYG{o}{\PYGZgt{}\PYGZgt{}}\PYG{o}{*}\PYG{o}{*}

\PYG{n+nd}{@include}\PYG{p}{(}\PYG{n}{text} \PYG{n}{block} \PYG{n}{name}\PYG{p}{)}
\end{Verbatim}
\end{quote}


\section{@code}
\label{getting_started:code}\begin{quote}

Of course you want parts of your source code in a Block in order to e.g. describe the function of it. You can do that by following this example. The code in between the directives will be normally recognized as code but also included in the documentation:

\begin{Verbatim}[commandchars=\\\{\},numbers=left,firstnumber=1,stepnumber=1]
\PYG{n+nd}{@code}
\PYG{c+c1}{\PYGZsh{}End of comment section}

\PYG{n}{Put} \PYG{n}{your} \PYG{n}{code} \PYG{n}{here}

\PYG{c+c1}{\PYGZsh{}Beginning of next comment section}
\PYG{n+nd}{@edoc}
\end{Verbatim}
\end{quote}


\section{@edoc}
\label{getting_started:edoc}\begin{quote}

The \sphinxcode{@edoc} directive ends a previously started \sphinxcode{@code} directive
\end{quote}


\section{Titles}
\label{getting_started:titles}\begin{quote}

There are also different types of titles with different indentation in the index. antiweb wants the indication marks, e.g. \#\#\#\# to
be exactly as long as the title. Creating a headline below a higher level headline makes it a sub-headline of the higher one, also
shown in the index table

\begin{Verbatim}[commandchars=\\\{\},numbers=left,firstnumber=1,stepnumber=1]
\PYG{c+c1}{\PYGZsh{}\PYGZsh{}\PYGZsh{}\PYGZsh{}\PYGZsh{}}
\PYG{n}{Title} \PYG{c+c1}{\PYGZsh{}This is the top level headline}
\PYG{c+c1}{\PYGZsh{}\PYGZsh{}\PYGZsh{}\PYGZsh{}\PYGZsh{}}

\PYG{o}{*}\PYG{o}{*}\PYG{o}{*}\PYG{o}{*}\PYG{o}{*}
\PYG{n}{Title} \PYG{c+c1}{\PYGZsh{}This is the mid level headline}
\PYG{o}{*}\PYG{o}{*}\PYG{o}{*}\PYG{o}{*}\PYG{o}{*}

\PYG{n}{Title} \PYG{c+c1}{\PYGZsh{}This is the low level headline}
\PYG{o}{==}\PYG{o}{==}\PYG{o}{=}
\end{Verbatim}
\end{quote}


\section{@indent spaces}
\label{getting_started:indent-spaces}\begin{quote}

You can indicate antiweb to make a manual indentation with the \sphinxcode{@indent spaces} directive, replacing \sphinxcode{spaces} by three would indent the text by three spaces
\end{quote}


\section{@define(identifier{[}, substitution{]})}
\label{getting_started:define-identifier-substitution}\begin{quote}

The \sphinxcode{@define} directive defines a macro, that can be used
with a \sphinxcode{@subst} directive. If a \sphinxcode{substitution}
argument is given, the macro defines an inline substitution.
Otherwise the \sphinxcode{@define} has to be ended by an \sphinxcode{@enifed}
directive.
\end{quote}


\section{@enifed(identifier)}
\label{getting_started:enifed-identifier}\begin{quote}

The \sphinxcode{@enifed} directive ends a macro defined by the
\sphinxcode{@define} directive.
\end{quote}


\section{@subst(identifier)}
\label{getting_started:subst-identifier}\begin{quote}

The \sphinxcode{@subst} directive is replaced by the substitution,
defined by a \sphinxcode{@define} directive. There are two predefined
macros:
\begin{quote}
\begin{description}
\item[{\sphinxcode{\_\_line\_\_}}] \leavevmode
Define the current line within the source code. The
\sphinxcode{@subst} can also handle operation with \sphinxcode{\_\_line\_\_}
like \sphinxcode{\_\_line\_\_ + 2}.

\item[{\sphinxcode{\_\_file\_\_}}] \leavevmode
Defines the current source file name.

\end{description}
\end{quote}
\end{quote}


\section{@if(token name)}
\label{getting_started:if-token-name}\begin{quote}

The \sphinxcode{@if} directive is used for conditional weaving.
The content of an \sphinxcode{@if}, \sphinxcode{@fi} block is waved if the
named token argument of \sphinxcode{@if}, is defined in the command line
by the \sphinxcode{-{-}token} option.
\end{quote}


\section{@fi(token name)}
\label{getting_started:fi-token-name}\begin{quote}

The \sphinxcode{@fi} ends an \sphinxcode{@if} directive
\end{quote}


\section{@ignore}
\label{getting_started:ignore}\begin{quote}

The \sphinxcode{@ignore} directive ignores the line in the
documentation output. It can be used for commentaries.
\end{quote}


\section{Indentation matters}
\label{getting_started:indentation-matters}\begin{quote}

In sphinx and antiweb, the indentation matters. To effectively nest blocks, create sub headlines and more you have to keep the indentation in mind. To nest a block or headline you have to indent it farther than its parent. In addition, your documentation looks much cleaner when structured like this.
\end{quote}
\begin{itemize}
\item {} 
\sphinxcode{This is the end of the basic introduction. For more information on antiweb simply read on.}

\end{itemize}


\chapter{Antiweb documentation}
\label{antiweb:antiweb-documentation}\label{antiweb::doc}
If you just want to generate the documentation from a source file use
the following function:
\index{generate() (built-in function)}

\begin{fulllineitems}
\phantomsection\label{antiweb:generate}\pysiglinewithargsret{\sphinxbfcode{generate}}{\emph{fname}, \emph{tokens}, \emph{warnings}}{}
Generates a rst file from a source file.
\begin{quote}\begin{description}
\item[{Parameters}] \leavevmode\begin{itemize}
\item {} 
\textbf{\texttt{fname}} (\emph{\texttt{string}}) -- The path to the source file.

\item {} 
\textbf{\texttt{tokens}} (\emph{\texttt{list}}) -- A list of string tokens, used for @if directives.

\item {} 
\textbf{\texttt{show\_warnings}} (\emph{\texttt{bool}}) -- Warnings will be written
via the logging module.

\end{itemize}

\end{description}\end{quote}

\begin{Verbatim}[commandchars=\\\{\},numbers=left,firstnumber=1,stepnumber=1]
\PYG{k}{def} \PYG{n+nf}{generate}\PYG{p}{(}\PYG{n}{fname}\PYG{p}{,} \PYG{n}{tokens}\PYG{p}{,} \PYG{n}{show\PYGZus{}warnings}\PYG{o}{=}\PYG{n+nb+bp}{False}\PYG{p}{)}\PYG{p}{:}
    \PYG{k}{try}\PYG{p}{:}
        \PYG{k}{with} \PYG{n+nb}{open}\PYG{p}{(}\PYG{n}{fname}\PYG{p}{,} \PYG{l+s+s2}{\PYGZdq{}}\PYG{l+s+s2}{r}\PYG{l+s+s2}{\PYGZdq{}}\PYG{p}{)} \PYG{k}{as} \PYG{n}{f}\PYG{p}{:}
            \PYG{n}{text} \PYG{o}{=} \PYG{n}{f}\PYG{o}{.}\PYG{n}{read}\PYG{p}{(}\PYG{p}{)}
    \PYG{k}{except} \PYG{n+ne}{IOError}\PYG{p}{:}
        \PYG{n}{logger}\PYG{o}{.}\PYG{n}{error}\PYG{p}{(}\PYG{l+s+s2}{\PYGZdq{}}\PYG{l+s+s2}{file not found: }\PYG{l+s+si}{\PYGZpc{}s}\PYG{l+s+s2}{\PYGZdq{}}\PYG{p}{,} \PYG{n}{fname}\PYG{p}{)}
        \PYG{n}{sys}\PYG{o}{.}\PYG{n}{exit}\PYG{p}{(}\PYG{l+m+mi}{1}\PYG{p}{)}

    \PYG{n}{lexer} \PYG{o}{=} \PYG{n}{pm}\PYG{o}{.}\PYG{n}{get\PYGZus{}lexer\PYGZus{}for\PYGZus{}filename}\PYG{p}{(}\PYG{n}{fname}\PYG{p}{)}
    \PYG{n}{reader} \PYG{o}{=} \PYG{n}{readers}\PYG{o}{.}\PYG{n}{get}\PYG{p}{(}\PYG{n}{lexer}\PYG{o}{.}\PYG{n}{name}\PYG{p}{,} \PYG{n}{Reader}\PYG{p}{)}\PYG{p}{(}\PYG{n}{lexer}\PYG{p}{)}

    \PYG{n}{document} \PYG{o}{=} \PYG{n}{Document}\PYG{p}{(}\PYG{n}{text}\PYG{p}{,} \PYG{n}{reader}\PYG{p}{,} \PYG{n}{fname}\PYG{p}{,} \PYG{n}{tokens}\PYG{p}{)}
    \PYG{k}{return} \PYG{n}{document}\PYG{o}{.}\PYG{n}{process}\PYG{p}{(}\PYG{n}{show\PYGZus{}warnings}\PYG{p}{)}
\end{Verbatim}

\end{fulllineitems}



\section{Objects}
\label{antiweb:objects}
The graph below show the main objects of antiweb:
\noindent
The {\hyperref[antiweb:Document]{\sphinxcrossref{\sphinxcode{document}}}} manages the complete transformation: It uses a
{\hyperref[antiweb:Reader]{\sphinxcrossref{\sphinxcode{reader}}}}  to parse source code. The {\hyperref[antiweb:Reader]{\sphinxcrossref{\sphinxcode{reader}}}}
creates {\hyperref[antiweb:directives]{\sphinxcrossref{\DUrole{std,std-ref}{directives}}}} objects for each found antiweb directive in the source
code. The source code is split in text blocks which consists of several
{\hyperref[antiweb:Line]{\sphinxcrossref{\sphinxcode{lines}}}}. The {\hyperref[antiweb:Document]{\sphinxcrossref{\sphinxcode{document}}}} process all
{\hyperref[antiweb:directives]{\sphinxcrossref{\DUrole{std,std-ref}{directives}}}}  to generate the output document.


\section{Directives}
\label{antiweb:directives}\label{antiweb:id1}

\subsection{Directive}
\label{antiweb:directive}\index{Directive (built-in class)}

\begin{fulllineitems}
\phantomsection\label{antiweb:Directive}\pysiglinewithargsret{\sphinxstrong{class }\sphinxbfcode{Directive}}{\emph{line}\sphinxoptional{, \emph{mo}}}{}
The base class of all directives.
Directives can be distinguished by the different tasks,
they handle, these Task are generally:
\begin{itemize}
\item {} 
identifying a text block ({\hyperref[antiweb:Directive.collect_block]{\sphinxcrossref{\sphinxcode{collect\_block()}}}})

\item {} 
inserting text in the output ({\hyperref[antiweb:Directive.process]{\sphinxcrossref{\sphinxcode{process()}}}})

\item {} 
modifying text in the output ({\hyperref[antiweb:Directive.process]{\sphinxcrossref{\sphinxcode{process()}}}})

\item {} 
deleting text in the output ({\hyperref[antiweb:Directive.process]{\sphinxcrossref{\sphinxcode{process()}}}})

\end{itemize}
\begin{quote}\begin{description}
\item[{Parameters}] \leavevmode\begin{itemize}
\item {} 
\textbf{\texttt{line}} -- the line number the directive was found

\item {} 
\textbf{\texttt{mo}} -- a match object of an regular expression

\end{itemize}

\end{description}\end{quote}

\begin{Verbatim}[commandchars=\\\{\},numbers=left,firstnumber=1,stepnumber=1]
\PYG{k}{class} \PYG{n+nc}{Directive}\PYG{p}{(}\PYG{n+nb}{object}\PYG{p}{)}\PYG{p}{:}
    \PYG{c+c1}{\PYGZsh{}Attributes}
    \PYG{o}{\PYGZlt{}\PYGZlt{}}\PYG{n}{Directive}\PYG{o}{.}\PYG{n}{expression}\PYG{o}{\PYGZgt{}\PYGZgt{}}
    \PYG{o}{\PYGZlt{}\PYGZlt{}}\PYG{n}{Directive}\PYG{o}{.}\PYG{n}{priority}\PYG{o}{\PYGZgt{}\PYGZgt{}}
    \PYG{o}{\PYGZlt{}\PYGZlt{}}\PYG{n}{Directive}\PYG{o}{.}\PYG{n}{line}\PYG{o}{\PYGZgt{}\PYGZgt{}}

    \PYG{c+c1}{\PYGZsh{}Methods}
    \PYG{o}{\PYGZlt{}\PYGZlt{}}\PYG{n}{Directive}\PYG{o}{.}\PYG{n}{\PYGZus{}\PYGZus{}init\PYGZus{}\PYGZus{}}\PYG{o}{\PYGZgt{}\PYGZgt{}}
    \PYG{o}{\PYGZlt{}\PYGZlt{}}\PYG{n}{Directive}\PYG{o}{.}\PYG{n}{collect\PYGZus{}block}\PYG{o}{\PYGZgt{}\PYGZgt{}}
    \PYG{o}{\PYGZlt{}\PYGZlt{}}\PYG{n}{Directive}\PYG{o}{.}\PYG{n}{process}\PYG{o}{\PYGZgt{}\PYGZgt{}}
    \PYG{o}{\PYGZlt{}\PYGZlt{}}\PYG{n}{Directive}\PYG{o}{.}\PYG{n}{match}\PYG{o}{\PYGZgt{}\PYGZgt{}}
    \PYG{o}{\PYGZlt{}\PYGZlt{}}\PYG{n}{Directive}\PYG{o}{.}\PYG{n}{\PYGZus{}\PYGZus{}repr\PYGZus{}\PYGZus{}}\PYG{o}{\PYGZgt{}\PYGZgt{}}
\end{Verbatim}
\index{expression (Directive attribute)}

\begin{fulllineitems}
\phantomsection\label{antiweb:Directive.expression}\pysigline{\sphinxbfcode{expression}}
A regular expression defining the directive.

\begin{Verbatim}[commandchars=\\\{\}]
\PYG{n}{expression} \PYG{o}{=} \PYG{l+s+s2}{\PYGZdq{}}\PYG{l+s+s2}{\PYGZdq{}}
\end{Verbatim}

\end{fulllineitems}

\index{priority (Directive attribute)}

\begin{fulllineitems}
\phantomsection\label{antiweb:Directive.priority}\pysigline{\sphinxbfcode{priority}}
An integer process priority. Directives with a lower priority
will be processed earlier.

\begin{Verbatim}[commandchars=\\\{\}]
\PYG{n}{priority} \PYG{o}{=} \PYG{l+m+mi}{10}
\end{Verbatim}

\end{fulllineitems}

\index{line (Directive attribute)}

\begin{fulllineitems}
\phantomsection\label{antiweb:Directive.line}\pysigline{\sphinxbfcode{line}}
A integer defining the original line number of the directive.

\begin{Verbatim}[commandchars=\\\{\}]
\PYG{n}{line} \PYG{o}{=} \PYG{n+nb+bp}{None}
\end{Verbatim}

\end{fulllineitems}

\index{\_\_init\_\_() (Directive method)}

\begin{fulllineitems}
\phantomsection\label{antiweb:Directive.__init__}\pysiglinewithargsret{\sphinxbfcode{\_\_init\_\_}}{\emph{line}\sphinxoptional{, \emph{mo}}}{}
The constructor

\begin{Verbatim}[commandchars=\\\{\}]
\PYG{k}{def} \PYG{n+nf}{\PYGZus{}\PYGZus{}init\PYGZus{}\PYGZus{}}\PYG{p}{(}\PYG{n+nb+bp}{self}\PYG{p}{,} \PYG{n}{line}\PYG{p}{,} \PYG{n}{mo}\PYG{o}{=}\PYG{n+nb+bp}{None}\PYG{p}{)}\PYG{p}{:}
    \PYG{n+nb+bp}{self}\PYG{o}{.}\PYG{n}{line} \PYG{o}{=} \PYG{n}{line}
\end{Verbatim}

\end{fulllineitems}

\index{collect\_block() (Directive method)}

\begin{fulllineitems}
\phantomsection\label{antiweb:Directive.collect_block}\pysiglinewithargsret{\sphinxbfcode{collect\_block}}{\emph{document}, \emph{index}}{}
This method is called by {\hyperref[antiweb:Document]{\sphinxcrossref{\sphinxcode{Document}}}}.
If the directive is defining a text block. It
retrieves the text lines of the block from the document
and return them.
\begin{quote}\begin{description}
\item[{Parameters}] \leavevmode\begin{itemize}
\item {} 
\textbf{\texttt{document}} ({\hyperref[antiweb:Document]{\sphinxcrossref{\sphinxcode{Document}}}}) -- the document calling the function.

\item {} 
\textbf{\texttt{index}} (\emph{\texttt{integer}}) -- the line index of the directive.

\end{itemize}

\item[{Returns}] \leavevmode
If the directive collects a block the return value
is a tuple \sphinxcode{(directive name, block of lines)}, or
\sphinxcode{None} otherwise.

\end{description}\end{quote}

\begin{Verbatim}[commandchars=\\\{\}]
\PYG{k}{def} \PYG{n+nf}{collect\PYGZus{}block}\PYG{p}{(}\PYG{n+nb+bp}{self}\PYG{p}{,} \PYG{n}{document}\PYG{p}{,} \PYG{n}{index}\PYG{p}{)}\PYG{p}{:}
    \PYG{k}{return} \PYG{n+nb+bp}{None}
\end{Verbatim}

\end{fulllineitems}

\index{process() (Directive method)}

\begin{fulllineitems}
\phantomsection\label{antiweb:Directive.process}\pysiglinewithargsret{\sphinxbfcode{process}}{\emph{document}, \emph{block}, \emph{index}}{}
This method is called by {\hyperref[antiweb:Document]{\sphinxcrossref{\sphinxcode{Document}}}}.
The directive should do whatever it is supposed to do.
\begin{quote}\begin{description}
\item[{Parameters}] \leavevmode\begin{itemize}
\item {} 
\textbf{\texttt{document}} ({\hyperref[antiweb:Document]{\sphinxcrossref{\sphinxcode{Document}}}}) -- the document calling the function.

\item {} 
\textbf{\texttt{block}} -- The line block the directive is in.

\item {} 
\textbf{\texttt{index}} (\emph{\texttt{integer}}) -- the line index of the directive
within the block.

\end{itemize}

\end{description}\end{quote}

\begin{Verbatim}[commandchars=\\\{\}]
\PYG{k}{def} \PYG{n+nf}{process}\PYG{p}{(}\PYG{n+nb+bp}{self}\PYG{p}{,} \PYG{n}{document}\PYG{p}{,} \PYG{n}{block}\PYG{p}{,} \PYG{n}{index}\PYG{p}{)}\PYG{p}{:}
    \PYG{k}{pass}
\end{Verbatim}

\end{fulllineitems}

\index{match() (Directive method)}

\begin{fulllineitems}
\phantomsection\label{antiweb:Directive.match}\pysiglinewithargsret{\sphinxbfcode{match}}{\emph{lines}}{}
This method is called by {\hyperref[antiweb:Document]{\sphinxcrossref{\sphinxcode{Document}}}}.
It gives the directive the chance to find and manipulate other
directives.
\begin{quote}\begin{description}
\item[{Parameters}] \leavevmode
\textbf{\texttt{lines}} (\emph{\texttt{list}}) -- a list of all document lines.

\end{description}\end{quote}

\begin{Verbatim}[commandchars=\\\{\}]
\PYG{k}{def} \PYG{n+nf}{match}\PYG{p}{(}\PYG{n+nb+bp}{self}\PYG{p}{,} \PYG{n}{lines}\PYG{p}{)}\PYG{p}{:}
    \PYG{k}{pass}
\end{Verbatim}

\end{fulllineitems}

\index{\_\_repr\_\_() (Directive method)}

\begin{fulllineitems}
\phantomsection\label{antiweb:Directive.__repr__}\pysiglinewithargsret{\sphinxbfcode{\_\_repr\_\_}}{}{}
returns a textual representation of the directive.

\begin{Verbatim}[commandchars=\\\{\}]
\PYG{k}{def} \PYG{n+nf}{\PYGZus{}\PYGZus{}repr\PYGZus{}\PYGZus{}}\PYG{p}{(}\PYG{n+nb+bp}{self}\PYG{p}{)}\PYG{p}{:}
    \PYG{k}{return} \PYG{l+s+s2}{\PYGZdq{}}\PYG{l+s+s2}{\PYGZlt{}}\PYG{l+s+si}{\PYGZpc{}s}\PYG{l+s+s2}{ at }\PYG{l+s+si}{\PYGZpc{}i}\PYG{l+s+s2}{\PYGZgt{}}\PYG{l+s+s2}{\PYGZdq{}} \PYG{o}{\PYGZpc{}} \PYG{p}{(}\PYG{n+nb+bp}{self}\PYG{o}{.}\PYG{n}{\PYGZus{}\PYGZus{}class\PYGZus{}\PYGZus{}}\PYG{o}{.}\PYG{n}{\PYGZus{}\PYGZus{}name\PYGZus{}\PYGZus{}}\PYG{p}{,} \PYG{n+nb+bp}{self}\PYG{o}{.}\PYG{n}{line}\PYG{p}{)}
\end{Verbatim}

\end{fulllineitems}


\end{fulllineitems}



\subsection{NameDirective}
\label{antiweb:namedirective}\index{NameDirective (built-in class)}

\begin{fulllineitems}
\phantomsection\label{antiweb:NameDirective}\pysiglinewithargsret{\sphinxstrong{class }\sphinxbfcode{NameDirective}}{\emph{line}, \emph{mo}}{}
The base class for directives with a name argument.
It inherits {\hyperref[antiweb:Directive]{\sphinxcrossref{\sphinxcode{Directive}}}}.
\begin{quote}\begin{description}
\item[{Parameters}] \leavevmode\begin{itemize}
\item {} 
\textbf{\texttt{line}} -- the line number the directive was found

\item {} 
\textbf{\texttt{mo}} -- a match object of an regular expression or
a string defining the name.

\end{itemize}

\end{description}\end{quote}
\index{name (NameDirective attribute)}

\begin{fulllineitems}
\phantomsection\label{antiweb:NameDirective.name}\pysigline{\sphinxbfcode{name}}
A string defining the argument of the directive.

\end{fulllineitems}


\begin{Verbatim}[commandchars=\\\{\},numbers=left,firstnumber=1,stepnumber=1]
\PYG{k}{class} \PYG{n+nc}{NameDirective}\PYG{p}{(}\PYG{n}{Directive}\PYG{p}{)}\PYG{p}{:}
    \PYG{k}{def} \PYG{n+nf}{\PYGZus{}\PYGZus{}init\PYGZus{}\PYGZus{}}\PYG{p}{(}\PYG{n+nb+bp}{self}\PYG{p}{,} \PYG{n}{line}\PYG{p}{,} \PYG{n}{mo}\PYG{p}{)}\PYG{p}{:}
        \PYG{n+nb}{super}\PYG{p}{(}\PYG{n}{NameDirective}\PYG{p}{,} \PYG{n+nb+bp}{self}\PYG{p}{)}\PYG{o}{.}\PYG{n}{\PYGZus{}\PYGZus{}init\PYGZus{}\PYGZus{}}\PYG{p}{(}\PYG{n}{line}\PYG{p}{,} \PYG{n}{mo}\PYG{p}{)}
        \PYG{k}{if} \PYG{n+nb}{isinstance}\PYG{p}{(}\PYG{n}{mo}\PYG{p}{,} \PYG{n+nb}{str}\PYG{p}{)}\PYG{p}{:}
            \PYG{n+nb+bp}{self}\PYG{o}{.}\PYG{n}{name} \PYG{o}{=} \PYG{n}{mo}
        \PYG{k}{else}\PYG{p}{:}
            \PYG{n+nb+bp}{self}\PYG{o}{.}\PYG{n}{name} \PYG{o}{=} \PYG{n}{mo}\PYG{o}{.}\PYG{n}{group}\PYG{p}{(}\PYG{l+m+mi}{1}\PYG{p}{)}


    \PYG{k}{def} \PYG{n+nf}{\PYGZus{}\PYGZus{}repr\PYGZus{}\PYGZus{}}\PYG{p}{(}\PYG{n+nb+bp}{self}\PYG{p}{)}\PYG{p}{:}
        \PYG{k}{return} \PYG{l+s+s2}{\PYGZdq{}}\PYG{l+s+s2}{\PYGZlt{}}\PYG{l+s+si}{\PYGZpc{}s}\PYG{l+s+s2}{(}\PYG{l+s+si}{\PYGZpc{}s}\PYG{l+s+s2}{) }\PYG{l+s+si}{\PYGZpc{}i}\PYG{l+s+s2}{\PYGZgt{}}\PYG{l+s+s2}{\PYGZdq{}} \PYG{o}{\PYGZpc{}} \PYG{p}{(}\PYG{n+nb+bp}{self}\PYG{o}{.}\PYG{n}{\PYGZus{}\PYGZus{}class\PYGZus{}\PYGZus{}}\PYG{o}{.}\PYG{n}{\PYGZus{}\PYGZus{}name\PYGZus{}\PYGZus{}}\PYG{p}{,}
                                \PYG{n+nb+bp}{self}\PYG{o}{.}\PYG{n}{name}\PYG{p}{,} \PYG{n+nb+bp}{self}\PYG{o}{.}\PYG{n}{line}\PYG{p}{)}
\end{Verbatim}

\end{fulllineitems}



\subsection{Start}
\label{antiweb:start}\index{Start (built-in class)}

\begin{fulllineitems}
\phantomsection\label{antiweb:Start}\pysigline{\sphinxstrong{class }\sphinxbfcode{Start}}
This class represents a \sphinxcode{@start} directive. It inherits
{\hyperref[antiweb:NameDirective]{\sphinxcrossref{\sphinxcode{NameDirective}}}}.

The \sphinxcode{@start} directive defines the beginning of
a text block. It is called with an argument defining
the name of the text block. There are two special text
blocks:
\begin{itemize}
\item {} 
\sphinxcode{()} The empty one defining the main text block

\item {} 
\sphinxcode{(\_\_macro\_\_)} defining a text block for implementing macros.

\end{itemize}

There are several possibilities to end a text block.
\begin{enumerate}
\item {} 
The end of the file

\item {} 
A line with a smaller indentation as the \sphinxcode{@start} directive.

\item {} 
Another start directive with same indentation.

\item {} 
An unnamed end (\sphinxcode{@}) directive with the same indentation as
the \sphinxcode{@start} directive.

\item {} 
A named end directive closing this block or an outer block.

\end{enumerate}

Text blocks defined by \sphinxcode{@start} can be nested.

\begin{Verbatim}[commandchars=\\\{\},numbers=left,firstnumber=1,stepnumber=1]
\PYG{k}{class} \PYG{n+nc}{Start}\PYG{p}{(}\PYG{n}{NameDirective}\PYG{p}{)}\PYG{p}{:}
    \PYG{c+c1}{\PYGZsh{}Attributes}
    \PYG{o}{\PYGZlt{}\PYGZlt{}}\PYG{n}{Start}\PYG{o}{.}\PYG{n}{has\PYGZus{}named\PYGZus{}end}\PYG{o}{\PYGZgt{}\PYGZgt{}}
    \PYG{o}{\PYGZlt{}\PYGZlt{}}\PYG{n}{Start}\PYG{o}{.}\PYG{n}{inherited} \PYG{n}{attributes}\PYG{o}{\PYGZgt{}\PYGZgt{}}

    \PYG{c+c1}{\PYGZsh{}Methods}
    \PYG{o}{\PYGZlt{}\PYGZlt{}}\PYG{n}{Start}\PYG{o}{.}\PYG{n}{\PYGZus{}find\PYGZus{}matching\PYGZus{}end}\PYG{o}{\PYGZgt{}\PYGZgt{}}
    \PYG{o}{\PYGZlt{}\PYGZlt{}}\PYG{n}{Start}\PYG{o}{.}\PYG{n}{collect\PYGZus{}block}\PYG{o}{\PYGZgt{}\PYGZgt{}}
    \PYG{o}{\PYGZlt{}\PYGZlt{}}\PYG{n}{Start}\PYG{o}{.}\PYG{n}{process}\PYG{o}{\PYGZgt{}\PYGZgt{}}
\end{Verbatim}
\index{has\_named\_end (Start attribute)}

\begin{fulllineitems}
\phantomsection\label{antiweb:Start.has_named_end}\pysigline{\sphinxbfcode{has\_named\_end}}
A boolean value, signalizing if the directive is
ended by a named end directive.

\begin{Verbatim}[commandchars=\\\{\}]
\PYG{n}{has\PYGZus{}named\PYGZus{}end} \PYG{o}{=} \PYG{n+nb+bp}{False}
\end{Verbatim}

\end{fulllineitems}

\phantomsection\label{antiweb:start-inherited-attributes}
\textbf{\textless{}\textless{}Start.inherited attributes\textgreater{}\textgreater{}}

\begin{Verbatim}[commandchars=\\\{\}]
\PYG{n}{expression} \PYG{o}{=} \PYG{n}{re}\PYG{o}{.}\PYG{n}{compile}\PYG{p}{(}\PYG{l+s+s2}{r\PYGZdq{}}\PYG{l+s+s2}{@start}\PYG{l+s+s2}{\PYGZbs{}}\PYG{l+s+s2}{((.*)}\PYG{l+s+s2}{\PYGZbs{}}\PYG{l+s+s2}{)}\PYG{l+s+s2}{\PYGZdq{}}\PYG{p}{)}
\PYG{n}{priority} \PYG{o}{=} \PYG{l+m+mi}{5}
\end{Verbatim}
\index{collect\_block() (Start method)}

\begin{fulllineitems}
\phantomsection\label{antiweb:Start.collect_block}\pysiglinewithargsret{\sphinxbfcode{collect\_block}}{\emph{document}, \emph{index}}{}
See {\hyperref[antiweb:Directive.collect_block]{\sphinxcrossref{\sphinxcode{Directive.collect\_block()}}}}.
The returned lines are unindented to column 0.

\begin{Verbatim}[commandchars=\\\{\},numbers=left,firstnumber=1,stepnumber=1]
\PYG{k}{def} \PYG{n+nf}{collect\PYGZus{}block}\PYG{p}{(}\PYG{n+nb+bp}{self}\PYG{p}{,} \PYG{n}{document}\PYG{p}{,} \PYG{n}{index}\PYG{p}{)}\PYG{p}{:}
    \PYG{n}{end} \PYG{o}{=} \PYG{n+nb+bp}{self}\PYG{o}{.}\PYG{n}{\PYGZus{}find\PYGZus{}matching\PYGZus{}end}\PYG{p}{(}\PYG{n}{document}\PYG{o}{.}\PYG{n}{lines}\PYG{p}{[}\PYG{n}{index}\PYG{p}{:}\PYG{p}{]}\PYG{p}{)}
    \PYG{n}{block} \PYG{o}{=} \PYG{n}{document}\PYG{o}{.}\PYG{n}{lines}\PYG{p}{[}\PYG{n}{index}\PYG{o}{+}\PYG{l+m+mi}{1}\PYG{p}{:}\PYG{n}{index}\PYG{o}{+}\PYG{n}{end}\PYG{p}{]}

    \PYG{n}{reduce\PYGZus{}block} \PYG{o}{=} \PYG{n+nb}{list}\PYG{p}{(}\PYG{n+nb}{filter}\PYG{p}{(}\PYG{n+nb}{bool}\PYG{p}{,} \PYG{n}{block}\PYG{p}{)}\PYG{p}{)}
    \PYG{k}{if} \PYG{o+ow}{not} \PYG{n}{reduce\PYGZus{}block}\PYG{p}{:}
        \PYG{n}{document}\PYG{o}{.}\PYG{n}{add\PYGZus{}error}\PYG{p}{(}\PYG{n+nb+bp}{self}\PYG{o}{.}\PYG{n}{line}\PYG{p}{,} \PYG{l+s+s2}{\PYGZdq{}}\PYG{l+s+s2}{Empty }\PYG{l+s+s2}{\PYGZsq{}}\PYG{l+s+si}{\PYGZpc{}s}\PYG{l+s+s2}{\PYGZsq{}}\PYG{l+s+s2}{ block}\PYG{l+s+s2}{\PYGZdq{}} \PYG{o}{\PYGZpc{}} \PYG{n+nb+bp}{self}\PYG{o}{.}\PYG{n}{name}\PYG{p}{)}
        \PYG{k}{return} \PYG{n+nb+bp}{None}

    \PYG{c+c1}{\PYGZsh{}unindent the block, empty lines may not count (filter(bool, block))}
    \PYG{n}{indent\PYGZus{}getter} \PYG{o}{=} \PYG{n}{operator}\PYG{o}{.}\PYG{n}{attrgetter}\PYG{p}{(}\PYG{l+s+s2}{\PYGZdq{}}\PYG{l+s+s2}{indent}\PYG{l+s+s2}{\PYGZdq{}}\PYG{p}{)}
    \PYG{n}{min\PYGZus{}indent} \PYG{o}{=} \PYG{n+nb}{min}\PYG{p}{(}\PYG{n+nb}{list}\PYG{p}{(}\PYG{n+nb}{map}\PYG{p}{(}\PYG{n}{indent\PYGZus{}getter}\PYG{p}{,} \PYG{n}{reduce\PYGZus{}block}\PYG{p}{)}\PYG{p}{)}\PYG{p}{)}
    \PYG{n}{block} \PYG{o}{=} \PYG{p}{[} \PYG{n}{l}\PYG{o}{.}\PYG{n}{clone}\PYG{p}{(}\PYG{p}{)}\PYG{o}{.}\PYG{n}{change\PYGZus{}indent}\PYG{p}{(}\PYG{o}{\PYGZhy{}}\PYG{n}{min\PYGZus{}indent}\PYG{p}{)} \PYG{k}{for} \PYG{n}{l} \PYG{o+ow}{in} \PYG{n}{block} \PYG{p}{]}
    \PYG{k}{return} \PYG{n+nb+bp}{self}\PYG{o}{.}\PYG{n}{name}\PYG{p}{,} \PYG{n}{block}
\end{Verbatim}

\end{fulllineitems}

\index{process() (Start method)}

\begin{fulllineitems}
\phantomsection\label{antiweb:Start.process}\pysiglinewithargsret{\sphinxbfcode{process}}{\emph{document}, \emph{block}, \emph{index}}{}
See {\hyperref[antiweb:Directive.process]{\sphinxcrossref{\sphinxcode{Directive.process()}}}}.
Removes all lines of the text block from
the containing block.

\begin{Verbatim}[commandchars=\\\{\}]
\PYG{k}{def} \PYG{n+nf}{process}\PYG{p}{(}\PYG{n+nb+bp}{self}\PYG{p}{,} \PYG{n}{document}\PYG{p}{,} \PYG{n}{block}\PYG{p}{,} \PYG{n}{index}\PYG{p}{)}\PYG{p}{:}
    \PYG{n}{end} \PYG{o}{=} \PYG{n+nb+bp}{self}\PYG{o}{.}\PYG{n}{\PYGZus{}find\PYGZus{}matching\PYGZus{}end}\PYG{p}{(}\PYG{n}{block}\PYG{p}{[}\PYG{n}{index}\PYG{p}{:}\PYG{p}{]}\PYG{p}{)}
    \PYG{k}{del} \PYG{n}{block}\PYG{p}{[}\PYG{n}{index}\PYG{p}{:}\PYG{n}{index}\PYG{o}{+}\PYG{n}{end}\PYG{p}{]}
\end{Verbatim}

\end{fulllineitems}

\index{\_find\_matching\_end() (Start method)}

\begin{fulllineitems}
\phantomsection\label{antiweb:Start._find_matching_end}\pysiglinewithargsret{\sphinxbfcode{\_find\_matching\_end}}{\emph{block}}{}
Finds the matching end for the text block.
\begin{quote}\begin{description}
\item[{Parameters}] \leavevmode
\textbf{\texttt{block}} (\emph{\texttt{list}}) -- A list of lines beginning with start

\item[{Returns}] \leavevmode
The line index of the found end.

\end{description}\end{quote}

\begin{Verbatim}[commandchars=\\\{\},numbers=left,firstnumber=1,stepnumber=1]
\PYG{k}{def} \PYG{n+nf}{\PYGZus{}find\PYGZus{}matching\PYGZus{}end}\PYG{p}{(}\PYG{n+nb+bp}{self}\PYG{p}{,} \PYG{n}{block}\PYG{p}{)}\PYG{p}{:}
    \PYG{k}{if} \PYG{n+nb+bp}{self}\PYG{o}{.}\PYG{n}{has\PYGZus{}named\PYGZus{}end}\PYG{p}{:}
        \PYG{c+c1}{\PYGZsh{} ignore all other ending conditions and directly}
        \PYG{c+c1}{\PYGZsh{} find the matching end directive}
        \PYG{k}{for} \PYG{n}{j}\PYG{p}{,} \PYG{n}{l} \PYG{o+ow}{in} \PYG{n+nb}{enumerate}\PYG{p}{(}\PYG{n}{block}\PYG{p}{[}\PYG{l+m+mi}{1}\PYG{p}{:}\PYG{p}{]}\PYG{p}{)}\PYG{p}{:}
            \PYG{n}{j} \PYG{o}{+}\PYG{o}{=} \PYG{l+m+mi}{1}
            \PYG{n}{d} \PYG{o}{=} \PYG{n}{l}\PYG{o}{.}\PYG{n}{directive}
            \PYG{k}{if} \PYG{n+nb}{isinstance}\PYG{p}{(}\PYG{n}{d}\PYG{p}{,} \PYG{n}{End}\PYG{p}{)} \PYG{o+ow}{and} \PYG{n}{d}\PYG{o}{.}\PYG{n}{name} \PYG{o}{==} \PYG{n+nb+bp}{self}\PYG{o}{.}\PYG{n}{name}\PYG{p}{:}
                \PYG{k}{return} \PYG{n}{j}

    \PYG{n}{start\PYGZus{}indent} \PYG{o}{=} \PYG{n}{block}\PYG{p}{[}\PYG{l+m+mi}{0}\PYG{p}{]}\PYG{o}{.}\PYG{n}{indent}
    \PYG{k}{for} \PYG{n}{j}\PYG{p}{,} \PYG{n}{l} \PYG{o+ow}{in} \PYG{n+nb}{enumerate}\PYG{p}{(}\PYG{n}{block}\PYG{p}{[}\PYG{l+m+mi}{1}\PYG{p}{:}\PYG{p}{]}\PYG{p}{)}\PYG{p}{:}
        \PYG{n}{j} \PYG{o}{+}\PYG{o}{=} \PYG{l+m+mi}{1}

        \PYG{n}{lindent} \PYG{o}{=} \PYG{n}{l}\PYG{o}{.}\PYG{n}{indent}
        \PYG{n}{d} \PYG{o}{=} \PYG{n}{l}\PYG{o}{.}\PYG{n}{directive}

        \PYG{k}{if} \PYG{n+nb}{isinstance}\PYG{p}{(}\PYG{n}{d}\PYG{p}{,} \PYG{n}{End}\PYG{p}{)}\PYG{p}{:}
            \PYG{k}{if} \PYG{n}{d}\PYG{o}{.}\PYG{n}{name} \PYG{o+ow}{is} \PYG{n+nb+bp}{None} \PYG{o+ow}{and} \PYG{n}{lindent} \PYG{o}{==} \PYG{n}{start\PYGZus{}indent}\PYG{p}{:}
                \PYG{c+c1}{\PYGZsh{}case 4: An unnamed @ directive with the same indentation}
                \PYG{c+c1}{\PYGZsh{}        as the @start directive.}
                \PYG{k}{return} \PYG{n}{j}

            \PYG{k}{if} \PYG{n}{d}\PYG{o}{.}\PYG{n}{start\PYGZus{}line} \PYG{o}{\PYGZlt{}}\PYG{o}{=} \PYG{n+nb+bp}{self}\PYG{o}{.}\PYG{n}{line}\PYG{p}{:}
                \PYG{c+c1}{\PYGZsh{}case 5: A named @ directive closing this block}
                \PYG{c+c1}{\PYGZsh{}        or an outer block.}
                \PYG{k}{return} \PYG{n}{j}

        \PYG{k}{if} \PYG{n+nb}{isinstance}\PYG{p}{(}\PYG{n}{d}\PYG{p}{,} \PYG{n}{Start}\PYG{p}{)} \PYG{o+ow}{and} \PYG{n}{lindent} \PYG{o}{==} \PYG{n}{start\PYGZus{}indent}\PYG{p}{:}
            \PYG{c+c1}{\PYGZsh{}case 3: Another @start directive with same indentation.}
            \PYG{k}{return} \PYG{n}{j}

        \PYG{k}{if} \PYG{n}{lindent} \PYG{o}{\PYGZlt{}} \PYG{n}{start\PYGZus{}indent} \PYG{o+ow}{and} \PYG{n}{l}\PYG{p}{:}
            \PYG{c+c1}{\PYGZsh{}case 2: A line with a smaller indentation as the @start directive.}
            \PYG{c+c1}{\PYGZsh{}        (an empty line doesn\PYGZsq{}t count)}
            \PYG{k}{return} \PYG{n}{j}

    \PYG{c+c1}{\PYGZsh{}case 1: The end of the file}
    \PYG{k}{return} \PYG{n+nb}{len}\PYG{p}{(}\PYG{n}{block}\PYG{p}{)}
\end{Verbatim}

\end{fulllineitems}


\end{fulllineitems}



\subsection{RStart}
\label{antiweb:rstart}\index{RStart (built-in class)}

\begin{fulllineitems}
\phantomsection\label{antiweb:RStart}\pysigline{\sphinxstrong{class }\sphinxbfcode{RStart}}
This class represents a \sphinxcode{@rstart} directive. It inherits
{\hyperref[antiweb:Start]{\sphinxcrossref{\sphinxcode{Start}}}}.

The \sphinxcode{@rstart} directive works like the \sphinxcode{@start}
directive. While \sphinxcode{@start} removes it's block completely
from the containing block. \sphinxcode{@rstart} replaces the lines
with a \sphinxcode{\textless{}\textless{}name\textgreater{}\textgreater{}} - Sentinel.

\begin{Verbatim}[commandchars=\\\{\},numbers=left,firstnumber=1,stepnumber=1]
\PYG{k}{class} \PYG{n+nc}{RStart}\PYG{p}{(}\PYG{n}{Start}\PYG{p}{)}\PYG{p}{:}
    \PYG{n}{expression} \PYG{o}{=} \PYG{n}{re}\PYG{o}{.}\PYG{n}{compile}\PYG{p}{(}\PYG{l+s+s2}{r\PYGZdq{}}\PYG{l+s+s2}{@rstart}\PYG{l+s+s2}{\PYGZbs{}}\PYG{l+s+s2}{((.*)}\PYG{l+s+s2}{\PYGZbs{}}\PYG{l+s+s2}{)}\PYG{l+s+s2}{\PYGZdq{}}\PYG{p}{)}

    \PYG{k}{def} \PYG{n+nf}{process}\PYG{p}{(}\PYG{n+nb+bp}{self}\PYG{p}{,} \PYG{n}{document}\PYG{p}{,} \PYG{n}{block}\PYG{p}{,} \PYG{n}{index}\PYG{p}{)}\PYG{p}{:}
        \PYG{n}{end} \PYG{o}{=} \PYG{n+nb+bp}{self}\PYG{o}{.}\PYG{n}{\PYGZus{}find\PYGZus{}matching\PYGZus{}end}\PYG{p}{(}\PYG{n}{block}\PYG{p}{[}\PYG{n}{index}\PYG{p}{:}\PYG{p}{]}\PYG{p}{)}
        \PYG{n}{line} \PYG{o}{=} \PYG{n}{block}\PYG{p}{[}\PYG{n}{index}\PYG{p}{]}
        \PYG{n}{block}\PYG{p}{[}\PYG{n}{index}\PYG{p}{:}\PYG{n}{index}\PYG{o}{+}\PYG{n}{end}\PYG{p}{]} \PYG{o}{=} \PYG{p}{[} \PYG{n}{line}\PYG{o}{.}\PYG{n}{like}\PYG{p}{(}\PYG{l+s+s2}{\PYGZdq{}}\PYG{l+s+s2}{\PYGZlt{}\PYGZlt{}}\PYG{l+s+si}{\PYGZpc{}s}\PYG{l+s+s2}{\PYGZgt{}\PYGZgt{}}\PYG{l+s+s2}{\PYGZdq{}} \PYG{o}{\PYGZpc{}} \PYG{n+nb+bp}{self}\PYG{o}{.}\PYG{n}{name}\PYG{p}{)} \PYG{p}{]}
\end{Verbatim}

\end{fulllineitems}



\subsection{CStart}
\label{antiweb:cstart}\index{CStart (built-in class)}

\begin{fulllineitems}
\phantomsection\label{antiweb:CStart}\pysigline{\sphinxstrong{class }\sphinxbfcode{CStart}}
This class represents a \sphinxcode{@rstart} directive. It inherits
{\hyperref[antiweb:RStart]{\sphinxcrossref{\sphinxcode{RStart}}}}.

The \sphinxcode{@cstart(name)} directive is a replacement for

\begin{Verbatim}[commandchars=\\\{\}]
\PYG{n+nd}{@rstart}\PYG{p}{(}\PYG{n}{name}\PYG{p}{)}
\PYG{n+nd}{@code}
\end{Verbatim}

\begin{Verbatim}[commandchars=\\\{\},numbers=left,firstnumber=1,stepnumber=1]
\PYG{k}{class} \PYG{n+nc}{CStart}\PYG{p}{(}\PYG{n}{RStart}\PYG{p}{)}\PYG{p}{:}
    \PYG{n}{expression} \PYG{o}{=} \PYG{n}{re}\PYG{o}{.}\PYG{n}{compile}\PYG{p}{(}\PYG{l+s+s2}{r\PYGZdq{}}\PYG{l+s+s2}{@cstart}\PYG{l+s+s2}{\PYGZbs{}}\PYG{l+s+s2}{((.*)}\PYG{l+s+s2}{\PYGZbs{}}\PYG{l+s+s2}{)}\PYG{l+s+s2}{\PYGZdq{}}\PYG{p}{)}

    \PYG{k}{def} \PYG{n+nf}{collect\PYGZus{}block}\PYG{p}{(}\PYG{n+nb+bp}{self}\PYG{p}{,} \PYG{n}{document}\PYG{p}{,} \PYG{n}{index}\PYG{p}{)}\PYG{p}{:}
        \PYG{n}{name\PYGZus{}block} \PYG{o}{=} \PYG{n+nb}{super}\PYG{p}{(}\PYG{n}{CStart}\PYG{p}{,} \PYG{n+nb+bp}{self}\PYG{p}{)}\PYG{o}{.}\PYG{n}{collect\PYGZus{}block}\PYG{p}{(}\PYG{n}{document}\PYG{p}{,} \PYG{n}{index}\PYG{p}{)}

        \PYG{k}{if} \PYG{o+ow}{not} \PYG{n}{name\PYGZus{}block}\PYG{p}{:} \PYG{k}{return} \PYG{n+nb+bp}{None}

        \PYG{n}{name}\PYG{p}{,} \PYG{n}{block} \PYG{o}{=} \PYG{n}{name\PYGZus{}block}

        \PYG{n}{first} \PYG{o}{=} \PYG{n}{block}\PYG{p}{[}\PYG{l+m+mi}{0}\PYG{p}{]}
        \PYG{n}{sd} \PYG{o}{=} \PYG{p}{[} \PYG{n}{Code}\PYG{p}{(}\PYG{n}{first}\PYG{o}{.}\PYG{n}{index}\PYG{p}{)} \PYG{p}{]}
        \PYG{n}{block}\PYG{o}{.}\PYG{n}{insert}\PYG{p}{(}\PYG{l+m+mi}{0}\PYG{p}{,} \PYG{n}{first}\PYG{o}{.}\PYG{n}{like}\PYG{p}{(}\PYG{l+s+s2}{\PYGZdq{}}\PYG{l+s+s2}{@code}\PYG{l+s+s2}{\PYGZdq{}}\PYG{p}{)}\PYG{o}{.}\PYG{n}{set}\PYG{p}{(}\PYG{n}{directives}\PYG{o}{=}\PYG{n}{sd}\PYG{p}{,}
                                                \PYG{n}{index}\PYG{o}{=}\PYG{n}{first}\PYG{o}{.}\PYG{n}{index}\PYG{o}{\PYGZhy{}}\PYG{l+m+mi}{1}\PYG{p}{)}\PYG{p}{)}

        \PYG{k}{return} \PYG{n}{name}\PYG{p}{,} \PYG{n}{block}
\end{Verbatim}

\end{fulllineitems}



\subsection{End}
\label{antiweb:end}\index{End (built-in class)}

\begin{fulllineitems}
\phantomsection\label{antiweb:End}\pysigline{\sphinxstrong{class }\sphinxbfcode{End}}
This class represents an end directive. It inherits
{\hyperref[antiweb:NameDirective]{\sphinxcrossref{\sphinxcode{NameDirective}}}}.

The end (\sphinxcode{@}) directive ends a text block.

\begin{Verbatim}[commandchars=\\\{\},numbers=left,firstnumber=1,stepnumber=1]
\PYG{k}{class} \PYG{n+nc}{End}\PYG{p}{(}\PYG{n}{NameDirective}\PYG{p}{)}\PYG{p}{:}
    \PYG{n}{expression} \PYG{o}{=} \PYG{n}{re}\PYG{o}{.}\PYG{n}{compile}\PYG{p}{(}\PYG{l+s+s2}{r\PYGZdq{}}\PYG{l+s+s2}{@(}\PYG{l+s+s2}{\PYGZbs{}}\PYG{l+s+s2}{((.*)}\PYG{l+s+s2}{\PYGZbs{}}\PYG{l+s+s2}{))?}\PYG{l+s+s2}{\PYGZbs{}}\PYG{l+s+s2}{s*\PYGZdl{}}\PYG{l+s+s2}{\PYGZdq{}}\PYG{p}{,} \PYG{n}{re}\PYG{o}{.}\PYG{n}{M}\PYG{p}{)}

    \PYG{k}{def} \PYG{n+nf}{\PYGZus{}\PYGZus{}init\PYGZus{}\PYGZus{}}\PYG{p}{(}\PYG{n+nb+bp}{self}\PYG{p}{,} \PYG{n}{line}\PYG{p}{,} \PYG{n}{mo}\PYG{p}{)}\PYG{p}{:}
        \PYG{n+nb}{super}\PYG{p}{(}\PYG{n}{NameDirective}\PYG{p}{,} \PYG{n+nb+bp}{self}\PYG{p}{)}\PYG{o}{.}\PYG{n}{\PYGZus{}\PYGZus{}init\PYGZus{}\PYGZus{}}\PYG{p}{(}\PYG{n}{line}\PYG{p}{,} \PYG{n}{mo}\PYG{p}{)}
        \PYG{n+nb+bp}{self}\PYG{o}{.}\PYG{n}{start\PYGZus{}line} \PYG{o}{=} \PYG{n+nb+bp}{self}\PYG{o}{.}\PYG{n}{line}

        \PYG{k}{if} \PYG{n+nb}{isinstance}\PYG{p}{(}\PYG{n}{mo}\PYG{p}{,} \PYG{n+nb}{str}\PYG{p}{)}\PYG{p}{:}
            \PYG{n+nb+bp}{self}\PYG{o}{.}\PYG{n}{name} \PYG{o}{=} \PYG{n}{mo}
        \PYG{k}{else}\PYG{p}{:}
            \PYG{n+nb+bp}{self}\PYG{o}{.}\PYG{n}{name} \PYG{o}{=} \PYG{n}{mo}\PYG{o}{.}\PYG{n}{group}\PYG{p}{(}\PYG{l+m+mi}{2}\PYG{p}{)}


    \PYG{k}{def} \PYG{n+nf}{match}\PYG{p}{(}\PYG{n+nb+bp}{self}\PYG{p}{,} \PYG{n}{lines}\PYG{p}{)}\PYG{p}{:}
        \PYG{k}{if} \PYG{n+nb+bp}{self}\PYG{o}{.}\PYG{n}{name} \PYG{o+ow}{is} \PYG{n+nb+bp}{None}\PYG{p}{:} \PYG{k}{return}

        \PYG{c+c1}{\PYGZsh{}find the matching start and inform it for the named end}
        \PYG{k}{for} \PYG{n}{l} \PYG{o+ow}{in} \PYG{n+nb}{reversed}\PYG{p}{(}\PYG{n}{lines}\PYG{p}{[}\PYG{p}{:}\PYG{n+nb+bp}{self}\PYG{o}{.}\PYG{n}{line}\PYG{p}{]}\PYG{p}{)}\PYG{p}{:}
            \PYG{k}{for} \PYG{n}{d} \PYG{o+ow}{in} \PYG{n}{l}\PYG{o}{.}\PYG{n}{directives}\PYG{p}{:}
                \PYG{k}{if} \PYG{n+nb}{isinstance}\PYG{p}{(}\PYG{n}{d}\PYG{p}{,} \PYG{n}{Start}\PYG{p}{)} \PYG{o+ow}{and} \PYG{n}{d}\PYG{o}{.}\PYG{n}{name} \PYG{o}{==} \PYG{n+nb+bp}{self}\PYG{o}{.}\PYG{n}{name}\PYG{p}{:}
                    \PYG{n}{d}\PYG{o}{.}\PYG{n}{has\PYGZus{}named\PYGZus{}end} \PYG{o}{=} \PYG{n+nb+bp}{True}
                    \PYG{n+nb+bp}{self}\PYG{o}{.}\PYG{n}{start\PYGZus{}line} \PYG{o}{=} \PYG{n}{d}\PYG{o}{.}\PYG{n}{line}
                    \PYG{k}{return}


    \PYG{k}{def} \PYG{n+nf}{process}\PYG{p}{(}\PYG{n+nb+bp}{self}\PYG{p}{,} \PYG{n}{document}\PYG{p}{,} \PYG{n}{block}\PYG{p}{,} \PYG{n}{index}\PYG{p}{)}\PYG{p}{:}
        \PYG{c+c1}{\PYGZsh{}completely remove the directive from the containing block}
        \PYG{k}{del} \PYG{n}{block}\PYG{p}{[}\PYG{n}{index}\PYG{p}{]}
\end{Verbatim}

\end{fulllineitems}



\subsection{Include}
\label{antiweb:include}\index{Include (built-in class)}

\begin{fulllineitems}
\phantomsection\label{antiweb:Include}\pysigline{\sphinxstrong{class }\sphinxbfcode{Include}}
This class represents an \sphinxcode{@include} directive. It inherits
{\hyperref[antiweb:NameDirective]{\sphinxcrossref{\sphinxcode{NameDirective}}}}.

The \sphinxcode{@include} directive inserts the contents of the
text block with the same name. The lines have the same
indentation as the \sphinxcode{@include} directive.

The directive can have a second \emph{file} argument. If given
the directive inserts the text block of the specified file.

\begin{Verbatim}[commandchars=\\\{\},numbers=left,firstnumber=1,stepnumber=1]
\PYG{k}{class} \PYG{n+nc}{Include}\PYG{p}{(}\PYG{n}{NameDirective}\PYG{p}{)}\PYG{p}{:}
    \PYG{n}{expression} \PYG{o}{=} \PYG{n}{re}\PYG{o}{.}\PYG{n}{compile}\PYG{p}{(}\PYG{l+s+s2}{r\PYGZdq{}}\PYG{l+s+s2}{@include}\PYG{l+s+s2}{\PYGZbs{}}\PYG{l+s+s2}{((.+)}\PYG{l+s+s2}{\PYGZbs{}}\PYG{l+s+s2}{)}\PYG{l+s+s2}{\PYGZdq{}}\PYG{p}{)}


    \PYG{k}{def} \PYG{n+nf}{process}\PYG{p}{(}\PYG{n+nb+bp}{self}\PYG{p}{,} \PYG{n}{document}\PYG{p}{,} \PYG{n}{block}\PYG{p}{,} \PYG{n}{index}\PYG{p}{)}\PYG{p}{:}
        \PYG{c+c1}{\PYGZsh{}check if the name contains 2 arguments}
        \PYG{n}{args} \PYG{o}{=} \PYG{n+nb+bp}{self}\PYG{o}{.}\PYG{n}{name}\PYG{o}{.}\PYG{n}{split}\PYG{p}{(}\PYG{l+s+s2}{\PYGZdq{}}\PYG{l+s+s2}{,}\PYG{l+s+s2}{\PYGZdq{}}\PYG{p}{)}
        \PYG{n}{name} \PYG{o}{=} \PYG{n}{args}\PYG{o}{.}\PYG{n}{pop}\PYG{p}{(}\PYG{l+m+mi}{0}\PYG{p}{)}\PYG{o}{.}\PYG{n}{strip}\PYG{p}{(}\PYG{p}{)}

        \PYG{n}{document}\PYG{o}{.}\PYG{n}{blocks\PYGZus{}included}\PYG{o}{.}\PYG{n}{add}\PYG{p}{(}\PYG{n}{name}\PYG{p}{)}

        \PYG{k}{if} \PYG{n}{args}\PYG{p}{:}
            \PYG{c+c1}{\PYGZsh{}a file name is given, fetch block from that file}
            \PYG{n}{fname} \PYG{o}{=} \PYG{n}{args}\PYG{p}{[}\PYG{l+m+mi}{0}\PYG{p}{]}\PYG{o}{.}\PYG{n}{strip}\PYG{p}{(}\PYG{p}{)}
            \PYG{n}{subdoc} \PYG{o}{=} \PYG{n}{document}\PYG{o}{.}\PYG{n}{get\PYGZus{}subdoc}\PYG{p}{(}\PYG{n}{fname}\PYG{p}{)}
            \PYG{k}{if} \PYG{n}{subdoc}\PYG{p}{:}
                \PYG{n}{include} \PYG{o}{=} \PYG{n}{subdoc}\PYG{o}{.}\PYG{n}{get\PYGZus{}compiled\PYGZus{}block}\PYG{p}{(}\PYG{n}{name}\PYG{p}{)}
            \PYG{k}{else}\PYG{p}{:}
                \PYG{n}{include} \PYG{o}{=} \PYG{n+nb+bp}{None}
        \PYG{k}{else}\PYG{p}{:}
            \PYG{n}{include} \PYG{o}{=} \PYG{n}{document}\PYG{o}{.}\PYG{n}{get\PYGZus{}compiled\PYGZus{}block}\PYG{p}{(}\PYG{n}{name}\PYG{p}{)}

        \PYG{k}{if} \PYG{o+ow}{not} \PYG{n}{include}\PYG{p}{:}
            \PYG{c+c1}{\PYGZsh{}print \PYGZdq{}error include\PYGZdq{}, self.line, name}
            \PYG{n}{document}\PYG{o}{.}\PYG{n}{add\PYGZus{}error}\PYG{p}{(}\PYG{n+nb+bp}{self}\PYG{o}{.}\PYG{n}{line}\PYG{p}{,}
                               \PYG{l+s+s2}{\PYGZdq{}}\PYG{l+s+s2}{Cannot find text block: }\PYG{l+s+si}{\PYGZpc{}s}\PYG{l+s+s2}{\PYGZdq{}} \PYG{o}{\PYGZpc{}} \PYG{n}{name}\PYG{p}{)}
            \PYG{k}{return}

        \PYG{c+c1}{\PYGZsh{}replace the directive with its content}
        \PYG{n}{indent} \PYG{o}{=} \PYG{n}{block}\PYG{p}{[}\PYG{n}{index}\PYG{p}{]}\PYG{o}{.}\PYG{n}{indent}
        \PYG{n}{include} \PYG{o}{=} \PYG{p}{[} \PYG{n}{l}\PYG{o}{.}\PYG{n}{clone}\PYG{p}{(}\PYG{p}{)}\PYG{o}{.}\PYG{n}{change\PYGZus{}indent}\PYG{p}{(}\PYG{n}{indent}\PYG{p}{)} \PYG{k}{for} \PYG{n}{l} \PYG{o+ow}{in} \PYG{n}{include} \PYG{p}{]}
        \PYG{n}{block}\PYG{p}{[}\PYG{n}{index}\PYG{p}{:}\PYG{n}{index}\PYG{o}{+}\PYG{l+m+mi}{1}\PYG{p}{]} \PYG{o}{=} \PYG{n}{include}
\end{Verbatim}

\end{fulllineitems}



\subsection{RInclude}
\label{antiweb:rinclude}\index{RInclude (built-in class)}

\begin{fulllineitems}
\phantomsection\label{antiweb:RInclude}\pysigline{\sphinxstrong{class }\sphinxbfcode{RInclude}}
This class represents an \sphinxcode{@rinclude} directive. It inherits
{\hyperref[antiweb:Include]{\sphinxcrossref{\sphinxcode{Include}}}}.

The \sphinxcode{@rinclude(text block name)} directive is a is a replacement for:

\begin{Verbatim}[commandchars=\\\{\}]
\PYG{o}{.}\PYG{o}{.} \PYG{n}{\PYGZus{}text} \PYG{n}{block} \PYG{n}{name}\PYG{p}{:}

\PYG{o}{*}\PYG{o}{*}\PYG{o}{\PYGZlt{}\PYGZlt{}}\PYG{n}{text} \PYG{n}{block} \PYG{n}{name}\PYG{o}{\PYGZgt{}\PYGZgt{}}\PYG{o}{*}\PYG{o}{*}

\PYG{n+nd}{@include}\PYG{p}{(}\PYG{n}{text} \PYG{n}{block} \PYG{n}{name}\PYG{p}{)}
\end{Verbatim}

\begin{Verbatim}[commandchars=\\\{\},numbers=left,firstnumber=1,stepnumber=1]
\PYG{k}{class} \PYG{n+nc}{RInclude}\PYG{p}{(}\PYG{n}{Include}\PYG{p}{)}\PYG{p}{:}
    \PYG{n}{expression} \PYG{o}{=} \PYG{n}{re}\PYG{o}{.}\PYG{n}{compile}\PYG{p}{(}\PYG{l+s+s2}{r\PYGZdq{}}\PYG{l+s+s2}{@rinclude}\PYG{l+s+s2}{\PYGZbs{}}\PYG{l+s+s2}{((.+)}\PYG{l+s+s2}{\PYGZbs{}}\PYG{l+s+s2}{)}\PYG{l+s+s2}{\PYGZdq{}}\PYG{p}{)}

    \PYG{k}{def} \PYG{n+nf}{process}\PYG{p}{(}\PYG{n+nb+bp}{self}\PYG{p}{,} \PYG{n}{document}\PYG{p}{,} \PYG{n}{block}\PYG{p}{,} \PYG{n}{index}\PYG{p}{)}\PYG{p}{:}
        \PYG{n}{l} \PYG{o}{=} \PYG{n}{block}\PYG{p}{[}\PYG{n}{index}\PYG{p}{]}
        \PYG{n+nb}{super}\PYG{p}{(}\PYG{n}{RInclude}\PYG{p}{,} \PYG{n+nb+bp}{self}\PYG{p}{)}\PYG{o}{.}\PYG{n}{process}\PYG{p}{(}\PYG{n}{document}\PYG{p}{,} \PYG{n}{block}\PYG{p}{,} \PYG{n}{index}\PYG{p}{)}

        \PYG{n}{block}\PYG{p}{[}\PYG{n}{index}\PYG{p}{:}\PYG{n}{index}\PYG{p}{]} \PYG{o}{=} \PYG{p}{[} \PYG{n}{l}\PYG{o}{.}\PYG{n}{like}\PYG{p}{(}\PYG{l+s+s2}{\PYGZdq{}}\PYG{l+s+s2}{\PYGZdq{}}\PYG{p}{)}\PYG{p}{,}
                               \PYG{n}{l}\PYG{o}{.}\PYG{n}{like}\PYG{p}{(}\PYG{l+s+s2}{\PYGZdq{}}\PYG{l+s+s2}{.. \PYGZus{}}\PYG{l+s+si}{\PYGZpc{}s}\PYG{l+s+s2}{:}\PYG{l+s+s2}{\PYGZdq{}} \PYG{o}{\PYGZpc{}} \PYG{n+nb+bp}{self}\PYG{o}{.}\PYG{n}{name}\PYG{p}{)}\PYG{p}{,}
                               \PYG{n}{l}\PYG{o}{.}\PYG{n}{like}\PYG{p}{(}\PYG{l+s+s2}{\PYGZdq{}}\PYG{l+s+s2}{\PYGZdq{}}\PYG{p}{)}\PYG{p}{,}
                               \PYG{n}{l}\PYG{o}{.}\PYG{n}{like}\PYG{p}{(}\PYG{l+s+s2}{\PYGZdq{}}\PYG{l+s+s2}{**\PYGZlt{}\PYGZlt{}}\PYG{l+s+si}{\PYGZpc{}s}\PYG{l+s+s2}{\PYGZgt{}\PYGZgt{}**}\PYG{l+s+s2}{\PYGZdq{}} \PYG{o}{\PYGZpc{}} \PYG{n+nb+bp}{self}\PYG{o}{.}\PYG{n}{name}\PYG{p}{)}\PYG{p}{,}
                               \PYG{n}{l}\PYG{o}{.}\PYG{n}{like}\PYG{p}{(}\PYG{l+s+s2}{\PYGZdq{}}\PYG{l+s+s2}{\PYGZdq{}}\PYG{p}{)} \PYG{p}{]}
\end{Verbatim}

\end{fulllineitems}



\subsection{Code}
\label{antiweb:code}\index{Code (built-in class)}

\begin{fulllineitems}
\phantomsection\label{antiweb:Code}\pysigline{\sphinxstrong{class }\sphinxbfcode{Code}}
This class represents an \sphinxcode{@code} directive. It inherits
{\hyperref[antiweb:Directive]{\sphinxcrossref{\sphinxcode{Directive}}}}.

The \sphinxcode{@code} directive starts a code block. All
lines following \sphinxcode{@code} will be displayed as source code.
\begin{description}
\item[{A \sphinxcode{@code} directive ends,}] \leavevmode\begin{itemize}
\item {} 
if the text block ends

\item {} 
if an \sphinxcode{@edoc} occurs.

\end{itemize}

\end{description}

The content of the special macro \sphinxcode{\_\_codeprefix\_\_} is inserted
before each code block. \sphinxcode{\_\_codeprefix\_\_} is empty by default
and can be defined by a \sphinxcode{@define} directive.

\begin{Verbatim}[commandchars=\\\{\},numbers=left,firstnumber=1,stepnumber=1]
\PYG{k}{class} \PYG{n+nc}{Code}\PYG{p}{(}\PYG{n}{Directive}\PYG{p}{)}\PYG{p}{:}
    \PYG{n}{expression} \PYG{o}{=} \PYG{n}{re}\PYG{o}{.}\PYG{n}{compile}\PYG{p}{(}\PYG{l+s+s2}{r\PYGZdq{}}\PYG{l+s+s2}{@code}\PYG{l+s+s2}{\PYGZdq{}}\PYG{p}{)}

    \PYG{k}{def} \PYG{n+nf}{process}\PYG{p}{(}\PYG{n+nb+bp}{self}\PYG{p}{,} \PYG{n}{document}\PYG{p}{,} \PYG{n}{block}\PYG{p}{,} \PYG{n}{index}\PYG{p}{)}\PYG{p}{:}
        \PYG{n}{line} \PYG{o}{=} \PYG{n}{block}\PYG{p}{[}\PYG{n}{index}\PYG{p}{]}

        \PYG{c+c1}{\PYGZsh{}change the indentation the code lines}
        \PYG{k}{for} \PYG{n}{j} \PYG{o+ow}{in} \PYG{n+nb}{range}\PYG{p}{(}\PYG{n}{index}\PYG{o}{+}\PYG{l+m+mi}{1}\PYG{p}{,} \PYG{n+nb}{len}\PYG{p}{(}\PYG{n}{block}\PYG{p}{)}\PYG{p}{)}\PYG{p}{:}
            \PYG{n}{l} \PYG{o}{=} \PYG{n}{block}\PYG{p}{[}\PYG{n}{j}\PYG{p}{]}

            \PYG{k}{if} \PYG{n+nb}{isinstance}\PYG{p}{(}\PYG{n}{l}\PYG{o}{.}\PYG{n}{directive}\PYG{p}{,} \PYG{n}{Edoc}\PYG{p}{)}\PYG{p}{:}
                \PYG{k}{break}

            \PYG{n}{block}\PYG{p}{[}\PYG{n}{j}\PYG{p}{]} \PYG{o}{=} \PYG{n}{l}\PYG{o}{.}\PYG{n}{clone}\PYG{p}{(}\PYG{p}{)}\PYG{o}{.}\PYG{n}{change\PYGZus{}indent}\PYG{p}{(}\PYG{l+m+mi}{4}\PYG{p}{)}\PYG{o}{.}\PYG{n}{set}\PYG{p}{(}\PYG{n+nb}{type}\PYG{o}{=}\PYG{l+s+s1}{\PYGZsq{}}\PYG{l+s+s1}{c}\PYG{l+s+s1}{\PYGZsq{}}\PYG{p}{)}

        \PYG{c+c1}{\PYGZsh{}insert the rst prefix}
        \PYG{n}{sd} \PYG{o}{=} \PYG{p}{[}\PYG{n}{Subst}\PYG{p}{(}\PYG{n+nb+bp}{self}\PYG{o}{.}\PYG{n}{line}\PYG{p}{,} \PYG{l+s+s2}{\PYGZdq{}}\PYG{l+s+s2}{\PYGZus{}\PYGZus{}codeprefix\PYGZus{}\PYGZus{}}\PYG{l+s+s2}{\PYGZdq{}}\PYG{p}{)}\PYG{p}{]}
        \PYG{n}{new\PYGZus{}block} \PYG{o}{=} \PYG{p}{[}
            \PYG{n}{line}\PYG{o}{.}\PYG{n}{like}\PYG{p}{(}\PYG{l+s+s2}{\PYGZdq{}}\PYG{l+s+s2}{@subst(\PYGZus{}\PYGZus{}codeprefix\PYGZus{}\PYGZus{})}\PYG{l+s+s2}{\PYGZdq{}}\PYG{p}{)}\PYG{o}{.}\PYG{n}{set}\PYG{p}{(}\PYG{n}{directives}\PYG{o}{=}\PYG{n}{sd}\PYG{p}{)}\PYG{p}{,}
            \PYG{n}{line}\PYG{o}{.}\PYG{n}{like}\PYG{p}{(}\PYG{l+s+s2}{\PYGZdq{}}\PYG{l+s+s2}{::}\PYG{l+s+s2}{\PYGZdq{}}\PYG{p}{)}\PYG{p}{,}
            \PYG{n}{line}\PYG{o}{.}\PYG{n}{like}\PYG{p}{(}\PYG{l+s+s2}{\PYGZdq{}}\PYG{l+s+s2}{\PYGZdq{}}\PYG{p}{)}
            \PYG{p}{]}

        \PYG{n}{block}\PYG{p}{[}\PYG{n}{index}\PYG{p}{:}\PYG{n}{index}\PYG{o}{+}\PYG{l+m+mi}{1}\PYG{p}{]} \PYG{o}{=} \PYG{n}{new\PYGZus{}block}
        \PYG{n}{block}\PYG{o}{.}\PYG{n}{append}\PYG{p}{(}\PYG{n}{line}\PYG{o}{.}\PYG{n}{like}\PYG{p}{(}\PYG{l+s+s2}{\PYGZdq{}}\PYG{l+s+s2}{\PYGZdq{}}\PYG{p}{)}\PYG{p}{)}
\end{Verbatim}

\end{fulllineitems}



\subsection{Edoc}
\label{antiweb:edoc}\index{Edoc (built-in class)}

\begin{fulllineitems}
\phantomsection\label{antiweb:Edoc}\pysigline{\sphinxstrong{class }\sphinxbfcode{Edoc}}
This class represents an \sphinxcode{@edoc} directive. It inherits
{\hyperref[antiweb:Directive]{\sphinxcrossref{\sphinxcode{Directive}}}}.

The \sphinxcode{@edoc} directive ends a previously started \sphinxcode{@code} directive

\begin{Verbatim}[commandchars=\\\{\}]
\PYG{k}{class} \PYG{n+nc}{Edoc}\PYG{p}{(}\PYG{n}{Directive}\PYG{p}{)}\PYG{p}{:}
    \PYG{n}{expression} \PYG{o}{=} \PYG{n}{re}\PYG{o}{.}\PYG{n}{compile}\PYG{p}{(}\PYG{l+s+s2}{r\PYGZdq{}}\PYG{l+s+s2}{@edoc}\PYG{l+s+s2}{\PYGZdq{}}\PYG{p}{)}

    \PYG{k}{def} \PYG{n+nf}{process}\PYG{p}{(}\PYG{n+nb+bp}{self}\PYG{p}{,} \PYG{n}{document}\PYG{p}{,} \PYG{n}{block}\PYG{p}{,} \PYG{n}{index}\PYG{p}{)}\PYG{p}{:}
        \PYG{k}{del} \PYG{n}{block}\PYG{p}{[}\PYG{n}{index}\PYG{p}{]}
\end{Verbatim}

\end{fulllineitems}



\subsection{If}
\label{antiweb:if}\index{If (built-in class)}

\begin{fulllineitems}
\phantomsection\label{antiweb:If}\pysigline{\sphinxstrong{class }\sphinxbfcode{If}}
This class represents an \sphinxcode{@if} directive. It inherits
{\hyperref[antiweb:NameDirective]{\sphinxcrossref{\sphinxcode{NameDirective}}}}.

The \sphinxcode{@if} directive is used for conditional weaving.
The content of an \sphinxcode{@if}, \sphinxcode{@fi} block is waved if the
named token argument of \sphinxcode{@if}, is defined in the command line
by the \sphinxcode{-{-}token} option.

\begin{Verbatim}[commandchars=\\\{\},numbers=left,firstnumber=1,stepnumber=1]
\PYG{k}{class} \PYG{n+nc}{If}\PYG{p}{(}\PYG{n}{NameDirective}\PYG{p}{)}\PYG{p}{:}
    \PYG{n}{expression} \PYG{o}{=} \PYG{n}{re}\PYG{o}{.}\PYG{n}{compile}\PYG{p}{(}\PYG{l+s+s2}{r\PYGZdq{}}\PYG{l+s+s2}{@if}\PYG{l+s+s2}{\PYGZbs{}}\PYG{l+s+s2}{((.+)}\PYG{l+s+s2}{\PYGZbs{}}\PYG{l+s+s2}{)}\PYG{l+s+s2}{\PYGZdq{}}\PYG{p}{)}
    \PYG{n}{priority} \PYG{o}{=} \PYG{l+m+mi}{4}

    \PYG{k}{def} \PYG{n+nf}{process}\PYG{p}{(}\PYG{n+nb+bp}{self}\PYG{p}{,} \PYG{n}{document}\PYG{p}{,} \PYG{n}{block}\PYG{p}{,} \PYG{n}{index}\PYG{p}{)}\PYG{p}{:}
        \PYG{n}{line} \PYG{o}{=} \PYG{n}{block}\PYG{p}{[}\PYG{n}{index}\PYG{p}{]}

        \PYG{k}{for} \PYG{n}{j} \PYG{o+ow}{in} \PYG{n+nb}{range}\PYG{p}{(}\PYG{n}{index}\PYG{o}{+}\PYG{l+m+mi}{1}\PYG{p}{,} \PYG{n+nb}{len}\PYG{p}{(}\PYG{n}{block}\PYG{p}{)}\PYG{p}{)}\PYG{p}{:}
            \PYG{n}{d} \PYG{o}{=} \PYG{n}{block}\PYG{p}{[}\PYG{n}{j}\PYG{p}{]}\PYG{o}{.}\PYG{n}{directive}
            \PYG{k}{if} \PYG{n+nb}{isinstance}\PYG{p}{(}\PYG{n}{d}\PYG{p}{,} \PYG{n}{Fi}\PYG{p}{)} \PYG{o+ow}{and} \PYG{n}{d}\PYG{o}{.}\PYG{n}{name} \PYG{o}{==} \PYG{n+nb+bp}{self}\PYG{o}{.}\PYG{n}{name}\PYG{p}{:}
                \PYG{k}{break}

        \PYG{k}{else}\PYG{p}{:}
            \PYG{n}{document}\PYG{o}{.}\PYG{n}{add\PYGZus{}error}\PYG{p}{(}\PYG{n+nb+bp}{self}\PYG{o}{.}\PYG{n}{line}\PYG{p}{,} \PYG{l+s+s2}{\PYGZdq{}}\PYG{l+s+s2}{No fi for if }\PYG{l+s+si}{\PYGZpc{}s}\PYG{l+s+s2}{\PYGZdq{}} \PYG{o}{\PYGZpc{}} \PYG{n+nb+bp}{self}\PYG{o}{.}\PYG{n}{name}\PYG{p}{)}
            \PYG{k}{return}

        \PYG{k}{if} \PYG{n+nb+bp}{self}\PYG{o}{.}\PYG{n}{name} \PYG{o+ow}{in} \PYG{n}{document}\PYG{o}{.}\PYG{n}{tokens}\PYG{p}{:}
            \PYG{k}{del} \PYG{n}{block}\PYG{p}{[}\PYG{n}{index}\PYG{p}{]}

        \PYG{k}{else}\PYG{p}{:}
            \PYG{k}{del} \PYG{n}{block}\PYG{p}{[}\PYG{n}{index}\PYG{p}{:}\PYG{n}{j}\PYG{p}{]}
\end{Verbatim}

\end{fulllineitems}



\subsection{Fi}
\label{antiweb:fi}\index{Fi (built-in class)}

\begin{fulllineitems}
\phantomsection\label{antiweb:Fi}\pysigline{\sphinxstrong{class }\sphinxbfcode{Fi}}
This class represents a \sphinxtitleref{@fi} directive. It inherits
{\hyperref[antiweb:NameDirective]{\sphinxcrossref{\sphinxcode{NameDirective}}}}.

The \sphinxcode{@fi} ends an \sphinxcode{@if} directive

\begin{Verbatim}[commandchars=\\\{\}]
\PYG{k}{class} \PYG{n+nc}{Fi}\PYG{p}{(}\PYG{n}{NameDirective}\PYG{p}{)}\PYG{p}{:}
    \PYG{n}{expression} \PYG{o}{=} \PYG{n}{re}\PYG{o}{.}\PYG{n}{compile}\PYG{p}{(}\PYG{l+s+s2}{r\PYGZdq{}}\PYG{l+s+s2}{@fi}\PYG{l+s+s2}{\PYGZbs{}}\PYG{l+s+s2}{((.+)}\PYG{l+s+s2}{\PYGZbs{}}\PYG{l+s+s2}{)}\PYG{l+s+s2}{\PYGZdq{}}\PYG{p}{)}

    \PYG{k}{def} \PYG{n+nf}{process}\PYG{p}{(}\PYG{n+nb+bp}{self}\PYG{p}{,} \PYG{n}{document}\PYG{p}{,} \PYG{n}{block}\PYG{p}{,} \PYG{n}{index}\PYG{p}{)}\PYG{p}{:}
        \PYG{k}{del} \PYG{n}{block}\PYG{p}{[}\PYG{n}{index}\PYG{p}{]}
\end{Verbatim}

\end{fulllineitems}



\subsection{Ignore}
\label{antiweb:ignore}\index{Ignore (built-in class)}

\begin{fulllineitems}
\phantomsection\label{antiweb:Ignore}\pysigline{\sphinxstrong{class }\sphinxbfcode{Ignore}}
This class represents an \sphinxcode{@ignore} directive. It inherits
{\hyperref[antiweb:Directive]{\sphinxcrossref{\sphinxcode{Directive}}}}.

The \sphinxcode{@ignore} directive ignores the line in the
documentation output. It can be used for commentaries.

\begin{Verbatim}[commandchars=\\\{\}]
\PYG{k}{class} \PYG{n+nc}{Ignore}\PYG{p}{(}\PYG{n}{Directive}\PYG{p}{)}\PYG{p}{:}
    \PYG{n}{expression} \PYG{o}{=} \PYG{n}{re}\PYG{o}{.}\PYG{n}{compile}\PYG{p}{(}\PYG{l+s+s2}{\PYGZdq{}}\PYG{l+s+s2}{@ignore}\PYG{l+s+s2}{\PYGZdq{}}\PYG{p}{)}

    \PYG{k}{def} \PYG{n+nf}{process}\PYG{p}{(}\PYG{n+nb+bp}{self}\PYG{p}{,} \PYG{n}{document}\PYG{p}{,} \PYG{n}{block}\PYG{p}{,} \PYG{n}{index}\PYG{p}{)}\PYG{p}{:}
        \PYG{k}{del} \PYG{n}{block}\PYG{p}{[}\PYG{n}{index}\PYG{p}{]}
\end{Verbatim}

\end{fulllineitems}



\subsection{Define}
\label{antiweb:define}\index{Define (built-in class)}

\begin{fulllineitems}
\phantomsection\label{antiweb:Define}\pysigline{\sphinxstrong{class }\sphinxbfcode{Define}}
This class represents an \sphinxcode{@define} directive. It inherits
{\hyperref[antiweb:NameDirective]{\sphinxcrossref{\sphinxcode{NameDirective}}}}.

The \sphinxcode{@define} directive defines a macro, that can be used
with a \sphinxcode{@subst} directive. If a \sphinxcode{substitution}
argument is given, the macro defines an inline substitution.
Otherwise the \sphinxcode{@define} has to be ended by an \sphinxcode{@enifed}
directive.

\begin{Verbatim}[commandchars=\\\{\},numbers=left,firstnumber=1,stepnumber=1]
\PYG{k}{class} \PYG{n+nc}{Define}\PYG{p}{(}\PYG{n}{NameDirective}\PYG{p}{)}\PYG{p}{:}

    \PYG{n}{expression} \PYG{o}{=} \PYG{n}{re}\PYG{o}{.}\PYG{n}{compile}\PYG{p}{(}\PYG{l+s+s2}{r\PYGZdq{}}\PYG{l+s+s2}{@define}\PYG{l+s+s2}{\PYGZbs{}}\PYG{l+s+s2}{((.+)}\PYG{l+s+s2}{\PYGZbs{}}\PYG{l+s+s2}{)}\PYG{l+s+s2}{\PYGZdq{}}\PYG{p}{)}
    \PYG{n}{priority} \PYG{o}{=} \PYG{l+m+mi}{1}

    \PYG{k}{def} \PYG{n+nf}{process}\PYG{p}{(}\PYG{n+nb+bp}{self}\PYG{p}{,} \PYG{n}{document}\PYG{p}{,} \PYG{n}{block}\PYG{p}{,} \PYG{n}{index}\PYG{p}{)}\PYG{p}{:}
        \PYG{n}{args} \PYG{o}{=} \PYG{n+nb+bp}{self}\PYG{o}{.}\PYG{n}{name}\PYG{o}{.}\PYG{n}{split}\PYG{p}{(}\PYG{l+s+s2}{\PYGZdq{}}\PYG{l+s+s2}{,}\PYG{l+s+s2}{\PYGZdq{}}\PYG{p}{)}
        \PYG{n}{name} \PYG{o}{=} \PYG{n}{args}\PYG{o}{.}\PYG{n}{pop}\PYG{p}{(}\PYG{l+m+mi}{0}\PYG{p}{)}\PYG{o}{.}\PYG{n}{strip}\PYG{p}{(}\PYG{p}{)}

        \PYG{k}{if} \PYG{n}{args}\PYG{p}{:}
            \PYG{c+c1}{\PYGZsh{}more than one argument ==\PYGZgt{} an inline substitution}
            \PYG{n}{document}\PYG{o}{.}\PYG{n}{macros}\PYG{p}{[}\PYG{n}{name}\PYG{p}{]} \PYG{o}{=} \PYG{n}{args}\PYG{p}{[}\PYG{l+m+mi}{0}\PYG{p}{]}\PYG{o}{.}\PYG{n}{strip}\PYG{p}{(}\PYG{p}{)}
            \PYG{k}{return}

        \PYG{c+c1}{\PYGZsh{}search for the matching @enifed}
        \PYG{k}{for} \PYG{n}{j} \PYG{o+ow}{in} \PYG{n+nb}{range}\PYG{p}{(}\PYG{n}{index}\PYG{o}{+}\PYG{l+m+mi}{1}\PYG{p}{,} \PYG{n+nb}{len}\PYG{p}{(}\PYG{n}{block}\PYG{p}{)}\PYG{p}{)}\PYG{p}{:}
            \PYG{n}{d} \PYG{o}{=} \PYG{n}{block}\PYG{p}{[}\PYG{n}{j}\PYG{p}{]}\PYG{o}{.}\PYG{n}{directive}
            \PYG{k}{if} \PYG{n+nb}{isinstance}\PYG{p}{(}\PYG{n}{d}\PYG{p}{,} \PYG{n}{Enifed}\PYG{p}{)} \PYG{o+ow}{and} \PYG{n}{d}\PYG{o}{.}\PYG{n}{name} \PYG{o}{==} \PYG{n}{name}\PYG{p}{:}
                \PYG{k}{break}

        \PYG{k}{else}\PYG{p}{:}
            \PYG{n}{document}\PYG{o}{.}\PYG{n}{add\PYGZus{}error}\PYG{p}{(}\PYG{n+nb+bp}{self}\PYG{o}{.}\PYG{n}{line}\PYG{p}{,} \PYG{l+s+s2}{\PYGZdq{}}\PYG{l+s+s2}{No enifed for define }\PYG{l+s+si}{\PYGZpc{}s}\PYG{l+s+s2}{\PYGZdq{}} \PYG{o}{\PYGZpc{}} \PYG{n}{name}\PYG{p}{)}
            \PYG{k}{return}

        \PYG{n}{document}\PYG{o}{.}\PYG{n}{macros}\PYG{p}{[}\PYG{n}{name}\PYG{p}{]} \PYG{o}{=} \PYG{p}{[} \PYG{n}{l}\PYG{o}{.}\PYG{n}{clone}\PYG{p}{(}\PYG{p}{)} \PYG{k}{for} \PYG{n}{l} \PYG{o+ow}{in} \PYG{n}{block}\PYG{p}{[}\PYG{n}{index}\PYG{o}{+}\PYG{l+m+mi}{1}\PYG{p}{:}\PYG{n}{j}\PYG{p}{]} \PYG{p}{]}
\end{Verbatim}

\end{fulllineitems}



\subsection{Enifed}
\label{antiweb:enifed}\index{Enifed (built-in class)}

\begin{fulllineitems}
\phantomsection\label{antiweb:Enifed}\pysigline{\sphinxstrong{class }\sphinxbfcode{Enifed}}
This class represents an \sphinxcode{@enifed} directive. It inherits
{\hyperref[antiweb:NameDirective]{\sphinxcrossref{\sphinxcode{NameDirective}}}}.

The \sphinxcode{@enifed} directive ends a macro defined by the
\sphinxcode{@define} directive.

\begin{Verbatim}[commandchars=\\\{\},numbers=left,firstnumber=1,stepnumber=1]
\PYG{k}{class} \PYG{n+nc}{Enifed}\PYG{p}{(}\PYG{n}{NameDirective}\PYG{p}{)}\PYG{p}{:}

    \PYG{n}{expression} \PYG{o}{=} \PYG{n}{re}\PYG{o}{.}\PYG{n}{compile}\PYG{p}{(}\PYG{l+s+s2}{r\PYGZdq{}}\PYG{l+s+s2}{@enifed}\PYG{l+s+s2}{\PYGZbs{}}\PYG{l+s+s2}{((.+)}\PYG{l+s+s2}{\PYGZbs{}}\PYG{l+s+s2}{)}\PYG{l+s+s2}{\PYGZdq{}}\PYG{p}{)}

    \PYG{k}{def} \PYG{n+nf}{process}\PYG{p}{(}\PYG{n+nb+bp}{self}\PYG{p}{,} \PYG{n}{document}\PYG{p}{,} \PYG{n}{block}\PYG{p}{,} \PYG{n}{index}\PYG{p}{)}\PYG{p}{:}
        \PYG{k}{del} \PYG{n}{block}\PYG{p}{[}\PYG{n}{index}\PYG{p}{]}
\end{Verbatim}

\end{fulllineitems}



\subsection{Subst}
\label{antiweb:subst}\index{Subst (built-in class)}

\begin{fulllineitems}
\phantomsection\label{antiweb:Subst}\pysigline{\sphinxstrong{class }\sphinxbfcode{Subst}}
This class represents a \sphinxcode{@subst} directive. It inherits
{\hyperref[antiweb:NameDirective]{\sphinxcrossref{\sphinxcode{NameDirective}}}}.

The \sphinxcode{@subst} directive is replaced by the substitution,
defined by a \sphinxcode{@define} directive. There are two predefined
macros:
\begin{quote}
\begin{description}
\item[{\sphinxcode{\_\_line\_\_}}] \leavevmode
Define the current line within the source code. The
\sphinxcode{@subst} can also handle operation with \sphinxcode{\_\_line\_\_}
like \sphinxcode{\_\_line\_\_ + 2}.

\item[{\sphinxcode{\_\_file\_\_}}] \leavevmode
Defines the current source file name.

\end{description}
\end{quote}

\begin{Verbatim}[commandchars=\\\{\},numbers=left,firstnumber=1,stepnumber=1]
\PYG{k}{class} \PYG{n+nc}{Subst}\PYG{p}{(}\PYG{n}{NameDirective}\PYG{p}{)}\PYG{p}{:}
    \PYG{n}{expression} \PYG{o}{=} \PYG{n}{re}\PYG{o}{.}\PYG{n}{compile}\PYG{p}{(}\PYG{l+s+s2}{r\PYGZdq{}}\PYG{l+s+s2}{@subst}\PYG{l+s+s2}{\PYGZbs{}}\PYG{l+s+s2}{((.+?)}\PYG{l+s+s2}{\PYGZbs{}}\PYG{l+s+s2}{)}\PYG{l+s+s2}{\PYGZdq{}}\PYG{p}{)}
    \PYG{n}{priority} \PYG{o}{=} \PYG{l+m+mi}{2}

    \PYG{k}{def} \PYG{n+nf}{process}\PYG{p}{(}\PYG{n+nb+bp}{self}\PYG{p}{,} \PYG{n}{document}\PYG{p}{,} \PYG{n}{block}\PYG{p}{,} \PYG{n}{index}\PYG{p}{)}\PYG{p}{:}
        \PYG{n}{line} \PYG{o}{=} \PYG{n}{block}\PYG{p}{[}\PYG{n}{index}\PYG{p}{]}

        \PYG{c+c1}{\PYGZsh{}find the substitution}
        \PYG{k}{if} \PYG{n+nb+bp}{self}\PYG{o}{.}\PYG{n}{name}\PYG{o}{.}\PYG{n}{startswith}\PYG{p}{(}\PYG{l+s+s2}{\PYGZdq{}}\PYG{l+s+s2}{\PYGZus{}\PYGZus{}line\PYGZus{}\PYGZus{}}\PYG{l+s+s2}{\PYGZdq{}}\PYG{p}{)}\PYG{p}{:}
            \PYG{n}{expression} \PYG{o}{=} \PYG{n+nb+bp}{self}\PYG{o}{.}\PYG{n}{name}\PYG{o}{.}\PYG{n}{replace}\PYG{p}{(}\PYG{l+s+s2}{\PYGZdq{}}\PYG{l+s+s2}{\PYGZus{}\PYGZus{}line\PYGZus{}\PYGZus{}}\PYG{l+s+s2}{\PYGZdq{}}\PYG{p}{,} \PYG{n+nb}{str}\PYG{p}{(}\PYG{n+nb+bp}{self}\PYG{o}{.}\PYG{n}{line}\PYG{o}{+}\PYG{l+m+mi}{1}\PYG{p}{)}\PYG{p}{)}
            \PYG{n}{subst} \PYG{o}{=} \PYG{n+nb}{str}\PYG{p}{(}\PYG{n+nb}{eval}\PYG{p}{(}\PYG{n}{expression}\PYG{p}{)}\PYG{p}{)}

        \PYG{k}{elif} \PYG{n+nb+bp}{self}\PYG{o}{.}\PYG{n}{name} \PYG{o+ow}{not} \PYG{o+ow}{in} \PYG{n}{document}\PYG{o}{.}\PYG{n}{macros}\PYG{p}{:}
            \PYG{n}{document}\PYG{o}{.}\PYG{n}{add\PYGZus{}error}\PYG{p}{(}\PYG{n+nb+bp}{self}\PYG{o}{.}\PYG{n}{line}\PYG{p}{,} \PYG{l+s+s2}{\PYGZdq{}}\PYG{l+s+s2}{No macro }\PYG{l+s+si}{\PYGZpc{}s}\PYG{l+s+s2}{ found}\PYG{l+s+s2}{\PYGZdq{}} \PYG{o}{\PYGZpc{}} \PYG{n+nb+bp}{self}\PYG{o}{.}\PYG{n}{name}\PYG{p}{)}
            \PYG{k}{return}

        \PYG{k}{else}\PYG{p}{:}
            \PYG{n}{subst} \PYG{o}{=} \PYG{n}{document}\PYG{o}{.}\PYG{n}{macros}\PYG{p}{[}\PYG{n+nb+bp}{self}\PYG{o}{.}\PYG{n}{name}\PYG{p}{]}

        \PYG{k}{if} \PYG{n+nb}{isinstance}\PYG{p}{(}\PYG{n}{subst}\PYG{p}{,} \PYG{n+nb}{str}\PYG{p}{)}\PYG{p}{:}
            \PYG{c+c1}{\PYGZsh{}inline substitution}
            \PYG{n}{l} \PYG{o}{=} \PYG{n}{line}\PYG{o}{.}\PYG{n}{clone}\PYG{p}{(}\PYG{p}{)}
            \PYG{n}{l}\PYG{o}{.}\PYG{n}{text} \PYG{o}{=} \PYG{n}{line}\PYG{o}{.}\PYG{n}{text}\PYG{o}{.}\PYG{n}{replace}\PYG{p}{(}\PYG{l+s+s2}{\PYGZdq{}}\PYG{l+s+s2}{@subst(}\PYG{l+s+si}{\PYGZpc{}s}\PYG{l+s+s2}{)}\PYG{l+s+s2}{\PYGZdq{}} \PYG{o}{\PYGZpc{}} \PYG{n+nb+bp}{self}\PYG{o}{.}\PYG{n}{name}\PYG{p}{,} \PYG{n}{subst}\PYG{p}{)}
            \PYG{n}{block}\PYG{p}{[}\PYG{n}{index}\PYG{p}{]} \PYG{o}{=} \PYG{n}{l}
        \PYG{k}{else}\PYG{p}{:}
            \PYG{n}{ln} \PYG{o}{=} \PYG{n}{line}\PYG{o}{.}\PYG{n}{index}
            \PYG{n}{block}\PYG{p}{[}\PYG{n}{index}\PYG{p}{:}\PYG{n}{index}\PYG{o}{+}\PYG{l+m+mi}{1}\PYG{p}{]} \PYG{o}{=} \PYG{p}{[} \PYG{n}{l}\PYG{o}{.}\PYG{n}{clone}\PYG{p}{(}\PYG{n+nb+bp}{self}\PYG{o}{.}\PYG{n}{line}\PYG{o}{+}\PYG{n}{j}\PYG{p}{)}\PYGZbs{}
                                         \PYG{o}{.}\PYG{n}{change\PYGZus{}indent}\PYG{p}{(}\PYG{n}{line}\PYG{o}{.}\PYG{n}{indent}\PYG{p}{)}\PYGZbs{}
                                         \PYG{o}{.}\PYG{n}{set}\PYG{p}{(}\PYG{n}{index}\PYG{o}{=}\PYG{n}{ln}\PYG{o}{+}\PYG{n}{j}\PYG{p}{)}
                                     \PYG{k}{for} \PYG{n}{j}\PYG{p}{,} \PYG{n}{l} \PYG{o+ow}{in} \PYG{n+nb}{enumerate}\PYG{p}{(}\PYG{n}{subst}\PYG{p}{)} \PYG{p}{]}
\end{Verbatim}

\end{fulllineitems}



\subsection{Indent}
\label{antiweb:indent}\index{Indent (built-in class)}

\begin{fulllineitems}
\phantomsection\label{antiweb:Indent}\pysigline{\sphinxstrong{class }\sphinxbfcode{Indent}}
This class represents an \sphinxcode{@indent} directive. It inherits
{\hyperref[antiweb:Directive]{\sphinxcrossref{\sphinxcode{Directive}}}}.

The \sphinxcode{@indent} directive changes the indentation of the
following lines. For example a  call \sphinxcode{@indent -4}
dedents the following lines by 4 spaces.

\begin{Verbatim}[commandchars=\\\{\},numbers=left,firstnumber=1,stepnumber=1]
\PYG{k}{class} \PYG{n+nc}{Indent}\PYG{p}{(}\PYG{n}{Directive}\PYG{p}{)}\PYG{p}{:}
    \PYG{n}{expression} \PYG{o}{=} \PYG{n}{re}\PYG{o}{.}\PYG{n}{compile}\PYG{p}{(}\PYG{l+s+s2}{\PYGZdq{}}\PYG{l+s+s2}{@indent}\PYG{l+s+s2}{\PYGZbs{}}\PYG{l+s+s2}{s+([+\PYGZhy{}]?}\PYG{l+s+s2}{\PYGZbs{}}\PYG{l+s+s2}{d+)}\PYG{l+s+s2}{\PYGZdq{}}\PYG{p}{)}

    \PYG{k}{def} \PYG{n+nf}{\PYGZus{}\PYGZus{}init\PYGZus{}\PYGZus{}}\PYG{p}{(}\PYG{n+nb+bp}{self}\PYG{p}{,} \PYG{n}{line}\PYG{p}{,} \PYG{n}{mo}\PYG{p}{)}\PYG{p}{:}
        \PYG{n+nb}{super}\PYG{p}{(}\PYG{n}{Indent}\PYG{p}{,} \PYG{n+nb+bp}{self}\PYG{p}{)}\PYG{o}{.}\PYG{n}{\PYGZus{}\PYGZus{}init\PYGZus{}\PYGZus{}}\PYG{p}{(}\PYG{n}{line}\PYG{p}{,} \PYG{n}{mo}\PYG{p}{)}
        \PYG{n+nb+bp}{self}\PYG{o}{.}\PYG{n}{indent} \PYG{o}{=} \PYG{n+nb}{int}\PYG{p}{(}\PYG{n}{mo}\PYG{o}{.}\PYG{n}{group}\PYG{p}{(}\PYG{l+m+mi}{1}\PYG{p}{)}\PYG{p}{)}


    \PYG{k}{def} \PYG{n+nf}{process}\PYG{p}{(}\PYG{n+nb+bp}{self}\PYG{p}{,} \PYG{n}{document}\PYG{p}{,} \PYG{n}{block}\PYG{p}{,} \PYG{n}{index}\PYG{p}{)}\PYG{p}{:}
        \PYG{n}{lines} \PYG{o}{=} \PYG{p}{[} \PYG{n}{l}\PYG{o}{.}\PYG{n}{clone}\PYG{p}{(}\PYG{p}{)}\PYG{o}{.}\PYG{n}{change\PYGZus{}indent}\PYG{p}{(}\PYG{n+nb+bp}{self}\PYG{o}{.}\PYG{n}{indent}\PYG{p}{)}
                  \PYG{k}{for} \PYG{n}{l} \PYG{o+ow}{in} \PYG{n}{block}\PYG{p}{[}\PYG{n}{index}\PYG{o}{+}\PYG{l+m+mi}{1}\PYG{p}{:}\PYG{p}{]} \PYG{p}{]}
        \PYG{n}{block}\PYG{p}{[}\PYG{n}{index}\PYG{p}{:}\PYG{p}{]} \PYG{o}{=} \PYG{n}{lines}
\end{Verbatim}

\end{fulllineitems}



\section{Readers}
\label{antiweb:readers}
Readers are responsible for the language dependent
source parsing.


\subsection{Reader}
\label{antiweb:reader}\index{Reader (built-in class)}

\begin{fulllineitems}
\phantomsection\label{antiweb:Reader}\pysiglinewithargsret{\sphinxstrong{class }\sphinxbfcode{Reader}}{\emph{lexer}}{}
This is the base class for all readers. The public functions
exposed to {\hyperref[antiweb:Document]{\sphinxcrossref{\sphinxcode{Document}}}} are {\hyperref[antiweb:Reader.process]{\sphinxcrossref{\sphinxcode{process()}}}},
and {\hyperref[antiweb:Reader.filter_output]{\sphinxcrossref{\sphinxcode{filter\_output()}}}}.

The main tasks for a reader is:
\begin{itemize}
\item {} 
Recognize lines that can contain directives. (comment lines or doc strings).

\item {} 
Modify the source for language specific optimizations.

\item {} 
Filter the processed output.

\end{itemize}
\begin{quote}\begin{description}
\item[{Parameters}] \leavevmode
\textbf{\texttt{lexer}} -- A pygments lexer for the specified language

\end{description}\end{quote}

\begin{Verbatim}[commandchars=\\\{\},numbers=left,firstnumber=1,stepnumber=1]
\PYG{n}{re\PYGZus{}line\PYGZus{}start} \PYG{o}{=} \PYG{n}{re}\PYG{o}{.}\PYG{n}{compile}\PYG{p}{(}\PYG{l+s+s2}{\PYGZdq{}}\PYG{l+s+s2}{\PYGZca{}}\PYG{l+s+s2}{\PYGZdq{}}\PYG{p}{,} \PYG{n}{re}\PYG{o}{.}\PYG{n}{M}\PYG{p}{)} \PYG{c+c1}{\PYGZsh{}to find the line start indices}

\PYG{k}{class} \PYG{n+nc}{Reader}\PYG{p}{(}\PYG{n+nb}{object}\PYG{p}{)}\PYG{p}{:}
    \PYG{c+c1}{\PYGZsh{}Public Methods}
    \PYG{o}{\PYGZlt{}\PYGZlt{}}\PYG{n}{Reader}\PYG{o}{.}\PYG{n}{\PYGZus{}\PYGZus{}init\PYGZus{}\PYGZus{}}\PYG{o}{\PYGZgt{}\PYGZgt{}}
    \PYG{o}{\PYGZlt{}\PYGZlt{}}\PYG{n}{Reader}\PYG{o}{.}\PYG{n}{process}\PYG{o}{\PYGZgt{}\PYGZgt{}}
    \PYG{o}{\PYGZlt{}\PYGZlt{}}\PYG{n}{Reader}\PYG{o}{.}\PYG{n}{filter\PYGZus{}output}\PYG{o}{\PYGZgt{}\PYGZgt{}}

    \PYG{c+c1}{\PYGZsh{}Protected Methods}
    \PYG{o}{\PYGZlt{}\PYGZlt{}}\PYG{n}{Reader}\PYG{o}{.}\PYG{n}{\PYGZus{}accept\PYGZus{}token}\PYG{o}{\PYGZgt{}\PYGZgt{}}
    \PYG{o}{\PYGZlt{}\PYGZlt{}}\PYG{n}{Reader}\PYG{o}{.}\PYG{n}{\PYGZus{}post\PYGZus{}process}\PYG{o}{\PYGZgt{}\PYGZgt{}}
    \PYG{o}{\PYGZlt{}\PYGZlt{}}\PYG{n}{Reader}\PYG{o}{.}\PYG{n}{\PYGZus{}handle\PYGZus{}token}\PYG{o}{\PYGZgt{}\PYGZgt{}}
    \PYG{o}{\PYGZlt{}\PYGZlt{}}\PYG{n}{Reader}\PYG{o}{.}\PYG{n}{\PYGZus{}cut\PYGZus{}comment}\PYG{o}{\PYGZgt{}\PYGZgt{}}
\end{Verbatim}
\index{\_\_init\_\_() (Reader method)}

\begin{fulllineitems}
\phantomsection\label{antiweb:Reader.__init__}\pysiglinewithargsret{\sphinxbfcode{\_\_init\_\_}}{\emph{lexer}}{}
The constructor

\begin{Verbatim}[commandchars=\\\{\}]
\PYG{k}{def} \PYG{n+nf}{\PYGZus{}\PYGZus{}init\PYGZus{}\PYGZus{}}\PYG{p}{(}\PYG{n+nb+bp}{self}\PYG{p}{,} \PYG{n}{lexer}\PYG{p}{)}\PYG{p}{:}
    \PYG{n+nb+bp}{self}\PYG{o}{.}\PYG{n}{lexer} \PYG{o}{=} \PYG{n}{lexer}
\end{Verbatim}

\end{fulllineitems}

\index{process() (Reader method)}

\begin{fulllineitems}
\phantomsection\label{antiweb:Reader.process}\pysiglinewithargsret{\sphinxbfcode{process}}{\emph{fname}, \emph{text}}{}
Reads the source code and identifies the directives.
This method is call by {\hyperref[antiweb:Document]{\sphinxcrossref{\sphinxcode{Document}}}}.
\begin{quote}\begin{description}
\item[{Parameters}] \leavevmode\begin{itemize}
\item {} 
\textbf{\texttt{fname}} (\emph{\texttt{string}}) -- The file name of the source code

\item {} 
\textbf{\texttt{text}} (\emph{\texttt{string}}) -- The source code

\end{itemize}

\item[{Returns}] \leavevmode
A list of {\hyperref[antiweb:Line]{\sphinxcrossref{\sphinxcode{Line}}}} objects.

\end{description}\end{quote}

\begin{Verbatim}[commandchars=\\\{\},numbers=left,firstnumber=1,stepnumber=1]
\PYG{k}{def} \PYG{n+nf}{process}\PYG{p}{(}\PYG{n+nb+bp}{self}\PYG{p}{,} \PYG{n}{fname}\PYG{p}{,} \PYG{n}{text}\PYG{p}{)}\PYG{p}{:}
    \PYG{n}{text} \PYG{o}{=} \PYG{n}{text}\PYG{o}{.}\PYG{n}{replace}\PYG{p}{(}\PYG{l+s+s2}{\PYGZdq{}}\PYG{l+s+se}{\PYGZbs{}t}\PYG{l+s+s2}{\PYGZdq{}}\PYG{p}{,} \PYG{l+s+s2}{\PYGZdq{}}\PYG{l+s+s2}{ }\PYG{l+s+s2}{\PYGZdq{}}\PYG{o}{*}\PYG{l+m+mi}{8}\PYG{p}{)}
    \PYG{n}{starts} \PYG{o}{=} \PYG{p}{[} \PYG{n}{mo}\PYG{o}{.}\PYG{n}{start}\PYG{p}{(}\PYG{p}{)} \PYG{k}{for} \PYG{n}{mo} \PYG{o+ow}{in} \PYG{n}{re\PYGZus{}line\PYGZus{}start}\PYG{o}{.}\PYG{n}{finditer}\PYG{p}{(}\PYG{n}{text}\PYG{p}{)} \PYG{p}{]}
    \PYG{n}{lines} \PYG{o}{=} \PYG{p}{[} \PYG{n}{Line}\PYG{p}{(}\PYG{n}{fname}\PYG{p}{,} \PYG{n}{i}\PYG{p}{,} \PYG{n}{l}\PYG{p}{)} \PYG{k}{for} \PYG{n}{i}\PYG{p}{,} \PYG{n}{l} \PYG{o+ow}{in} \PYG{n+nb}{enumerate}\PYG{p}{(}\PYG{n}{text}\PYG{o}{.}\PYG{n}{splitlines}\PYG{p}{(}\PYG{p}{)}\PYG{p}{)} \PYG{p}{]}

    \PYG{n+nb+bp}{self}\PYG{o}{.}\PYG{n}{lines} \PYG{o}{=} \PYG{n}{lines}    \PYG{c+c1}{\PYGZsh{} A list of lines}
    \PYG{n+nb+bp}{self}\PYG{o}{.}\PYG{n}{starts} \PYG{o}{=} \PYG{n}{starts}  \PYG{c+c1}{\PYGZsh{} the start indices of the lines}

    \PYG{n}{tokens} \PYG{o}{=} \PYG{n+nb+bp}{self}\PYG{o}{.}\PYG{n}{lexer}\PYG{o}{.}\PYG{n}{get\PYGZus{}tokens\PYGZus{}unprocessed}\PYG{p}{(}\PYG{n}{text}\PYG{p}{)}
    \PYG{k}{for} \PYG{n}{index}\PYG{p}{,} \PYG{n}{token}\PYG{p}{,} \PYG{n}{value} \PYG{o+ow}{in} \PYG{n}{tokens}\PYG{p}{:}
        \PYG{n+nb+bp}{self}\PYG{o}{.}\PYG{n}{\PYGZus{}handle\PYGZus{}token}\PYG{p}{(}\PYG{n}{index}\PYG{p}{,} \PYG{n}{token}\PYG{p}{,} \PYG{n}{value}\PYG{p}{)}

    \PYG{n+nb+bp}{self}\PYG{o}{.}\PYG{n}{\PYGZus{}post\PYGZus{}process}\PYG{p}{(}\PYG{n}{fname}\PYG{p}{,} \PYG{n}{text}\PYG{p}{)}
    \PYG{k}{return} \PYG{n+nb+bp}{self}\PYG{o}{.}\PYG{n}{lines}
\end{Verbatim}

\end{fulllineitems}

\index{filter\_output() (Reader method)}

\begin{fulllineitems}
\phantomsection\label{antiweb:Reader.filter_output}\pysiglinewithargsret{\sphinxbfcode{filter\_output}}{\emph{lines}}{}
This method is call by {\hyperref[antiweb:Document]{\sphinxcrossref{\sphinxcode{Document}}}} and gives
the reader the chance to influence the final output.

\begin{Verbatim}[commandchars=\\\{\}]
\PYG{k}{def} \PYG{n+nf}{filter\PYGZus{}output}\PYG{p}{(}\PYG{n+nb+bp}{self}\PYG{p}{,} \PYG{n}{lines}\PYG{p}{)}\PYG{p}{:}
    \PYG{k}{return} \PYG{n}{lines}
\end{Verbatim}

\end{fulllineitems}

\index{\_handle\_token() (Reader method)}

\begin{fulllineitems}
\phantomsection\label{antiweb:Reader._handle_token}\pysiglinewithargsret{\sphinxbfcode{\_handle\_token}}{\emph{index}, \emph{token}, \emph{value}}{}
Find antiweb directives in valid pygments tokens.
\begin{quote}\begin{description}
\item[{Parameters}] \leavevmode\begin{itemize}
\item {} 
\textbf{\texttt{index}} (\emph{\texttt{integer}}) -- The index within the source code

\item {} 
\textbf{\texttt{token}} -- A pygments token.

\item {} 
\textbf{\texttt{value}} (\emph{\texttt{string}}) -- The token value.

\end{itemize}

\end{description}\end{quote}

\begin{Verbatim}[commandchars=\\\{\},numbers=left,firstnumber=1,stepnumber=1]
\PYG{k}{def} \PYG{n+nf}{\PYGZus{}handle\PYGZus{}token}\PYG{p}{(}\PYG{n+nb+bp}{self}\PYG{p}{,} \PYG{n}{index}\PYG{p}{,} \PYG{n}{token}\PYG{p}{,} \PYG{n}{value}\PYG{p}{)}\PYG{p}{:}

    \PYG{k}{if} \PYG{o+ow}{not} \PYG{n+nb+bp}{self}\PYG{o}{.}\PYG{n}{\PYGZus{}accept\PYGZus{}token}\PYG{p}{(}\PYG{n}{token}\PYG{p}{)}\PYG{p}{:} \PYG{k}{return}

    \PYG{n}{cvalue} \PYG{o}{=} \PYG{n+nb+bp}{self}\PYG{o}{.}\PYG{n}{\PYGZus{}cut\PYGZus{}comment}\PYG{p}{(}\PYG{n}{index}\PYG{p}{,} \PYG{n}{token}\PYG{p}{,} \PYG{n}{value}\PYG{p}{)}
    \PYG{n}{offset} \PYG{o}{=} \PYG{n}{value}\PYG{o}{.}\PYG{n}{index}\PYG{p}{(}\PYG{n}{cvalue}\PYG{p}{)}
    \PYG{n}{found} \PYG{o}{=} \PYG{n+nb+bp}{False}
    \PYG{k}{for} \PYG{n}{k}\PYG{p}{,} \PYG{n}{v} \PYG{o+ow}{in} \PYG{n+nb}{list}\PYG{p}{(}\PYG{n}{directives}\PYG{o}{.}\PYG{n}{items}\PYG{p}{(}\PYG{p}{)}\PYG{p}{)}\PYG{p}{:}
        \PYG{k}{for} \PYG{n}{mo} \PYG{o+ow}{in} \PYG{n}{v}\PYG{o}{.}\PYG{n}{expression}\PYG{o}{.}\PYG{n}{finditer}\PYG{p}{(}\PYG{n}{cvalue}\PYG{p}{)}\PYG{p}{:}
            \PYG{n}{li} \PYG{o}{=} \PYG{n}{bisect}\PYG{o}{.}\PYG{n}{bisect}\PYG{p}{(}\PYG{n+nb+bp}{self}\PYG{o}{.}\PYG{n}{starts}\PYG{p}{,} \PYG{n}{index}\PYG{o}{+}\PYG{n}{mo}\PYG{o}{.}\PYG{n}{start}\PYG{p}{(}\PYG{p}{)}\PYG{o}{+}\PYG{n}{offset}\PYG{p}{)}\PYG{o}{\PYGZhy{}}\PYG{l+m+mi}{1}
            \PYG{n}{line} \PYG{o}{=} \PYG{n+nb+bp}{self}\PYG{o}{.}\PYG{n}{lines}\PYG{p}{[}\PYG{n}{li}\PYG{p}{]}
            \PYG{n}{line}\PYG{o}{.}\PYG{n}{directives} \PYG{o}{=} \PYG{n+nb}{list}\PYG{p}{(}\PYG{n}{line}\PYG{o}{.}\PYG{n}{directives}\PYG{p}{)} \PYG{o}{+} \PYG{p}{[} \PYG{n}{v}\PYG{p}{(}\PYG{n}{line}\PYG{o}{.}\PYG{n}{index}\PYG{p}{,} \PYG{n}{mo}\PYG{p}{)} \PYG{p}{]}
\end{Verbatim}

\end{fulllineitems}

\index{\_cut\_comment() (Reader method)}

\begin{fulllineitems}
\phantomsection\label{antiweb:Reader._cut_comment}\pysiglinewithargsret{\sphinxbfcode{\_cut\_comment}}{\emph{index}, \emph{token}, \emph{value}}{}
Cuts of the comment identifiers.
\begin{quote}\begin{description}
\item[{Parameters}] \leavevmode\begin{itemize}
\item {} 
\textbf{\texttt{index}} (\emph{\texttt{integer}}) -- The index within the source code

\item {} 
\textbf{\texttt{token}} -- A pygments token.

\item {} 
\textbf{\texttt{value}} (\emph{\texttt{string}}) -- The token value.

\end{itemize}

\item[{Returns}] \leavevmode
value without comment identifiers.

\end{description}\end{quote}

\begin{Verbatim}[commandchars=\\\{\}]
\PYG{k}{def} \PYG{n+nf}{\PYGZus{}cut\PYGZus{}comment}\PYG{p}{(}\PYG{n+nb+bp}{self}\PYG{p}{,} \PYG{n}{index}\PYG{p}{,} \PYG{n}{token}\PYG{p}{,} \PYG{n}{value}\PYG{p}{)}\PYG{p}{:}
    \PYG{k}{return} \PYG{n}{text}
\end{Verbatim}

\end{fulllineitems}

\index{\_post\_process() (Reader method)}

\begin{fulllineitems}
\phantomsection\label{antiweb:Reader._post_process}\pysiglinewithargsret{\sphinxbfcode{\_post\_process}}{\emph{fname}, \emph{text}}{}
Does some post processing after the directives where found.

\begin{Verbatim}[commandchars=\\\{\},numbers=left,firstnumber=1,stepnumber=1]
\PYG{k}{def} \PYG{n+nf}{\PYGZus{}post\PYGZus{}process}\PYG{p}{(}\PYG{n+nb+bp}{self}\PYG{p}{,} \PYG{n}{fname}\PYG{p}{,} \PYG{n}{text}\PYG{p}{)}\PYG{p}{:}

    \PYG{c+c1}{\PYGZsh{}correct the line attribute of directives, in case there have}
    \PYG{c+c1}{\PYGZsh{}been lines inserted or deleted by subclasses of Reader}
    \PYG{k}{for} \PYG{n}{i}\PYG{p}{,} \PYG{n}{l} \PYG{o+ow}{in} \PYG{n+nb}{enumerate}\PYG{p}{(}\PYG{n+nb+bp}{self}\PYG{o}{.}\PYG{n}{lines}\PYG{p}{)}\PYG{p}{:}
        \PYG{k}{for} \PYG{n}{d} \PYG{o+ow}{in} \PYG{n}{l}\PYG{o}{.}\PYG{n}{directives}\PYG{p}{:}
            \PYG{n}{d}\PYG{o}{.}\PYG{n}{line} \PYG{o}{=} \PYG{n}{i}

    \PYG{c+c1}{\PYGZsh{}give the directives the chance to match}
    \PYG{k}{for} \PYG{n}{l} \PYG{o+ow}{in} \PYG{n+nb+bp}{self}\PYG{o}{.}\PYG{n}{lines}\PYG{p}{:}
        \PYG{k}{for} \PYG{n}{d} \PYG{o+ow}{in} \PYG{n}{l}\PYG{o}{.}\PYG{n}{directives}\PYG{p}{:}
            \PYG{n}{d}\PYG{o}{.}\PYG{n}{match}\PYG{p}{(}\PYG{n+nb+bp}{self}\PYG{o}{.}\PYG{n}{lines}\PYG{p}{)}
\end{Verbatim}

\end{fulllineitems}

\index{\_accept\_token() (Reader method)}

\begin{fulllineitems}
\phantomsection\label{antiweb:Reader._accept_token}\pysiglinewithargsret{\sphinxbfcode{\_accept\_token}}{\emph{token}}{}
Checks if the token type may contain a directive.
\begin{quote}\begin{description}
\item[{Parameters}] \leavevmode
\textbf{\texttt{token}} -- A pygments token

\item[{Returns}] \leavevmode
\sphinxcode{True} if the token may contain a directive.
\sphinxcode{False} otherwise.

\end{description}\end{quote}

\begin{Verbatim}[commandchars=\\\{\}]
\PYG{k}{def} \PYG{n+nf}{\PYGZus{}accept\PYGZus{}token}\PYG{p}{(}\PYG{n+nb+bp}{self}\PYG{p}{,} \PYG{n}{token}\PYG{p}{)}\PYG{p}{:}
    \PYG{k}{return} \PYG{n+nb+bp}{True}
\end{Verbatim}

\end{fulllineitems}


\end{fulllineitems}



\subsection{CReader}
\label{antiweb:creader}\index{CReader (built-in class)}

\begin{fulllineitems}
\phantomsection\label{antiweb:CReader}\pysigline{\sphinxstrong{class }\sphinxbfcode{CReader}}
A reader for C/C++ code. This class inherits {\hyperref[antiweb:Reader]{\sphinxcrossref{\sphinxcode{Reader}}}}.

\begin{Verbatim}[commandchars=\\\{\},numbers=left,firstnumber=1,stepnumber=1]
\PYG{k}{class} \PYG{n+nc}{CReader}\PYG{p}{(}\PYG{n}{Reader}\PYG{p}{)}\PYG{p}{:}
    \PYG{k}{def} \PYG{n+nf}{\PYGZus{}accept\PYGZus{}token}\PYG{p}{(}\PYG{n+nb+bp}{self}\PYG{p}{,} \PYG{n}{token}\PYG{p}{)}\PYG{p}{:}
        \PYG{k}{return} \PYG{n}{token} \PYG{o+ow}{in} \PYG{n}{Token}\PYG{o}{.}\PYG{n}{Comment}

    \PYG{k}{def} \PYG{n+nf}{\PYGZus{}cut\PYGZus{}comment}\PYG{p}{(}\PYG{n+nb+bp}{self}\PYG{p}{,} \PYG{n}{index}\PYG{p}{,} \PYG{n}{token}\PYG{p}{,} \PYG{n}{text}\PYG{p}{)}\PYG{p}{:}
        \PYG{k}{if} \PYG{n}{text}\PYG{o}{.}\PYG{n}{startswith}\PYG{p}{(}\PYG{l+s+s2}{\PYGZdq{}}\PYG{l+s+s2}{/*}\PYG{l+s+s2}{\PYGZdq{}}\PYG{p}{)}\PYG{p}{:}
            \PYG{n}{text} \PYG{o}{=} \PYG{n}{text}\PYG{p}{[}\PYG{l+m+mi}{2}\PYG{p}{:}\PYG{o}{\PYGZhy{}}\PYG{l+m+mi}{2}\PYG{p}{]}

        \PYG{k}{elif} \PYG{n}{text}\PYG{o}{.}\PYG{n}{startswith}\PYG{p}{(}\PYG{l+s+s2}{\PYGZdq{}}\PYG{l+s+s2}{//}\PYG{l+s+s2}{\PYGZdq{}}\PYG{p}{)}\PYG{p}{:}
            \PYG{n}{text} \PYG{o}{=} \PYG{n}{text}\PYG{p}{[}\PYG{l+m+mi}{2}\PYG{p}{:}\PYG{p}{]}

        \PYG{k}{return} \PYG{n}{text}

    \PYG{k}{def} \PYG{n+nf}{filter\PYGZus{}output}\PYG{p}{(}\PYG{n+nb+bp}{self}\PYG{p}{,} \PYG{n}{lines}\PYG{p}{)}\PYG{p}{:}
        \PYG{l+s+sd}{\PYGZdq{}\PYGZdq{}\PYGZdq{}}
\PYG{l+s+sd}{        .. py:method:: filter\PYGZus{}output(lines)}

\PYG{l+s+sd}{           See :py:meth:{}`Reader.filter\PYGZus{}output{}`.}
\PYG{l+s+sd}{        \PYGZdq{}\PYGZdq{}\PYGZdq{}}
        \PYG{k}{for} \PYG{n}{l} \PYG{o+ow}{in} \PYG{n}{lines}\PYG{p}{:}
            \PYG{k}{if} \PYG{n}{l}\PYG{o}{.}\PYG{n}{type} \PYG{o}{==} \PYG{l+s+s2}{\PYGZdq{}}\PYG{l+s+s2}{d}\PYG{l+s+s2}{\PYGZdq{}}\PYG{p}{:}
                \PYG{c+c1}{\PYGZsh{}remove comment chars in document lines}
                \PYG{n}{stext} \PYG{o}{=} \PYG{n}{l}\PYG{o}{.}\PYG{n}{text}\PYG{o}{.}\PYG{n}{lstrip}\PYG{p}{(}\PYG{p}{)}

                \PYG{k}{if} \PYG{n}{stext} \PYG{o}{==} \PYG{l+s+s1}{\PYGZsq{}}\PYG{l+s+s1}{/*}\PYG{l+s+s1}{\PYGZsq{}} \PYG{o+ow}{or} \PYG{n}{stext} \PYG{o}{==} \PYG{l+s+s2}{\PYGZdq{}}\PYG{l+s+s2}{*/}\PYG{l+s+s2}{\PYGZdq{}}\PYG{p}{:}
                    \PYG{c+c1}{\PYGZsh{}remove \PYGZdq{}\PYGZdq{}\PYGZdq{} and \PYGZsq{}\PYGZsq{}\PYGZsq{} from documentation lines}
                    \PYG{c+c1}{\PYGZsh{}see the l.text.lstrip()! if the lines ends with a white space}
                    \PYG{c+c1}{\PYGZsh{}the quotes will be kept! This is feature, to force the quotes}
                    \PYG{c+c1}{\PYGZsh{}in the output}
                    \PYG{k}{continue}

                \PYG{k}{if} \PYG{n}{stext}\PYG{o}{.}\PYG{n}{startswith}\PYG{p}{(}\PYG{l+s+s2}{\PYGZdq{}}\PYG{l+s+s2}{//}\PYG{l+s+s2}{\PYGZdq{}}\PYG{p}{)} \PYG{o+ow}{and} \PYG{o+ow}{not} \PYG{n}{stext}\PYG{o}{.}\PYG{n}{startswith}\PYG{p}{(}\PYG{l+s+s2}{\PYGZdq{}}\PYG{l+s+s2}{\PYGZsh{}\PYGZsh{}\PYGZsh{}\PYGZsh{}\PYGZsh{}}\PYG{l+s+s2}{\PYGZdq{}}\PYG{p}{)}\PYG{p}{:}
                    \PYG{c+c1}{\PYGZsh{}remove comments but not chapters}
                    \PYG{n}{l}\PYG{o}{.}\PYG{n}{text} \PYG{o}{=} \PYG{n}{l}\PYG{o}{.}\PYG{n}{indented}\PYG{p}{(}\PYG{n}{stext}\PYG{p}{[}\PYG{l+m+mi}{2}\PYG{p}{:}\PYG{p}{]}\PYG{p}{)}

            \PYG{k}{yield} \PYG{n}{l}
\end{Verbatim}

\end{fulllineitems}



\subsection{PythonReader}
\label{antiweb:pythonreader}\index{PythonReader (built-in class)}

\begin{fulllineitems}
\phantomsection\label{antiweb:PythonReader}\pysigline{\sphinxstrong{class }\sphinxbfcode{PythonReader}}
A reader for python code. This class inherits {\hyperref[antiweb:Reader]{\sphinxcrossref{\sphinxcode{Reader}}}}.
To reduce the number of sentinels, the python reader does some more
sophisticated source parsing:

A construction like:

\begin{Verbatim}[commandchars=\\\{\},numbers=left,firstnumber=1,stepnumber=1]
\PYG{n+nd}{@cstart}\PYG{p}{(}\PYG{n}{foo}\PYG{p}{)}
\PYG{k}{def} \PYG{n+nf}{foo}\PYG{p}{(}\PYG{n}{arg1}\PYG{p}{,} \PYG{n}{arg2}\PYG{p}{)}\PYG{p}{:}
   \PYG{l+s+sd}{\PYGZdq{}\PYGZdq{}\PYGZdq{}}
\PYG{l+s+sd}{   Foo\PYGZsq{}s documentation}
\PYG{l+s+sd}{   \PYGZdq{}\PYGZdq{}\PYGZdq{}}
   \PYG{n}{code}
\end{Verbatim}

is replaced by:

\begin{Verbatim}[commandchars=\\\{\},numbers=left,firstnumber=1,stepnumber=1]
@cstart(foo)
def foo(arg1, arg2):
   @start(foo doc)
   \PYGZdq{}\PYGZdq{}\PYGZdq{}
   Foo\PYGZsq{}s documentation
   \PYGZdq{}\PYGZdq{}\PYGZdq{}
   @include(foo)
   @(foo doc)
   code
\end{Verbatim}

The replacement will be done only:
\begin{itemize}
\item {} 
If the doc string begins with ``''``

\item {} 
If the block was started by a \sphinxcode{@rstart} or \sphinxcode{@cstart} directive

\item {} 
If there is no antiweb directive in the doc string.

\item {} 
Only a \sphinxcode{@cstart} will insert the @include directive.

\end{itemize}

Additionally the python reader removes all single line \sphinxcode{"""} and \sphinxcode{'{'}'}
from documentation lines. In the following lines:

\begin{Verbatim}[commandchars=\\\{\}]
\PYG{n+nd}{@start}\PYG{p}{(}\PYG{n}{foo}\PYG{p}{)}
\PYG{l+s+sd}{\PYGZdq{}\PYGZdq{}\PYGZdq{}}
\PYG{l+s+sd}{Documentation}
\PYG{l+s+sd}{\PYGZdq{}\PYGZdq{}\PYGZdq{}}
\end{Verbatim}

The \sphinxcode{"""} are automatically removed in the rst output. (see {\hyperref[antiweb:PythonReader.filter_output]{\sphinxcrossref{\sphinxcode{filter\_output()}}}}
for details).

\begin{Verbatim}[commandchars=\\\{\},numbers=left,firstnumber=1,stepnumber=1]
\PYG{k}{class} \PYG{n+nc}{PythonReader}\PYG{p}{(}\PYG{n}{Reader}\PYG{p}{)}\PYG{p}{:}
    \PYG{k}{def} \PYG{n+nf}{\PYGZus{}\PYGZus{}init\PYGZus{}\PYGZus{}}\PYG{p}{(}\PYG{n+nb+bp}{self}\PYG{p}{,} \PYG{n}{lexer}\PYG{p}{)}\PYG{p}{:}
        \PYG{n+nb}{super}\PYG{p}{(}\PYG{n}{PythonReader}\PYG{p}{,} \PYG{n+nb+bp}{self}\PYG{p}{)}\PYG{o}{.}\PYG{n}{\PYGZus{}\PYGZus{}init\PYGZus{}\PYGZus{}}\PYG{p}{(}\PYG{n}{lexer}\PYG{p}{)}
        \PYG{n+nb+bp}{self}\PYG{o}{.}\PYG{n}{doc\PYGZus{}lines} \PYG{o}{=} \PYG{p}{[}\PYG{p}{]}

    \PYG{o}{\PYGZlt{}\PYGZlt{}}\PYG{n}{PythonReader}\PYG{o}{.}\PYG{n}{\PYGZus{}post\PYGZus{}process}\PYG{o}{\PYGZgt{}\PYGZgt{}}
    \PYG{o}{\PYGZlt{}\PYGZlt{}}\PYG{n}{PythonReader}\PYG{o}{.}\PYG{n}{\PYGZus{}accept\PYGZus{}token}\PYG{o}{\PYGZgt{}\PYGZgt{}}
    \PYG{o}{\PYGZlt{}\PYGZlt{}}\PYG{n}{PythonReader}\PYG{o}{.}\PYG{n}{\PYGZus{}cut\PYGZus{}comment}\PYG{o}{\PYGZgt{}\PYGZgt{}}
    \PYG{o}{\PYGZlt{}\PYGZlt{}}\PYG{n}{PythonReader}\PYG{o}{.}\PYG{n}{filter\PYGZus{}output}\PYG{o}{\PYGZgt{}\PYGZgt{}}
\PYG{c+c1}{\PYGZsh{} The keys are the lexer names of pygments}
\PYG{n}{readers} \PYG{o}{=} \PYG{p}{\PYGZob{}}
    \PYG{l+s+s2}{\PYGZdq{}}\PYG{l+s+s2}{C}\PYG{l+s+s2}{\PYGZdq{}} \PYG{p}{:} \PYG{n}{CReader}\PYG{p}{,}
    \PYG{l+s+s2}{\PYGZdq{}}\PYG{l+s+s2}{C++}\PYG{l+s+s2}{\PYGZdq{}} \PYG{p}{:} \PYG{n}{CReader}\PYG{p}{,}
    \PYG{l+s+s2}{\PYGZdq{}}\PYG{l+s+s2}{C\PYGZsh{}}\PYG{l+s+s2}{\PYGZdq{}} \PYG{p}{:} \PYG{n}{CReader}\PYG{p}{,}
    \PYG{l+s+s2}{\PYGZdq{}}\PYG{l+s+s2}{Python}\PYG{l+s+s2}{\PYGZdq{}} \PYG{p}{:} \PYG{n}{PythonReader}\PYG{p}{,}
\PYG{p}{\PYGZcb{}}
\end{Verbatim}
\index{\_post\_process() (PythonReader method)}

\begin{fulllineitems}
\phantomsection\label{antiweb:PythonReader._post_process}\pysiglinewithargsret{\sphinxbfcode{\_post\_process}}{\emph{fname}, \emph{text}}{}
See {\hyperref[antiweb:Reader._post_process]{\sphinxcrossref{\sphinxcode{Reader.\_post\_process()}}}}.

This implementation \emph{decorates} doc strings
with antiweb directives.

\begin{Verbatim}[commandchars=\\\{\},numbers=left,firstnumber=1,stepnumber=1]
\PYG{k}{def} \PYG{n+nf}{\PYGZus{}post\PYGZus{}process}\PYG{p}{(}\PYG{n+nb+bp}{self}\PYG{p}{,} \PYG{n}{fname}\PYG{p}{,} \PYG{n}{text}\PYG{p}{)}\PYG{p}{:}
    \PYG{c+c1}{\PYGZsh{}from behind because we will probably insert some lines}
    \PYG{n+nb+bp}{self}\PYG{o}{.}\PYG{n}{doc\PYGZus{}lines}\PYG{o}{.}\PYG{n}{sort}\PYG{p}{(}\PYG{n}{reverse}\PYG{o}{=}\PYG{n+nb+bp}{True}\PYG{p}{)}

    \PYG{c+c1}{\PYGZsh{}handle each found doc string}
    \PYG{k}{for} \PYG{n}{start\PYGZus{}line}\PYG{p}{,} \PYG{n}{end\PYGZus{}line} \PYG{o+ow}{in} \PYG{n+nb+bp}{self}\PYG{o}{.}\PYG{n}{doc\PYGZus{}lines}\PYG{p}{:}
        \PYG{n}{indents} \PYG{o}{=} \PYG{n+nb}{set}\PYG{p}{(}\PYG{p}{)}

        \PYG{o}{\PYGZlt{}\PYGZlt{}}\PYG{n}{no} \PYG{n}{antiweb} \PYG{n}{directives} \PYG{o+ow}{in} \PYG{n}{doc} \PYG{n}{string}\PYG{o}{\PYGZgt{}\PYGZgt{}}
        \PYG{o}{\PYGZlt{}\PYGZlt{}}\PYG{n}{find} \PYG{n}{the} \PYG{n}{last} \PYG{n}{directive} \PYG{n}{before} \PYG{n}{the} \PYG{n}{doc} \PYG{n}{string}\PYG{o}{\PYGZgt{}\PYGZgt{}}

        \PYG{k}{if} \PYG{n+nb}{isinstance}\PYG{p}{(}\PYG{n}{last\PYGZus{}directive}\PYG{p}{,} \PYG{n}{RStart}\PYG{p}{)}\PYG{p}{:}
            \PYG{o}{\PYGZlt{}\PYGZlt{}}\PYG{n}{decorate} \PYG{n}{beginning} \PYG{o+ow}{and} \PYG{n}{end}\PYG{o}{\PYGZgt{}\PYGZgt{}}

            \PYG{k}{if} \PYG{n+nb}{isinstance}\PYG{p}{(}\PYG{n}{last\PYGZus{}directive}\PYG{p}{,} \PYG{n}{CStart}\PYG{p}{)}\PYG{p}{:}
                \PYG{o}{\PYGZlt{}\PYGZlt{}}\PYG{n}{insert} \PYG{n}{additional} \PYG{n}{include}\PYG{o}{\PYGZgt{}\PYGZgt{}}

    \PYG{n+nb}{super}\PYG{p}{(}\PYG{n}{PythonReader}\PYG{p}{,} \PYG{n+nb+bp}{self}\PYG{p}{)}\PYG{o}{.}\PYG{n}{\PYGZus{}post\PYGZus{}process}\PYG{p}{(}\PYG{n}{fname}\PYG{p}{,} \PYG{n}{text}\PYG{p}{)}
\end{Verbatim}
\phantomsection\label{antiweb:no-antiweb-directives-in-doc-string}
\textbf{\textless{}\textless{}no antiweb directives in doc string\textgreater{}\textgreater{}}

\begin{Verbatim}[commandchars=\\\{\},numbers=left,firstnumber=1,stepnumber=1]
\PYG{c+c1}{\PYGZsh{}If antiweb directives are within the doc string,}
\PYG{c+c1}{\PYGZsh{}the doc string will not be decorated!}
\PYG{n}{directives\PYGZus{}between\PYGZus{}start\PYGZus{}and\PYGZus{}end\PYGZus{}line} \PYG{o}{=} \PYG{n+nb+bp}{False}
\PYG{k}{for} \PYG{n}{l} \PYG{o+ow}{in} \PYG{n+nb+bp}{self}\PYG{o}{.}\PYG{n}{lines}\PYG{p}{[}\PYG{n}{start\PYGZus{}line}\PYG{o}{+}\PYG{l+m+mi}{1}\PYG{p}{:}\PYG{n}{end\PYGZus{}line}\PYG{p}{]}\PYG{p}{:}
    \PYG{k}{if} \PYG{n}{l}\PYG{p}{:}
        \PYG{c+c1}{\PYGZsh{}needed for \PYGZlt{}\PYGZlt{}insert additional include\PYGZgt{}\PYGZgt{}}
        \PYG{n}{indents}\PYG{o}{.}\PYG{n}{add}\PYG{p}{(}\PYG{n}{l}\PYG{o}{.}\PYG{n}{indent}\PYG{p}{)}

    \PYG{k}{if} \PYG{n}{l}\PYG{o}{.}\PYG{n}{directives}\PYG{p}{:}
        \PYG{n}{directives\PYGZus{}between\PYGZus{}start\PYGZus{}and\PYGZus{}end\PYGZus{}line} \PYG{o}{=} \PYG{n+nb+bp}{True}
        \PYG{k}{break}

\PYG{k}{if} \PYG{n}{directives\PYGZus{}between\PYGZus{}start\PYGZus{}and\PYGZus{}end\PYGZus{}line}\PYG{p}{:} \PYG{k}{continue}
\end{Verbatim}
\phantomsection\label{antiweb:find-the-last-directive-before-the-doc-string}
\textbf{\textless{}\textless{}find the last directive before the doc string\textgreater{}\textgreater{}}

\begin{Verbatim}[commandchars=\\\{\}]
\PYG{n}{last\PYGZus{}directive} \PYG{o}{=} \PYG{n+nb+bp}{None}
\PYG{k}{for} \PYG{n}{l} \PYG{o+ow}{in} \PYG{n+nb}{reversed}\PYG{p}{(}\PYG{n+nb+bp}{self}\PYG{o}{.}\PYG{n}{lines}\PYG{p}{[}\PYG{p}{:}\PYG{n}{start\PYGZus{}line}\PYG{p}{]}\PYG{p}{)}\PYG{p}{:}
    \PYG{k}{if} \PYG{n}{l}\PYG{o}{.}\PYG{n}{directives}\PYG{p}{:}
        \PYG{n}{last\PYGZus{}directive} \PYG{o}{=} \PYG{n}{l}\PYG{o}{.}\PYG{n}{directives}\PYG{p}{[}\PYG{l+m+mi}{0}\PYG{p}{]}
        \PYG{k}{break}
\end{Verbatim}
\phantomsection\label{antiweb:decorate-beginning-and-end}
\textbf{\textless{}\textless{}decorate beginning and end\textgreater{}\textgreater{}}

\begin{Verbatim}[commandchars=\\\{\},numbers=left,firstnumber=1,stepnumber=1]
\PYG{n}{l} \PYG{o}{=} \PYG{n+nb+bp}{self}\PYG{o}{.}\PYG{n}{lines}\PYG{p}{[}\PYG{n}{start\PYGZus{}line}\PYG{p}{]}
\PYG{n}{start} \PYG{o}{=} \PYG{n}{Start}\PYG{p}{(}\PYG{n}{start\PYGZus{}line}\PYG{p}{,} \PYG{n}{last\PYGZus{}directive}\PYG{o}{.}\PYG{n}{name} \PYG{o}{+} \PYG{l+s+s2}{\PYGZdq{}}\PYG{l+s+s2}{ doc}\PYG{l+s+s2}{\PYGZdq{}}\PYG{p}{)}
\PYG{n}{l}\PYG{o}{.}\PYG{n}{directives} \PYG{o}{=} \PYG{n+nb}{list}\PYG{p}{(}\PYG{n}{l}\PYG{o}{.}\PYG{n}{directives}\PYG{p}{)} \PYG{o}{+} \PYG{p}{[}\PYG{n}{start}\PYG{p}{]}

\PYG{n}{l} \PYG{o}{=} \PYG{n+nb+bp}{self}\PYG{o}{.}\PYG{n}{lines}\PYG{p}{[}\PYG{n}{end\PYGZus{}line}\PYG{p}{]}
\PYG{n}{end} \PYG{o}{=} \PYG{n}{End}\PYG{p}{(}\PYG{n}{end\PYGZus{}line}\PYG{p}{,} \PYG{n}{last\PYGZus{}directive}\PYG{o}{.}\PYG{n}{name} \PYG{o}{+} \PYG{l+s+s2}{\PYGZdq{}}\PYG{l+s+s2}{ doc}\PYG{l+s+s2}{\PYGZdq{}}\PYG{p}{)}
\PYG{n}{l}\PYG{o}{.}\PYG{n}{directives} \PYG{o}{=} \PYG{n+nb}{list}\PYG{p}{(}\PYG{n}{l}\PYG{o}{.}\PYG{n}{directives}\PYG{p}{)} \PYG{o}{+} \PYG{p}{[}\PYG{n}{end}\PYG{p}{]}
\end{Verbatim}
\phantomsection\label{antiweb:insert-additional-include}
\textbf{\textless{}\textless{}insert additional include\textgreater{}\textgreater{}}

\begin{Verbatim}[commandchars=\\\{\},numbers=left,firstnumber=1,stepnumber=1]
\PYG{n}{l} \PYG{o}{=} \PYG{n}{l}\PYG{o}{.}\PYG{n}{like}\PYG{p}{(}\PYG{l+s+s2}{\PYGZdq{}}\PYG{l+s+s2}{\PYGZdq{}}\PYG{p}{)}
\PYG{n}{include} \PYG{o}{=} \PYG{n}{Include}\PYG{p}{(}\PYG{n}{end\PYGZus{}line}\PYG{p}{,} \PYG{n}{last\PYGZus{}directive}\PYG{o}{.}\PYG{n}{name}\PYG{p}{)}
\PYG{n}{l}\PYG{o}{.}\PYG{n}{directives} \PYG{o}{=} \PYG{n+nb}{list}\PYG{p}{(}\PYG{n}{l}\PYG{o}{.}\PYG{n}{directives}\PYG{p}{)} \PYG{o}{+} \PYG{p}{[}\PYG{n}{include}\PYG{p}{]}
\PYG{n+nb+bp}{self}\PYG{o}{.}\PYG{n}{lines}\PYG{o}{.}\PYG{n}{insert}\PYG{p}{(}\PYG{n}{end\PYGZus{}line}\PYG{p}{,} \PYG{n}{l}\PYG{p}{)}

\PYG{c+c1}{\PYGZsh{}the include directive should have the same}
\PYG{c+c1}{\PYGZsh{}indentation as the .. py:function:: directive}
\PYG{c+c1}{\PYGZsh{}inside the doc string. (It should be second}
\PYG{c+c1}{\PYGZsh{}value of sorted indents)}
\PYG{n}{indents} \PYG{o}{=} \PYG{n+nb}{list}\PYG{p}{(}\PYG{n+nb}{sorted}\PYG{p}{(}\PYG{n}{indents}\PYG{p}{)}\PYG{p}{)}
\PYG{k}{if} \PYG{n+nb}{len}\PYG{p}{(}\PYG{n}{indents}\PYG{p}{)} \PYG{o}{\PYGZgt{}} \PYG{l+m+mi}{1}\PYG{p}{:}
    \PYG{n}{l}\PYG{o}{.}\PYG{n}{change\PYGZus{}indent}\PYG{p}{(}\PYG{n}{indents}\PYG{p}{[}\PYG{l+m+mi}{1}\PYG{p}{]}\PYG{o}{\PYGZhy{}}\PYG{n}{l}\PYG{o}{.}\PYG{n}{indent}\PYG{p}{)}
\end{Verbatim}

\end{fulllineitems}

\index{\_accept\_token() (PythonReader method)}

\begin{fulllineitems}
\phantomsection\label{antiweb:PythonReader._accept_token}\pysiglinewithargsret{\sphinxbfcode{\_accept\_token}}{\emph{token}}{}
See {\hyperref[antiweb:Reader._accept_token]{\sphinxcrossref{\sphinxcode{Reader.\_accept\_token()}}}}.

\begin{Verbatim}[commandchars=\\\{\}]
\PYG{k}{def} \PYG{n+nf}{\PYGZus{}accept\PYGZus{}token}\PYG{p}{(}\PYG{n+nb+bp}{self}\PYG{p}{,} \PYG{n}{token}\PYG{p}{)}\PYG{p}{:}
    \PYG{k}{return} \PYG{n}{token} \PYG{o+ow}{in} \PYG{n}{Token}\PYG{o}{.}\PYG{n}{Comment} \PYG{o+ow}{or} \PYG{n}{token} \PYG{o+ow}{in} \PYG{n}{Token}\PYG{o}{.}\PYG{n}{Literal}\PYG{o}{.}\PYG{n}{String}\PYG{o}{.}\PYG{n}{Doc}
\end{Verbatim}

\end{fulllineitems}

\index{filter\_output() (PythonReader method)}

\begin{fulllineitems}
\phantomsection\label{antiweb:PythonReader.filter_output}\pysiglinewithargsret{\sphinxbfcode{filter\_output}}{\emph{lines}}{}
See {\hyperref[antiweb:Reader.filter_output]{\sphinxcrossref{\sphinxcode{Reader.filter\_output()}}}}.

\begin{Verbatim}[commandchars=\\\{\},numbers=left,firstnumber=1,stepnumber=1]
\PYG{k}{def} \PYG{n+nf}{filter\PYGZus{}output}\PYG{p}{(}\PYG{n+nb+bp}{self}\PYG{p}{,} \PYG{n}{lines}\PYG{p}{)}\PYG{p}{:}
    \PYG{k}{for} \PYG{n}{l} \PYG{o+ow}{in} \PYG{n}{lines}\PYG{p}{:}
        \PYG{k}{if} \PYG{n}{l}\PYG{o}{.}\PYG{n}{type} \PYG{o}{==} \PYG{l+s+s2}{\PYGZdq{}}\PYG{l+s+s2}{d}\PYG{l+s+s2}{\PYGZdq{}}\PYG{p}{:}
            \PYG{c+c1}{\PYGZsh{}remove comment chars in document lines}
            \PYG{n}{stext} \PYG{o}{=} \PYG{n}{l}\PYG{o}{.}\PYG{n}{text}\PYG{o}{.}\PYG{n}{lstrip}\PYG{p}{(}\PYG{p}{)}

            \PYG{k}{if} \PYG{n}{stext} \PYG{o}{==} \PYG{l+s+s1}{\PYGZsq{}}\PYG{l+s+s1}{\PYGZdq{}}\PYG{l+s+s1}{\PYGZdq{}}\PYG{l+s+s1}{\PYGZdq{}}\PYG{l+s+s1}{\PYGZsq{}} \PYG{o+ow}{or} \PYG{n}{stext} \PYG{o}{==} \PYG{l+s+s2}{\PYGZdq{}}\PYG{l+s+s2}{\PYGZsq{}}\PYG{l+s+s2}{\PYGZsq{}}\PYG{l+s+s2}{\PYGZsq{}}\PYG{l+s+s2}{\PYGZdq{}}\PYG{p}{:}
                \PYG{c+c1}{\PYGZsh{}remove \PYGZdq{}\PYGZdq{}\PYGZdq{} and \PYGZsq{}\PYGZsq{}\PYGZsq{} from documentation lines}
                \PYG{c+c1}{\PYGZsh{}see the l.text.lstrip()! if the lines ends with a white space}
                \PYG{c+c1}{\PYGZsh{}the quotes will be kept! This is feature, to force the quotes}
                \PYG{c+c1}{\PYGZsh{}in the output}
                \PYG{k}{continue}

            \PYG{k}{if} \PYG{n}{stext}\PYG{o}{.}\PYG{n}{startswith}\PYG{p}{(}\PYG{l+s+s2}{\PYGZdq{}}\PYG{l+s+s2}{\PYGZsh{}}\PYG{l+s+s2}{\PYGZdq{}}\PYG{p}{)} \PYG{o+ow}{and} \PYG{o+ow}{not} \PYG{n}{stext}\PYG{o}{.}\PYG{n}{startswith}\PYG{p}{(}\PYG{l+s+s2}{\PYGZdq{}}\PYG{l+s+s2}{\PYGZsh{}\PYGZsh{}\PYGZsh{}\PYGZsh{}\PYGZsh{}}\PYG{l+s+s2}{\PYGZdq{}}\PYG{p}{)}\PYG{p}{:}
                \PYG{c+c1}{\PYGZsh{}remove comments but not chapters}
                \PYG{n}{l}\PYG{o}{.}\PYG{n}{text} \PYG{o}{=} \PYG{n}{l}\PYG{o}{.}\PYG{n}{indented}\PYG{p}{(}\PYG{n}{stext}\PYG{p}{[}\PYG{l+m+mi}{1}\PYG{p}{:}\PYG{p}{]}\PYG{p}{)}

        \PYG{k}{yield} \PYG{n}{l}
\end{Verbatim}

\end{fulllineitems}

\index{\_cut\_comment() (PythonReader method)}

\begin{fulllineitems}
\phantomsection\label{antiweb:PythonReader._cut_comment}\pysiglinewithargsret{\sphinxbfcode{\_cut\_comment}}{\emph{index}, \emph{token}, \emph{text}}{}
See {\hyperref[antiweb:Reader._cut_comment]{\sphinxcrossref{\sphinxcode{Reader.\_cut\_comment()}}}}.

\begin{Verbatim}[commandchars=\\\{\},numbers=left,firstnumber=1,stepnumber=1]
\PYG{k}{def} \PYG{n+nf}{\PYGZus{}cut\PYGZus{}comment}\PYG{p}{(}\PYG{n+nb+bp}{self}\PYG{p}{,} \PYG{n}{index}\PYG{p}{,} \PYG{n}{token}\PYG{p}{,} \PYG{n}{text}\PYG{p}{)}\PYG{p}{:}
    \PYG{k}{if} \PYG{n}{token} \PYG{o+ow}{in} \PYG{n}{Token}\PYG{o}{.}\PYG{n}{Literal}\PYG{o}{.}\PYG{n}{String}\PYG{o}{.}\PYG{n}{Doc}\PYG{p}{:}
        \PYG{k}{if} \PYG{n}{text}\PYG{o}{.}\PYG{n}{startswith}\PYG{p}{(}\PYG{l+s+s1}{\PYGZsq{}}\PYG{l+s+s1}{\PYGZdq{}}\PYG{l+s+s1}{\PYGZdq{}}\PYG{l+s+s1}{\PYGZdq{}}\PYG{l+s+s1}{\PYGZsq{}}\PYG{p}{)}\PYG{p}{:}
            \PYG{c+c1}{\PYGZsh{}save the start/end line of doc strings beginning with \PYGZdq{}\PYGZdq{}\PYGZdq{}}
            \PYG{c+c1}{\PYGZsh{}for further decoration processing in \PYGZus{}post\PYGZus{}process,}
            \PYG{n}{start\PYGZus{}line} \PYG{o}{=} \PYG{n}{bisect}\PYG{o}{.}\PYG{n}{bisect}\PYG{p}{(}\PYG{n+nb+bp}{self}\PYG{o}{.}\PYG{n}{starts}\PYG{p}{,} \PYG{n}{index}\PYG{p}{)}\PYG{o}{\PYGZhy{}}\PYG{l+m+mi}{1}
            \PYG{n}{end\PYGZus{}line} \PYG{o}{=} \PYG{n}{bisect}\PYG{o}{.}\PYG{n}{bisect}\PYG{p}{(}\PYG{n+nb+bp}{self}\PYG{o}{.}\PYG{n}{starts}\PYG{p}{,} \PYG{n}{index}\PYG{o}{+}\PYG{n+nb}{len}\PYG{p}{(}\PYG{n}{text}\PYG{p}{)}\PYG{o}{\PYGZhy{}}\PYG{l+m+mi}{3}\PYG{p}{)}\PYG{o}{\PYGZhy{}}\PYG{l+m+mi}{1}
            \PYG{n}{lines} \PYG{o}{=} \PYG{n+nb}{list}\PYG{p}{(}\PYG{n+nb}{filter}\PYG{p}{(}\PYG{n+nb}{bool}\PYG{p}{,} \PYG{n}{text}\PYG{p}{[}\PYG{l+m+mi}{3}\PYG{p}{:}\PYG{o}{\PYGZhy{}}\PYG{l+m+mi}{3}\PYG{p}{]}\PYG{o}{.}\PYG{n}{splitlines}\PYG{p}{(}\PYG{p}{)}\PYG{p}{)}\PYG{p}{)} \PYG{c+c1}{\PYGZsh{}filter out empty strings}
            \PYG{k}{if} \PYG{n}{lines}\PYG{p}{:}
                \PYG{n+nb+bp}{self}\PYG{o}{.}\PYG{n}{doc\PYGZus{}lines}\PYG{o}{.}\PYG{n}{append}\PYG{p}{(}\PYG{p}{(}\PYG{n}{start\PYGZus{}line}\PYG{p}{,} \PYG{n}{end\PYGZus{}line}\PYG{p}{)}\PYG{p}{)}

        \PYG{n}{text} \PYG{o}{=} \PYG{n}{text}\PYG{p}{[}\PYG{l+m+mi}{3}\PYG{p}{:}\PYG{o}{\PYGZhy{}}\PYG{l+m+mi}{3}\PYG{p}{]}

    \PYG{k}{return} \PYG{n}{text}
\end{Verbatim}

\end{fulllineitems}


\end{fulllineitems}



\section{Document}
\label{antiweb:document}

\subsection{Document}
\label{antiweb:id2}\index{Document (built-in class)}

\begin{fulllineitems}
\phantomsection\label{antiweb:Document}\pysiglinewithargsret{\sphinxstrong{class }\sphinxbfcode{Document}}{\emph{text}, \emph{reader}, \emph{fname}, \emph{tokens}}{}
This is the mediator communicating with all other classes
to generate rst output.
:param string text: the source code to parse.
:param reader: An instance of {\hyperref[antiweb:Reader]{\sphinxcrossref{\sphinxcode{Reader}}}}.
:param string fname: The file name of the source code.
:param tokens: A sequence of tokens usable for the \sphinxcode{@if} directive.

\begin{Verbatim}[commandchars=\\\{\},numbers=left,firstnumber=1,stepnumber=1]
\PYG{k}{class} \PYG{n+nc}{Document}\PYG{p}{(}\PYG{n+nb}{object}\PYG{p}{)}\PYG{p}{:}
    \PYG{c+c1}{\PYGZsh{}Attributes}
    \PYG{o}{\PYGZlt{}\PYGZlt{}}\PYG{n}{Document}\PYG{o}{.}\PYG{n}{errors}\PYG{o}{\PYGZgt{}\PYGZgt{}}
    \PYG{o}{\PYGZlt{}\PYGZlt{}}\PYG{n}{Document}\PYG{o}{.}\PYG{n}{blocks}\PYG{o}{\PYGZgt{}\PYGZgt{}}
    \PYG{o}{\PYGZlt{}\PYGZlt{}}\PYG{n}{Document}\PYG{o}{.}\PYG{n}{blocks\PYGZus{}included}\PYG{o}{\PYGZgt{}\PYGZgt{}}
    \PYG{o}{\PYGZlt{}\PYGZlt{}}\PYG{n}{Document}\PYG{o}{.}\PYG{n}{compiled\PYGZus{}blocks}\PYG{o}{\PYGZgt{}\PYGZgt{}}
    \PYG{o}{\PYGZlt{}\PYGZlt{}}\PYG{n}{Document}\PYG{o}{.}\PYG{n}{sub\PYGZus{}documents}\PYG{o}{\PYGZgt{}\PYGZgt{}}
    \PYG{o}{\PYGZlt{}\PYGZlt{}}\PYG{n}{Document}\PYG{o}{.}\PYG{n}{tokens}\PYG{o}{\PYGZgt{}\PYGZgt{}}
    \PYG{o}{\PYGZlt{}\PYGZlt{}}\PYG{n}{Document}\PYG{o}{.}\PYG{n}{macros}\PYG{o}{\PYGZgt{}\PYGZgt{}}
    \PYG{o}{\PYGZlt{}\PYGZlt{}}\PYG{n}{Document}\PYG{o}{.}\PYG{n}{fname}\PYG{o}{\PYGZgt{}\PYGZgt{}}
    \PYG{o}{\PYGZlt{}\PYGZlt{}}\PYG{n}{Document}\PYG{o}{.}\PYG{n}{reader}\PYG{o}{\PYGZgt{}\PYGZgt{}}
    \PYG{o}{\PYGZlt{}\PYGZlt{}}\PYG{n}{Document}\PYG{o}{.}\PYG{n}{lines}\PYG{o}{\PYGZgt{}\PYGZgt{}}

    \PYG{c+c1}{\PYGZsh{}Methods}
    \PYG{o}{\PYGZlt{}\PYGZlt{}}\PYG{n}{Document}\PYG{o}{.}\PYG{n}{\PYGZus{}\PYGZus{}init\PYGZus{}\PYGZus{}}\PYG{o}{\PYGZgt{}\PYGZgt{}}
    \PYG{o}{\PYGZlt{}\PYGZlt{}}\PYG{n}{Document}\PYG{o}{.}\PYG{n}{process}\PYG{o}{\PYGZgt{}\PYGZgt{}}
    \PYG{o}{\PYGZlt{}\PYGZlt{}}\PYG{n}{Document}\PYG{o}{.}\PYG{n}{get\PYGZus{}subdoc}\PYG{o}{\PYGZgt{}\PYGZgt{}}
    \PYG{o}{\PYGZlt{}\PYGZlt{}}\PYG{n}{Document}\PYG{o}{.}\PYG{n}{add\PYGZus{}error}\PYG{o}{\PYGZgt{}\PYGZgt{}}
    \PYG{o}{\PYGZlt{}\PYGZlt{}}\PYG{n}{Document}\PYG{o}{.}\PYG{n}{check\PYGZus{}errors}\PYG{o}{\PYGZgt{}\PYGZgt{}}
    \PYG{o}{\PYGZlt{}\PYGZlt{}}\PYG{n}{Document}\PYG{o}{.}\PYG{n}{collect\PYGZus{}blocks}\PYG{o}{\PYGZgt{}\PYGZgt{}}
    \PYG{o}{\PYGZlt{}\PYGZlt{}}\PYG{n}{Document}\PYG{o}{.}\PYG{n}{get\PYGZus{}compiled\PYGZus{}block}\PYG{o}{\PYGZgt{}\PYGZgt{}}
    \PYG{o}{\PYGZlt{}\PYGZlt{}}\PYG{n}{Document}\PYG{o}{.}\PYG{n}{compile\PYGZus{}block}\PYG{o}{\PYGZgt{}\PYGZgt{}}
\end{Verbatim}
\index{errors (Document attribute)}

\begin{fulllineitems}
\phantomsection\label{antiweb:Document.errors}\pysigline{\sphinxbfcode{errors}}
A list of errors found during generation.

\begin{Verbatim}[commandchars=\\\{\}]
\PYG{n}{errors} \PYG{o}{=} \PYG{p}{[}\PYG{p}{]}
\end{Verbatim}

\end{fulllineitems}

\index{blocks (Document attribute)}

\begin{fulllineitems}
\phantomsection\label{antiweb:Document.blocks}\pysigline{\sphinxbfcode{blocks}}
A dictionary of all found blocks: Name -\textgreater{} List of Lines

\begin{Verbatim}[commandchars=\\\{\}]
\PYG{n}{blocks} \PYG{o}{=} \PYG{p}{\PYGZob{}}\PYG{p}{\PYGZcb{}}
\end{Verbatim}

\end{fulllineitems}

\index{blocks\_included (Document attribute)}

\begin{fulllineitems}
\phantomsection\label{antiweb:Document.blocks_included}\pysigline{\sphinxbfcode{blocks\_included}}
A set containing all block names that have been included by
an @include directive.

\begin{Verbatim}[commandchars=\\\{\}]
\PYG{n}{blocks\PYGZus{}included} \PYG{o}{=} \PYG{n+nb}{set}\PYG{p}{(}\PYG{p}{)}
\end{Verbatim}

\end{fulllineitems}

\index{compiled\_blocks (Document attribute)}

\begin{fulllineitems}
\phantomsection\label{antiweb:Document.compiled_blocks}\pysigline{\sphinxbfcode{compiled\_blocks}}
A set containing all block names that have been already
compiled.

\begin{Verbatim}[commandchars=\\\{\}]
\PYG{n}{compiled\PYGZus{}blocks} \PYG{o}{=} \PYG{n+nb}{set}\PYG{p}{(}\PYG{p}{)}
\end{Verbatim}

\end{fulllineitems}

\index{sub\_documents (Document attribute)}

\begin{fulllineitems}
\phantomsection\label{antiweb:Document.sub_documents}\pysigline{\sphinxbfcode{sub\_documents}}
A cache dictionary of sub documents, referenced by
\sphinxcode{@include} directives: Filename -\textgreater{} Document

\begin{Verbatim}[commandchars=\\\{\}]
\PYG{n}{sub\PYGZus{}documents} \PYG{o}{=} \PYG{p}{\PYGZob{}}\PYG{p}{\PYGZcb{}}
\end{Verbatim}

\end{fulllineitems}

\index{tokens (Document attribute)}

\begin{fulllineitems}
\phantomsection\label{antiweb:Document.tokens}\pysigline{\sphinxbfcode{tokens}}
A set of token names that can be used for the \sphinxcode{@if} directive.

\begin{Verbatim}[commandchars=\\\{\}]
\PYG{n}{tokens} \PYG{o}{=} \PYG{n+nb}{set}\PYG{p}{(}\PYG{p}{)}
\end{Verbatim}

\end{fulllineitems}

\index{macros (Document attribute)}

\begin{fulllineitems}
\phantomsection\label{antiweb:Document.macros}\pysigline{\sphinxbfcode{macros}}
A dictionary containing the macros that can be used
by the \sphinxcode{@subst} directive: Macro name -\textgreater{} substitution.

\begin{Verbatim}[commandchars=\\\{\}]
\PYG{n}{macros} \PYG{o}{=} \PYG{p}{\PYGZob{}}\PYG{p}{\PYGZcb{}}
\end{Verbatim}

\end{fulllineitems}

\index{fname (Document attribute)}

\begin{fulllineitems}
\phantomsection\label{antiweb:Document.fname}\pysigline{\sphinxbfcode{fname}}
The file name of the document's source.

\begin{Verbatim}[commandchars=\\\{\}]
\PYG{n}{fname} \PYG{o}{=} \PYG{l+s+s2}{\PYGZdq{}}\PYG{l+s+s2}{\PYGZdq{}}
\end{Verbatim}

\end{fulllineitems}

\index{reader (Document attribute)}

\begin{fulllineitems}
\phantomsection\label{antiweb:Document.reader}\pysigline{\sphinxbfcode{reader}}
The instance of a {\hyperref[antiweb:Reader]{\sphinxcrossref{\sphinxcode{Reader}}}} object.

\begin{Verbatim}[commandchars=\\\{\}]
\PYG{n}{reader} \PYG{o}{=} \PYG{n+nb+bp}{None}
\end{Verbatim}

\end{fulllineitems}

\index{lines (Document attribute)}

\begin{fulllineitems}
\phantomsection\label{antiweb:Document.lines}\pysigline{\sphinxbfcode{lines}}
A list of {\hyperref[antiweb:Line]{\sphinxcrossref{\sphinxcode{Line}}}} objects representing the whole documents
split in lines.

\begin{Verbatim}[commandchars=\\\{\}]
\PYG{n}{lines} \PYG{o}{=} \PYG{p}{[}\PYG{p}{]}
\end{Verbatim}

\end{fulllineitems}

\index{\_\_init\_\_() (Document method)}

\begin{fulllineitems}
\phantomsection\label{antiweb:Document.__init__}\pysiglinewithargsret{\sphinxbfcode{\_\_init\_\_}}{\emph{text}, \emph{reader}, \emph{fname}, \emph{tokens}}{}
The constructor.

\begin{Verbatim}[commandchars=\\\{\},numbers=left,firstnumber=1,stepnumber=1]
\PYG{k}{def} \PYG{n+nf}{\PYGZus{}\PYGZus{}init\PYGZus{}\PYGZus{}}\PYG{p}{(}\PYG{n+nb+bp}{self}\PYG{p}{,} \PYG{n}{text}\PYG{p}{,} \PYG{n}{reader}\PYG{p}{,} \PYG{n}{fname}\PYG{p}{,} \PYG{n}{tokens}\PYG{p}{)}\PYG{p}{:}
    \PYG{n+nb+bp}{self}\PYG{o}{.}\PYG{n}{errors} \PYG{o}{=} \PYG{p}{[}\PYG{p}{]}
    \PYG{n+nb+bp}{self}\PYG{o}{.}\PYG{n}{blocks} \PYG{o}{=} \PYG{p}{\PYGZob{}}\PYG{p}{\PYGZcb{}}
    \PYG{n+nb+bp}{self}\PYG{o}{.}\PYG{n}{blocks\PYGZus{}included} \PYG{o}{=} \PYG{n+nb}{set}\PYG{p}{(}\PYG{p}{)}
    \PYG{n+nb+bp}{self}\PYG{o}{.}\PYG{n}{compiled\PYGZus{}blocks} \PYG{o}{=} \PYG{n+nb}{set}\PYG{p}{(}\PYG{p}{)}
    \PYG{n+nb+bp}{self}\PYG{o}{.}\PYG{n}{sub\PYGZus{}documents} \PYG{o}{=} \PYG{p}{\PYGZob{}}\PYG{p}{\PYGZcb{}}
    \PYG{n+nb+bp}{self}\PYG{o}{.}\PYG{n}{tokens} \PYG{o}{=} \PYG{n+nb}{set}\PYG{p}{(}\PYG{n}{tokens} \PYG{o+ow}{or} \PYG{p}{[}\PYG{p}{]}\PYG{p}{)}
    \PYG{n+nb+bp}{self}\PYG{o}{.}\PYG{n}{macros} \PYG{o}{=} \PYG{p}{\PYGZob{}} \PYG{l+s+s2}{\PYGZdq{}}\PYG{l+s+s2}{\PYGZus{}\PYGZus{}file\PYGZus{}\PYGZus{}}\PYG{l+s+s2}{\PYGZdq{}} \PYG{p}{:} \PYG{n}{os}\PYG{o}{.}\PYG{n}{path}\PYG{o}{.}\PYG{n}{split}\PYG{p}{(}\PYG{n}{fname}\PYG{p}{)}\PYG{p}{[}\PYG{o}{\PYGZhy{}}\PYG{l+m+mi}{1}\PYG{p}{]}\PYG{p}{,}
                    \PYG{l+s+s2}{\PYGZdq{}}\PYG{l+s+s2}{\PYGZus{}\PYGZus{}codeprefix\PYGZus{}\PYGZus{}}\PYG{l+s+s2}{\PYGZdq{}} \PYG{p}{:} \PYG{l+s+s2}{\PYGZdq{}}\PYG{l+s+s2}{\PYGZdq{}} \PYG{p}{\PYGZcb{}}
    \PYG{n+nb+bp}{self}\PYG{o}{.}\PYG{n}{fname} \PYG{o}{=} \PYG{n}{fname}
    \PYG{n+nb+bp}{self}\PYG{o}{.}\PYG{n}{reader} \PYG{o}{=} \PYG{n}{reader}
    \PYG{n+nb+bp}{self}\PYG{o}{.}\PYG{n}{lines} \PYG{o}{=} \PYG{n+nb+bp}{self}\PYG{o}{.}\PYG{n}{reader}\PYG{o}{.}\PYG{n}{process}\PYG{p}{(}\PYG{n}{fname}\PYG{p}{,} \PYG{n}{text}\PYG{p}{)}
\end{Verbatim}

\end{fulllineitems}

\index{process() (Document method)}

\begin{fulllineitems}
\phantomsection\label{antiweb:Document.process}\pysiglinewithargsret{\sphinxbfcode{process}}{\emph{show\_warnings}}{}
Processes the document and generates the output.
:param bool show\_warnings: If \sphinxcode{True} warnings are emitted.
:return: A string representing the rst output.

\begin{Verbatim}[commandchars=\\\{\},numbers=left,firstnumber=1,stepnumber=1]
\PYG{k}{def} \PYG{n+nf}{process}\PYG{p}{(}\PYG{n+nb+bp}{self}\PYG{p}{,} \PYG{n}{show\PYGZus{}warnings}\PYG{p}{)}\PYG{p}{:}
    \PYG{n+nb+bp}{self}\PYG{o}{.}\PYG{n}{collect\PYGZus{}blocks}\PYG{p}{(}\PYG{p}{)}
    \PYG{k}{if} \PYG{l+s+s2}{\PYGZdq{}}\PYG{l+s+s2}{\PYGZdq{}} \PYG{o+ow}{not} \PYG{o+ow}{in} \PYG{n+nb+bp}{self}\PYG{o}{.}\PYG{n}{blocks}\PYG{p}{:}
        \PYG{n+nb+bp}{self}\PYG{o}{.}\PYG{n}{add\PYGZus{}error}\PYG{p}{(}\PYG{l+m+mi}{0}\PYG{p}{,} \PYG{l+s+s2}{\PYGZdq{}}\PYG{l+s+s2}{no @start() directive found (I need one)}\PYG{l+s+s2}{\PYGZdq{}}\PYG{p}{)}
        \PYG{n+nb+bp}{self}\PYG{o}{.}\PYG{n}{check\PYGZus{}errors}\PYG{p}{(}\PYG{p}{)}

    \PYG{k}{try}\PYG{p}{:}
        \PYG{n}{text} \PYG{o}{=} \PYG{n+nb+bp}{self}\PYG{o}{.}\PYG{n}{get\PYGZus{}compiled\PYGZus{}block}\PYG{p}{(}\PYG{l+s+s2}{\PYGZdq{}}\PYG{l+s+s2}{\PYGZdq{}}\PYG{p}{)}
    \PYG{k}{finally}\PYG{p}{:}
        \PYG{n+nb+bp}{self}\PYG{o}{.}\PYG{n}{check\PYGZus{}errors}\PYG{p}{(}\PYG{p}{)}

    \PYG{k}{if} \PYG{n}{show\PYGZus{}warnings}\PYG{p}{:}
        \PYG{o}{\PYGZlt{}\PYGZlt{}}\PYG{n}{show} \PYG{n}{warnings}\PYG{o}{\PYGZgt{}\PYGZgt{}}

    \PYG{n}{text} \PYG{o}{=} \PYG{n+nb+bp}{self}\PYG{o}{.}\PYG{n}{reader}\PYG{o}{.}\PYG{n}{filter\PYGZus{}output}\PYG{p}{(}\PYG{n}{text}\PYG{p}{)}
    \PYG{k}{return} \PYG{l+s+s2}{\PYGZdq{}}\PYG{l+s+se}{\PYGZbs{}n}\PYG{l+s+s2}{\PYGZdq{}}\PYG{o}{.}\PYG{n}{join}\PYG{p}{(}\PYG{n+nb}{map}\PYG{p}{(}\PYG{n}{operator}\PYG{o}{.}\PYG{n}{attrgetter}\PYG{p}{(}\PYG{l+s+s2}{\PYGZdq{}}\PYG{l+s+s2}{text}\PYG{l+s+s2}{\PYGZdq{}}\PYG{p}{)}\PYG{p}{,} \PYG{n}{text}\PYG{p}{)}\PYG{p}{)}
\end{Verbatim}
\phantomsection\label{antiweb:show-warnings}
\textbf{\textless{}\textless{}show warnings\textgreater{}\textgreater{}}

\begin{Verbatim}[commandchars=\\\{\},numbers=left,firstnumber=1,stepnumber=1]
\PYG{n+nb+bp}{self}\PYG{o}{.}\PYG{n}{blocks\PYGZus{}included}\PYG{o}{.}\PYG{n}{add}\PYG{p}{(}\PYG{l+s+s2}{\PYGZdq{}}\PYG{l+s+s2}{\PYGZdq{}}\PYG{p}{)}           \PYG{c+c1}{\PYGZsh{}may not cause a warning}
\PYG{n+nb+bp}{self}\PYG{o}{.}\PYG{n}{blocks\PYGZus{}included}\PYG{o}{.}\PYG{n}{add}\PYG{p}{(}\PYG{l+s+s2}{\PYGZdq{}}\PYG{l+s+s2}{\PYGZus{}\PYGZus{}macros\PYGZus{}\PYGZus{}}\PYG{l+s+s2}{\PYGZdq{}}\PYG{p}{)} \PYG{c+c1}{\PYGZsh{}may not cause a warning}
\PYG{n}{unincluded} \PYG{o}{=} \PYG{n+nb}{set}\PYG{p}{(}\PYG{n+nb+bp}{self}\PYG{o}{.}\PYG{n}{blocks}\PYG{o}{.}\PYG{n}{keys}\PYG{p}{(}\PYG{p}{)}\PYG{p}{)}\PYG{o}{\PYGZhy{}}\PYG{n+nb+bp}{self}\PYG{o}{.}\PYG{n}{blocks\PYGZus{}included}
\PYG{k}{if} \PYG{n}{unincluded}\PYG{p}{:}
    \PYG{n}{logger}\PYG{o}{.}\PYG{n}{warning}\PYG{p}{(}\PYG{l+s+s2}{\PYGZdq{}}\PYG{l+s+s2}{The following block were not included:}\PYG{l+s+s2}{\PYGZdq{}}\PYG{p}{)}
    \PYG{n}{warnings} \PYG{o}{=} \PYG{p}{[} \PYG{p}{(}\PYG{n+nb+bp}{self}\PYG{o}{.}\PYG{n}{blocks}\PYG{p}{[}\PYG{n}{b}\PYG{p}{]}\PYG{p}{[}\PYG{l+m+mi}{0}\PYG{p}{]}\PYG{o}{.}\PYG{n}{index}\PYG{p}{,} \PYG{n}{b}\PYG{p}{)} \PYG{k}{for} \PYG{n}{b} \PYG{o+ow}{in} \PYG{n}{unincluded} \PYG{p}{]}
    \PYG{n}{warnings}\PYG{o}{.}\PYG{n}{sort}\PYG{p}{(}\PYG{n}{key}\PYG{o}{=}\PYG{n}{operator}\PYG{o}{.}\PYG{n}{itemgetter}\PYG{p}{(}\PYG{l+m+mi}{0}\PYG{p}{)}\PYG{p}{)}
    \PYG{k}{for} \PYG{n}{l}\PYG{p}{,} \PYG{n}{w} \PYG{o+ow}{in} \PYG{n}{warnings}\PYG{p}{:}
        \PYG{n}{logger}\PYG{o}{.}\PYG{n}{warning}\PYG{p}{(}\PYG{l+s+s2}{\PYGZdq{}}\PYG{l+s+s2}{  }\PYG{l+s+si}{\PYGZpc{}s}\PYG{l+s+s2}{(line }\PYG{l+s+si}{\PYGZpc{}i}\PYG{l+s+s2}{)}\PYG{l+s+s2}{\PYGZdq{}}\PYG{p}{,} \PYG{n}{w}\PYG{p}{,} \PYG{n}{l}\PYG{p}{)}
\end{Verbatim}

\end{fulllineitems}

\index{get\_subdoc() (Document method)}

\begin{fulllineitems}
\phantomsection\label{antiweb:Document.get_subdoc}\pysiglinewithargsret{\sphinxbfcode{get\_subdoc}}{\emph{rpath}}{}
Tries to compile a document with the relative path rpath.
:param string rpath: The relative path to the root
containing document.
:return: A {\hyperref[antiweb:Document]{\sphinxcrossref{\sphinxcode{Document}}}} reference to the sub document.

\begin{Verbatim}[commandchars=\\\{\},numbers=left,firstnumber=1,stepnumber=1]
\PYG{k}{def} \PYG{n+nf}{get\PYGZus{}subdoc}\PYG{p}{(}\PYG{n+nb+bp}{self}\PYG{p}{,} \PYG{n}{rpath}\PYG{p}{)}\PYG{p}{:}
    \PYG{o}{\PYGZlt{}\PYGZlt{}}\PYG{k}{return} \PYG{k+kn}{from} \PYG{n+nn}{cache} \PYG{n+nn}{if} \PYG{n+nn}{possible}\PYG{o}{\PYGZgt{}\PYGZgt{}}
    \PYG{o}{\PYGZlt{}\PYGZlt{}}\PYG{n}{insert} \PYG{n}{macros} \PYG{n}{function}\PYG{o}{\PYGZgt{}\PYGZgt{}}
    \PYG{o}{\PYGZlt{}\PYGZlt{}}\PYG{n}{read} \PYG{n}{the} \PYG{n}{source} \PYG{n+nb}{file}\PYG{o}{\PYGZgt{}\PYGZgt{}}

    \PYG{n+nb+bp}{self}\PYG{o}{.}\PYG{n}{sub\PYGZus{}documents}\PYG{p}{[}\PYG{n}{rpath}\PYG{p}{]} \PYG{o}{=} \PYG{n}{doc}
    \PYG{k}{return} \PYG{n}{doc}
\end{Verbatim}
\phantomsection\label{antiweb:return-from-cache-if-possible}
\textbf{\textless{}\textless{}return from cache if possible\textgreater{}\textgreater{}}

\begin{Verbatim}[commandchars=\\\{\}]
\PYG{k}{try}\PYG{p}{:}
    \PYG{k}{return} \PYG{n+nb+bp}{self}\PYG{o}{.}\PYG{n}{sub\PYGZus{}documents}\PYG{p}{[}\PYG{n}{rpath}\PYG{p}{]}
\PYG{k}{except} \PYG{n+ne}{KeyError}\PYG{p}{:}
    \PYG{k}{pass}
\end{Verbatim}
\phantomsection\label{antiweb:insert-macros-function}
\textbf{\textless{}\textless{}insert macros function\textgreater{}\textgreater{}}

\begin{Verbatim}[commandchars=\\\{\},numbers=left,firstnumber=1,stepnumber=1]
\PYG{k}{def} \PYG{n+nf}{insert\PYGZus{}macros}\PYG{p}{(}\PYG{n}{subdoc}\PYG{p}{)}\PYG{p}{:}
    \PYG{c+c1}{\PYGZsh{}if sub doc has no macros insert mine}
    \PYG{k}{if} \PYG{p}{(}\PYG{l+s+s2}{\PYGZdq{}}\PYG{l+s+s2}{\PYGZus{}\PYGZus{}macros\PYGZus{}\PYGZus{}}\PYG{l+s+s2}{\PYGZdq{}} \PYG{o+ow}{not} \PYG{o+ow}{in} \PYG{n}{subdoc}\PYG{o}{.}\PYG{n}{blocks}
        \PYG{o+ow}{and} \PYG{l+s+s2}{\PYGZdq{}}\PYG{l+s+s2}{\PYGZus{}\PYGZus{}macros\PYGZus{}\PYGZus{}}\PYG{l+s+s2}{\PYGZdq{}} \PYG{o+ow}{in} \PYG{n+nb+bp}{self}\PYG{o}{.}\PYG{n}{blocks}\PYG{p}{)}\PYG{p}{:}
        \PYG{n}{file\PYGZus{}} \PYG{o}{=} \PYG{n}{subdoc}\PYG{o}{.}\PYG{n}{macros}\PYG{p}{[}\PYG{l+s+s2}{\PYGZdq{}}\PYG{l+s+s2}{\PYGZus{}\PYGZus{}file\PYGZus{}\PYGZus{}}\PYG{l+s+s2}{\PYGZdq{}}\PYG{p}{]} \PYG{c+c1}{\PYGZsh{} preserve \PYGZus{}\PYGZus{}file\PYGZus{}\PYGZus{}}
        \PYG{n}{subdoc}\PYG{o}{.}\PYG{n}{macros}\PYG{o}{.}\PYG{n}{update}\PYG{p}{(}\PYG{n+nb+bp}{self}\PYG{o}{.}\PYG{n}{macros}\PYG{p}{)}
        \PYG{n}{subdoc}\PYG{o}{.}\PYG{n}{macros}\PYG{p}{[}\PYG{l+s+s2}{\PYGZdq{}}\PYG{l+s+s2}{\PYGZus{}\PYGZus{}file\PYGZus{}\PYGZus{}}\PYG{l+s+s2}{\PYGZdq{}}\PYG{p}{]} \PYG{o}{=} \PYG{n}{file\PYGZus{}}
\end{Verbatim}
\phantomsection\label{antiweb:read-the-source-file}
\textbf{\textless{}\textless{}read the source file\textgreater{}\textgreater{}}

\begin{Verbatim}[commandchars=\\\{\},numbers=left,firstnumber=1,stepnumber=1]
\PYG{n}{head}\PYG{p}{,} \PYG{n}{tail} \PYG{o}{=} \PYG{n}{os}\PYG{o}{.}\PYG{n}{path}\PYG{o}{.}\PYG{n}{split}\PYG{p}{(}\PYG{n+nb+bp}{self}\PYG{o}{.}\PYG{n}{fname}\PYG{p}{)}
\PYG{n}{fpath} \PYG{o}{=} \PYG{n}{os}\PYG{o}{.}\PYG{n}{path}\PYG{o}{.}\PYG{n}{join}\PYG{p}{(}\PYG{n}{head}\PYG{p}{,} \PYG{n}{rpath}\PYG{p}{)}

\PYG{k}{try}\PYG{p}{:}
    \PYG{c+c1}{\PYGZsh{}print \PYGZdq{}try open\PYGZdq{}, fpath}
    \PYG{k}{with} \PYG{n+nb}{open}\PYG{p}{(}\PYG{n}{fpath}\PYG{p}{,} \PYG{l+s+s2}{\PYGZdq{}}\PYG{l+s+s2}{r}\PYG{l+s+s2}{\PYGZdq{}}\PYG{p}{)} \PYG{k}{as} \PYG{n}{f}\PYG{p}{:}
        \PYG{n}{text} \PYG{o}{=} \PYG{n}{f}\PYG{o}{.}\PYG{n}{read}\PYG{p}{(}\PYG{p}{)}
\PYG{k}{except} \PYG{n+ne}{IOError}\PYG{p}{:}
    \PYG{n}{doc} \PYG{o}{=} \PYG{n+nb+bp}{None}
    \PYG{n}{logger}\PYG{o}{.}\PYG{n}{error}\PYG{p}{(}\PYG{l+s+s2}{\PYGZdq{}}\PYG{l+s+s2}{Could not open: }\PYG{l+s+si}{\PYGZpc{}s}\PYG{l+s+s2}{\PYGZdq{}}\PYG{p}{,} \PYG{n}{fpath}\PYG{p}{)}

\PYG{k}{else}\PYG{p}{:}
    \PYG{c+c1}{\PYGZsh{}parse the file}
    \PYG{n}{lexer} \PYG{o}{=} \PYG{n}{pm}\PYG{o}{.}\PYG{n}{get\PYGZus{}lexer\PYGZus{}for\PYGZus{}filename}\PYG{p}{(}\PYG{n}{rpath}\PYG{p}{)}
    \PYG{n}{reader} \PYG{o}{=} \PYG{n}{readers}\PYG{o}{.}\PYG{n}{get}\PYG{p}{(}\PYG{n}{lexer}\PYG{o}{.}\PYG{n}{name}\PYG{p}{,} \PYG{n}{Reader}\PYG{p}{)}\PYG{p}{(}\PYG{n}{lexer}\PYG{p}{)}
    \PYG{n}{doc} \PYG{o}{=} \PYG{n}{Document}\PYG{p}{(}\PYG{n}{text}\PYG{p}{,} \PYG{n}{reader}\PYG{p}{,} \PYG{n}{rpath}\PYG{p}{,} \PYG{n+nb+bp}{self}\PYG{o}{.}\PYG{n}{tokens}\PYG{p}{)}
    \PYG{n}{doc}\PYG{o}{.}\PYG{n}{collect\PYGZus{}blocks}\PYG{p}{(}\PYG{p}{)}
    \PYG{n}{insert\PYGZus{}macros}\PYG{p}{(}\PYG{n}{doc}\PYG{p}{)}
\end{Verbatim}

\end{fulllineitems}

\index{add\_error() (Document method)}

\begin{fulllineitems}
\phantomsection\label{antiweb:Document.add_error}\pysiglinewithargsret{\sphinxbfcode{add\_error}}{\emph{line}, \emph{text}}{}
Adds an error to the list.
:param integer line: The line number that causes the error.
:param string text: An error text.

\begin{Verbatim}[commandchars=\\\{\}]
\PYG{k}{def} \PYG{n+nf}{add\PYGZus{}error}\PYG{p}{(}\PYG{n+nb+bp}{self}\PYG{p}{,} \PYG{n}{line}\PYG{p}{,} \PYG{n}{text}\PYG{p}{)}\PYG{p}{:}
    \PYG{n+nb+bp}{self}\PYG{o}{.}\PYG{n}{errors}\PYG{o}{.}\PYG{n}{append}\PYG{p}{(}\PYG{p}{(}\PYG{n+nb+bp}{self}\PYG{o}{.}\PYG{n}{lines}\PYG{p}{[}\PYG{n}{line}\PYG{p}{]}\PYG{p}{,} \PYG{n}{text}\PYG{p}{)}\PYG{p}{)}
\end{Verbatim}

\end{fulllineitems}

\index{check\_errors() (Document method)}

\begin{fulllineitems}
\phantomsection\label{antiweb:Document.check_errors}\pysiglinewithargsret{\sphinxbfcode{check\_errors}}{}{}
Raises a \sphinxcode{WebError} exception if error were found.

\begin{Verbatim}[commandchars=\\\{\}]
\PYG{k}{def} \PYG{n+nf}{check\PYGZus{}errors}\PYG{p}{(}\PYG{n+nb+bp}{self}\PYG{p}{)}\PYG{p}{:}
    \PYG{k}{if} \PYG{n+nb+bp}{self}\PYG{o}{.}\PYG{n}{errors}\PYG{p}{:}
        \PYG{k}{raise} \PYG{n}{WebError}\PYG{p}{(}\PYG{n+nb+bp}{self}\PYG{o}{.}\PYG{n}{errors}\PYG{p}{)}
\end{Verbatim}

\end{fulllineitems}

\index{collect\_blocks() (Document method)}

\begin{fulllineitems}
\phantomsection\label{antiweb:Document.collect_blocks}\pysiglinewithargsret{\sphinxbfcode{collect\_blocks}}{}{}
Collects all text blocks.

\begin{Verbatim}[commandchars=\\\{\},numbers=left,firstnumber=1,stepnumber=1]
\PYG{k}{def} \PYG{n+nf}{collect\PYGZus{}blocks}\PYG{p}{(}\PYG{n+nb+bp}{self}\PYG{p}{)}\PYG{p}{:}
    \PYG{n}{blocks} \PYG{o}{=} \PYG{p}{[} \PYG{n}{d}\PYG{o}{.}\PYG{n}{collect\PYGZus{}block}\PYG{p}{(}\PYG{n+nb+bp}{self}\PYG{p}{,} \PYG{n}{i}\PYG{p}{)}
               \PYG{k}{for} \PYG{n}{i}\PYG{p}{,} \PYG{n}{l} \PYG{o+ow}{in} \PYG{n+nb}{enumerate}\PYG{p}{(}\PYG{n+nb+bp}{self}\PYG{o}{.}\PYG{n}{lines}\PYG{p}{)}
               \PYG{k}{for} \PYG{n}{d} \PYG{o+ow}{in} \PYG{n}{l}\PYG{o}{.}\PYG{n}{directives} \PYG{p}{]}

    \PYG{n+nb+bp}{self}\PYG{o}{.}\PYG{n}{blocks} \PYG{o}{=} \PYG{n+nb}{dict}\PYG{p}{(}\PYG{n+nb}{list}\PYG{p}{(}\PYG{n+nb}{filter}\PYG{p}{(}\PYG{n+nb}{bool}\PYG{p}{,} \PYG{n}{blocks}\PYG{p}{)}\PYG{p}{)}\PYG{p}{)}

    \PYG{k}{if} \PYG{l+s+s2}{\PYGZdq{}}\PYG{l+s+s2}{\PYGZus{}\PYGZus{}macros\PYGZus{}\PYGZus{}}\PYG{l+s+s2}{\PYGZdq{}} \PYG{o+ow}{in} \PYG{n+nb+bp}{self}\PYG{o}{.}\PYG{n}{blocks}\PYG{p}{:}
        \PYG{n+nb+bp}{self}\PYG{o}{.}\PYG{n}{get\PYGZus{}compiled\PYGZus{}block}\PYG{p}{(}\PYG{l+s+s2}{\PYGZdq{}}\PYG{l+s+s2}{\PYGZus{}\PYGZus{}macros\PYGZus{}\PYGZus{}}\PYG{l+s+s2}{\PYGZdq{}}\PYG{p}{)}
\end{Verbatim}

\end{fulllineitems}

\index{get\_compiled\_block() (Document method)}

\begin{fulllineitems}
\phantomsection\label{antiweb:Document.get_compiled_block}\pysiglinewithargsret{\sphinxbfcode{get\_compiled\_block}}{\emph{name}}{}
Returns the compiled version of a text block.
Compiled means: all directives where processed.
:param string name: The name of the text block:
:return: A list of {\hyperref[antiweb:Line]{\sphinxcrossref{\sphinxcode{Line}}}} objects representing
the text block.

\begin{Verbatim}[commandchars=\\\{\},numbers=left,firstnumber=1,stepnumber=1]
\PYG{k}{def} \PYG{n+nf}{get\PYGZus{}compiled\PYGZus{}block}\PYG{p}{(}\PYG{n+nb+bp}{self}\PYG{p}{,} \PYG{n}{name}\PYG{p}{)}\PYG{p}{:}
    \PYG{k}{if} \PYG{n}{name} \PYG{o+ow}{not} \PYG{o+ow}{in} \PYG{n+nb+bp}{self}\PYG{o}{.}\PYG{n}{blocks}\PYG{p}{:}
        \PYG{k}{return} \PYG{n+nb+bp}{None}

    \PYG{k}{if} \PYG{n}{name} \PYG{o+ow}{in} \PYG{n+nb+bp}{self}\PYG{o}{.}\PYG{n}{compiled\PYGZus{}blocks}\PYG{p}{:}
        \PYG{k}{return} \PYG{n+nb+bp}{self}\PYG{o}{.}\PYG{n}{blocks}\PYG{p}{[}\PYG{n}{name}\PYG{p}{]}

    \PYG{k}{return} \PYG{n+nb+bp}{self}\PYG{o}{.}\PYG{n}{compile\PYGZus{}block}\PYG{p}{(}\PYG{n}{name}\PYG{p}{,} \PYG{n+nb+bp}{self}\PYG{o}{.}\PYG{n}{blocks}\PYG{p}{[}\PYG{n}{name}\PYG{p}{]}\PYG{p}{)}
\end{Verbatim}

\end{fulllineitems}

\index{compile\_block() (Document method)}

\begin{fulllineitems}
\phantomsection\label{antiweb:Document.compile_block}\pysiglinewithargsret{\sphinxbfcode{compile\_block}}{\emph{name}, \emph{block}}{}
Compiles a text block.
:param string name: The name of the block
:param block: A list of {\hyperref[antiweb:Line]{\sphinxcrossref{\sphinxcode{Line}}}} objects representing
the text block to compile.
:return: A list of {\hyperref[antiweb:Line]{\sphinxcrossref{\sphinxcode{Line}}}} objects representing
the compiled text block.

\begin{Verbatim}[commandchars=\\\{\},numbers=left,firstnumber=1,stepnumber=1]
\PYG{k}{def} \PYG{n+nf}{compile\PYGZus{}block}\PYG{p}{(}\PYG{n+nb+bp}{self}\PYG{p}{,} \PYG{n}{name}\PYG{p}{,} \PYG{n}{block}\PYG{p}{)}\PYG{p}{:}
    \PYG{o}{\PYGZlt{}\PYGZlt{}}\PYG{n}{find\PYGZus{}next\PYGZus{}directive}\PYG{o}{\PYGZgt{}\PYGZgt{}}

    \PYG{k}{while} \PYG{n+nb+bp}{True}\PYG{p}{:}
        \PYG{n}{directive\PYGZus{}index} \PYG{o}{=} \PYG{n}{find\PYGZus{}next\PYGZus{}directive}\PYG{p}{(}\PYG{n}{block}\PYG{p}{)}
        \PYG{k}{if} \PYG{o+ow}{not} \PYG{n}{directive\PYGZus{}index}\PYG{p}{:} \PYG{k}{break}
        \PYG{n}{directive}\PYG{p}{,} \PYG{n}{index} \PYG{o}{=} \PYG{n}{directive\PYGZus{}index}
        \PYG{n}{directive}\PYG{o}{.}\PYG{n}{process}\PYG{p}{(}\PYG{n+nb+bp}{self}\PYG{p}{,} \PYG{n}{block}\PYG{p}{,} \PYG{n}{index}\PYG{p}{)}

    \PYG{n+nb+bp}{self}\PYG{o}{.}\PYG{n}{compiled\PYGZus{}blocks}\PYG{o}{.}\PYG{n}{add}\PYG{p}{(}\PYG{n}{name}\PYG{p}{)}
    \PYG{k}{return} \PYG{n}{block}
\end{Verbatim}
\phantomsection\label{antiweb:find-next-directive}
\textbf{\textless{}\textless{}find\_next\_directive\textgreater{}\textgreater{}}

\begin{Verbatim}[commandchars=\\\{\},numbers=left,firstnumber=1,stepnumber=1]
\PYG{k}{def} \PYG{n+nf}{find\PYGZus{}next\PYGZus{}directive}\PYG{p}{(}\PYG{n}{block}\PYG{p}{)}\PYG{p}{:}
    \PYG{c+c1}{\PYGZsh{} returns the next available directive}
    \PYG{n}{min\PYGZus{}line} \PYG{o}{=} \PYG{p}{[} \PYG{p}{(}\PYG{n}{l}\PYG{o}{.}\PYG{n}{directives}\PYG{p}{[}\PYG{l+m+mi}{0}\PYG{p}{]}\PYG{o}{.}\PYG{n}{priority}\PYG{p}{,} \PYG{n}{i}\PYG{p}{)}
                 \PYG{k}{for} \PYG{n}{i}\PYG{p}{,} \PYG{n}{l} \PYG{o+ow}{in} \PYG{n+nb}{enumerate}\PYG{p}{(}\PYG{n}{block}\PYG{p}{)} \PYG{k}{if} \PYG{n}{l}\PYG{o}{.}\PYG{n}{directives} \PYG{p}{]}
    \PYG{k}{if} \PYG{o+ow}{not} \PYG{n}{min\PYGZus{}line}\PYG{p}{:}
        \PYG{k}{return} \PYG{n+nb+bp}{None}

    \PYG{n}{prio}\PYG{p}{,} \PYG{n}{index} \PYG{o}{=} \PYG{n+nb}{min}\PYG{p}{(}\PYG{n}{min\PYGZus{}line}\PYG{p}{)}
    \PYG{k}{return} \PYG{n}{block}\PYG{p}{[}\PYG{n}{index}\PYG{p}{]}\PYG{o}{.}\PYG{n}{directives}\PYG{o}{.}\PYG{n}{pop}\PYG{p}{(}\PYG{l+m+mi}{0}\PYG{p}{)}\PYG{p}{,} \PYG{n}{index}
\end{Verbatim}

\end{fulllineitems}


\end{fulllineitems}



\subsection{Line}
\label{antiweb:line}\index{Line (built-in class)}

\begin{fulllineitems}
\phantomsection\label{antiweb:Line}\pysiglinewithargsret{\sphinxstrong{class }\sphinxbfcode{Line}}{\emph{fname}, \emph{index}, \emph{text}\sphinxoptional{, \emph{directives}\sphinxoptional{, \emph{type}}}}{}
This class represents a text line.

\begin{Verbatim}[commandchars=\\\{\},numbers=left,firstnumber=1,stepnumber=1]
\PYG{k}{class} \PYG{n+nc}{Line}\PYG{p}{(}\PYG{n+nb}{object}\PYG{p}{)}\PYG{p}{:}
    \PYG{c+c1}{\PYGZsh{}Attributes}
    \PYG{o}{\PYGZlt{}\PYGZlt{}}\PYG{n}{Line}\PYG{o}{.}\PYG{n}{\PYGZus{}directives}\PYG{o}{\PYGZgt{}\PYGZgt{}}
    \PYG{o}{\PYGZlt{}\PYGZlt{}}\PYG{n}{Line}\PYG{o}{.}\PYG{n}{fname}\PYG{o}{\PYGZgt{}\PYGZgt{}}
    \PYG{o}{\PYGZlt{}\PYGZlt{}}\PYG{n}{Line}\PYG{o}{.}\PYG{n}{index}\PYG{o}{\PYGZgt{}\PYGZgt{}}
    \PYG{o}{\PYGZlt{}\PYGZlt{}}\PYG{n}{Line}\PYG{o}{.}\PYG{n}{text}\PYG{o}{\PYGZgt{}\PYGZgt{}}
    \PYG{o}{\PYGZlt{}\PYGZlt{}}\PYG{n}{Line}\PYG{o}{.}\PYG{n}{type}\PYG{o}{\PYGZgt{}\PYGZgt{}}

    \PYG{c+c1}{\PYGZsh{}Methods}
    \PYG{o}{\PYGZlt{}\PYGZlt{}}\PYG{n}{Line}\PYG{o}{.}\PYG{n}{\PYGZus{}\PYGZus{}init\PYGZus{}\PYGZus{}}\PYG{o}{\PYGZgt{}\PYGZgt{}}
    \PYG{o}{\PYGZlt{}\PYGZlt{}}\PYG{n}{Line}\PYG{o}{.}\PYG{n}{set}\PYG{o}{\PYGZgt{}\PYGZgt{}}
    \PYG{o}{\PYGZlt{}\PYGZlt{}}\PYG{n}{Line}\PYG{o}{.}\PYG{n}{like}\PYG{o}{\PYGZgt{}\PYGZgt{}}
    \PYG{o}{\PYGZlt{}\PYGZlt{}}\PYG{n}{Line}\PYG{o}{.}\PYG{n}{indented}\PYG{o}{\PYGZgt{}\PYGZgt{}}
    \PYG{o}{\PYGZlt{}\PYGZlt{}}\PYG{n}{Line}\PYG{o}{.}\PYG{n}{change\PYGZus{}indent}\PYG{o}{\PYGZgt{}\PYGZgt{}}
    \PYG{o}{\PYGZlt{}\PYGZlt{}}\PYG{n}{Line}\PYG{o}{.}\PYG{n}{\PYGZus{}\PYGZus{}len\PYGZus{}\PYGZus{}}\PYG{o}{\PYGZgt{}\PYGZgt{}}
    \PYG{o}{\PYGZlt{}\PYGZlt{}}\PYG{n}{Line}\PYG{o}{.}\PYG{n}{\PYGZus{}\PYGZus{}repr\PYGZus{}\PYGZus{}}\PYG{o}{\PYGZgt{}\PYGZgt{}}

    \PYG{c+c1}{\PYGZsh{}Properties}
    \PYG{o}{\PYGZlt{}\PYGZlt{}}\PYG{n}{Line}\PYG{o}{.}\PYG{n}{indent}\PYG{o}{\PYGZgt{}\PYGZgt{}}
    \PYG{o}{\PYGZlt{}\PYGZlt{}}\PYG{n}{Line}\PYG{o}{.}\PYG{n}{sindent}\PYG{o}{\PYGZgt{}\PYGZgt{}}
    \PYG{o}{\PYGZlt{}\PYGZlt{}}\PYG{n}{Line}\PYG{o}{.}\PYG{n}{directives}\PYG{o}{\PYGZgt{}\PYGZgt{}}
    \PYG{o}{\PYGZlt{}\PYGZlt{}}\PYG{n}{Line}\PYG{o}{.}\PYG{n}{directive}\PYG{o}{\PYGZgt{}\PYGZgt{}}
\end{Verbatim}
\index{\_directives (Line attribute)}

\begin{fulllineitems}
\phantomsection\label{antiweb:Line._directives}\pysigline{\sphinxbfcode{\_directives}}
A list of {\hyperref[antiweb:Directive]{\sphinxcrossref{\sphinxcode{Directive}}}} objects, sorted
by their priority.

\begin{Verbatim}[commandchars=\\\{\}]
\PYG{n}{\PYGZus{}directives} \PYG{o}{=} \PYG{p}{(}\PYG{p}{)}
\end{Verbatim}

\end{fulllineitems}

\index{fname (Line attribute)}

\begin{fulllineitems}
\phantomsection\label{antiweb:Line.fname}\pysigline{\sphinxbfcode{fname}}
A string of the source's file name the line belongs to.

\begin{Verbatim}[commandchars=\\\{\}]
\PYG{n}{fname} \PYG{o}{=} \PYG{l+s+s2}{\PYGZdq{}}\PYG{l+s+s2}{\PYGZdq{}}
\end{Verbatim}

\end{fulllineitems}

\index{index (Line attribute)}

\begin{fulllineitems}
\phantomsection\label{antiweb:Line.index}\pysigline{\sphinxbfcode{index}}
The integer line index of the directive within the current block.

\begin{Verbatim}[commandchars=\\\{\}]
\PYG{n}{index} \PYG{o}{=} \PYG{l+m+mi}{0}
\end{Verbatim}

\end{fulllineitems}

\index{text (Line attribute)}

\begin{fulllineitems}
\phantomsection\label{antiweb:Line.text}\pysigline{\sphinxbfcode{text}}
A string containing the source line.

\begin{Verbatim}[commandchars=\\\{\}]
\PYG{n}{text} \PYG{o}{=} \PYG{l+s+s2}{\PYGZdq{}}\PYG{l+s+s2}{\PYGZdq{}}
\end{Verbatim}

\end{fulllineitems}

\index{type (Line attribute)}

\begin{fulllineitems}
\phantomsection\label{antiweb:Line.type}\pysigline{\sphinxbfcode{type}}
A char representing the line type:
\begin{itemize}
\item {} 
\sphinxcode{d} stands for a document line

\item {} 
\sphinxcode{c} stands for a code line

\end{itemize}

\begin{Verbatim}[commandchars=\\\{\}]
\PYG{n+nb}{type} \PYG{o}{=} \PYG{l+s+s2}{\PYGZdq{}}\PYG{l+s+s2}{d}\PYG{l+s+s2}{\PYGZdq{}}
\end{Verbatim}

\end{fulllineitems}

\index{indent (Line attribute)}

\begin{fulllineitems}
\phantomsection\label{antiweb:Line.indent}\pysigline{\sphinxbfcode{indent}}
\end{fulllineitems}


An integer representing the line's indentation.

\begin{Verbatim}[commandchars=\\\{\}]
\PYG{n+nd}{@property}
\PYG{k}{def} \PYG{n+nf}{indent}\PYG{p}{(}\PYG{n+nb+bp}{self}\PYG{p}{)}\PYG{p}{:}
    \PYG{k}{return} \PYG{n+nb}{len}\PYG{p}{(}\PYG{n+nb+bp}{self}\PYG{o}{.}\PYG{n}{text}\PYG{p}{)}\PYG{o}{\PYGZhy{}}\PYG{n+nb}{len}\PYG{p}{(}\PYG{n+nb+bp}{self}\PYG{o}{.}\PYG{n}{text}\PYG{o}{.}\PYG{n}{lstrip}\PYG{p}{(}\PYG{p}{)}\PYG{p}{)}
\end{Verbatim}
\index{sindent (Line attribute)}

\begin{fulllineitems}
\phantomsection\label{antiweb:Line.sindent}\pysigline{\sphinxbfcode{sindent}}
\end{fulllineitems}


A string representation of the line's indentation.

\begin{Verbatim}[commandchars=\\\{\}]
\PYG{n+nd}{@property}
\PYG{k}{def} \PYG{n+nf}{sindent}\PYG{p}{(}\PYG{n+nb+bp}{self}\PYG{p}{)}\PYG{p}{:}
    \PYG{k}{return} \PYG{l+s+s2}{\PYGZdq{}}\PYG{l+s+s2}{ }\PYG{l+s+s2}{\PYGZdq{}}\PYG{o}{*}\PYG{n+nb+bp}{self}\PYG{o}{.}\PYG{n}{indent}
\end{Verbatim}
\index{directives (Line attribute)}

\begin{fulllineitems}
\phantomsection\label{antiweb:Line.directives}\pysigline{\sphinxbfcode{directives}}
\end{fulllineitems}


A sorted sequence of {\hyperref[antiweb:Directive]{\sphinxcrossref{\sphinxcode{Directive}}}} objects.

\begin{Verbatim}[commandchars=\\\{\},numbers=left,firstnumber=1,stepnumber=1]
\PYG{n+nd}{@property}
\PYG{k}{def} \PYG{n+nf}{directives}\PYG{p}{(}\PYG{n+nb+bp}{self}\PYG{p}{)}\PYG{p}{:}
    \PYG{k}{return} \PYG{n+nb+bp}{self}\PYG{o}{.}\PYG{n}{\PYGZus{}directives}


\PYG{n+nd}{@directives.setter}
\PYG{k}{def} \PYG{n+nf}{directives}\PYG{p}{(}\PYG{n+nb+bp}{self}\PYG{p}{,} \PYG{n}{value}\PYG{p}{)}\PYG{p}{:}
    \PYG{n+nb+bp}{self}\PYG{o}{.}\PYG{n}{\PYGZus{}directives} \PYG{o}{=} \PYG{n}{value}\PYG{p}{[}\PYG{p}{:}\PYG{p}{]}
    \PYG{k}{if} \PYG{n+nb+bp}{self}\PYG{o}{.}\PYG{n}{\PYGZus{}directives}\PYG{p}{:}
        \PYG{n+nb+bp}{self}\PYG{o}{.}\PYG{n}{\PYGZus{}directives}\PYG{o}{.}\PYG{n}{sort}\PYG{p}{(}\PYG{n}{key}\PYG{o}{=}\PYG{n}{operator}\PYG{o}{.}\PYG{n}{attrgetter}\PYG{p}{(}\PYG{l+s+s2}{\PYGZdq{}}\PYG{l+s+s2}{priority}\PYG{l+s+s2}{\PYGZdq{}}\PYG{p}{)}\PYG{p}{)}
\end{Verbatim}
\index{directive (Line attribute)}

\begin{fulllineitems}
\phantomsection\label{antiweb:Line.directive}\pysigline{\sphinxbfcode{directive}}
The first of the contained {\hyperref[antiweb:Directive]{\sphinxcrossref{\sphinxcode{Directive}}}} objects.

\begin{Verbatim}[commandchars=\\\{\}]
\PYG{n+nd}{@property}
\PYG{k}{def} \PYG{n+nf}{directive}\PYG{p}{(}\PYG{n+nb+bp}{self}\PYG{p}{)}\PYG{p}{:}
    \PYG{k}{return} \PYG{n+nb+bp}{self}\PYG{o}{.}\PYG{n}{directives} \PYG{o+ow}{and} \PYG{n+nb+bp}{self}\PYG{o}{.}\PYG{n}{directives}\PYG{p}{[}\PYG{l+m+mi}{0}\PYG{p}{]}
\end{Verbatim}

\end{fulllineitems}

\index{\_\_init\_\_() (Line method)}

\begin{fulllineitems}
\phantomsection\label{antiweb:Line.__init__}\pysiglinewithargsret{\sphinxbfcode{\_\_init\_\_}}{\emph{name}, \emph{index}, \emph{text}\sphinxoptional{, \emph{directives}\sphinxoptional{, \emph{type}}}}{}
The constructor.

\begin{Verbatim}[commandchars=\\\{\},numbers=left,firstnumber=1,stepnumber=1]
\PYG{k}{def} \PYG{n+nf}{\PYGZus{}\PYGZus{}init\PYGZus{}\PYGZus{}}\PYG{p}{(}\PYG{n+nb+bp}{self}\PYG{p}{,} \PYG{n}{fname}\PYG{p}{,} \PYG{n}{index}\PYG{p}{,} \PYG{n}{text}\PYG{p}{,} \PYG{n}{directives}\PYG{o}{=}\PYG{p}{(}\PYG{p}{)}\PYG{p}{,} \PYG{n+nb}{type}\PYG{o}{=}\PYG{l+s+s1}{\PYGZsq{}}\PYG{l+s+s1}{d}\PYG{l+s+s1}{\PYGZsq{}}\PYG{p}{)}\PYG{p}{:}
    \PYG{n+nb+bp}{self}\PYG{o}{.}\PYG{n}{fname} \PYG{o}{=} \PYG{n}{fname}
    \PYG{n+nb+bp}{self}\PYG{o}{.}\PYG{n}{index} \PYG{o}{=} \PYG{n}{index}
    \PYG{n+nb+bp}{self}\PYG{o}{.}\PYG{n}{text} \PYG{o}{=} \PYG{n}{text}
    \PYG{n+nb+bp}{self}\PYG{o}{.}\PYG{n}{directives} \PYG{o}{=} \PYG{n}{directives}
    \PYG{n+nb+bp}{self}\PYG{o}{.}\PYG{n}{type} \PYG{o}{=} \PYG{n+nb}{type}
\end{Verbatim}

\end{fulllineitems}

\index{set() (Line method)}

\begin{fulllineitems}
\phantomsection\label{antiweb:Line.set}\pysiglinewithargsret{\sphinxbfcode{set}}{\sphinxoptional{\emph{index=None}\sphinxoptional{, \emph{type=None}\sphinxoptional{, \emph{directives=None}}}}}{}
Changes the attributes {\hyperref[antiweb:Line.index]{\sphinxcrossref{\sphinxcode{index}}}}, {\hyperref[antiweb:Line.type]{\sphinxcrossref{\sphinxcode{type}}}}
and {\hyperref[antiweb:Line.directives]{\sphinxcrossref{\sphinxcode{directives}}}} at once.
\begin{quote}\begin{description}
\item[{Parameters}] \leavevmode\begin{itemize}
\item {} 
\textbf{\texttt{index}} (\emph{\texttt{integer}}) -- the line index.

\item {} 
\textbf{\texttt{type}} (\emph{\texttt{char}}) -- Either \sphinxcode{'d'} or \sphinxcode{'c'}.

\item {} 
\textbf{\texttt{directives}} (\emph{\texttt{list}}) -- A list of \sphinxcode{DCirective} objects.

\end{itemize}

\item[{Returns}] \leavevmode
The {\hyperref[antiweb:Line]{\sphinxcrossref{\sphinxcode{Line}}}} object \sphinxcode{self}.

\end{description}\end{quote}

\begin{Verbatim}[commandchars=\\\{\},numbers=left,firstnumber=1,stepnumber=1]
\PYG{k}{def} \PYG{n+nf}{set}\PYG{p}{(}\PYG{n+nb+bp}{self}\PYG{p}{,} \PYG{n}{index}\PYG{o}{=}\PYG{n+nb+bp}{None}\PYG{p}{,} \PYG{n+nb}{type}\PYG{o}{=}\PYG{n+nb+bp}{None}\PYG{p}{,} \PYG{n}{directives}\PYG{o}{=}\PYG{n+nb+bp}{None}\PYG{p}{)}\PYG{p}{:}
    \PYG{k}{if} \PYG{n}{index} \PYG{o+ow}{is} \PYG{o+ow}{not} \PYG{n+nb+bp}{None}\PYG{p}{:}
        \PYG{n+nb+bp}{self}\PYG{o}{.}\PYG{n}{index} \PYG{o}{=} \PYG{n}{index}

    \PYG{k}{if} \PYG{n+nb}{type} \PYG{o+ow}{is} \PYG{o+ow}{not} \PYG{n+nb+bp}{None}\PYG{p}{:}
        \PYG{n+nb+bp}{self}\PYG{o}{.}\PYG{n}{type} \PYG{o}{=} \PYG{n+nb}{type}

    \PYG{k}{if} \PYG{n}{directives} \PYG{o+ow}{is} \PYG{o+ow}{not} \PYG{n+nb+bp}{None}\PYG{p}{:}
        \PYG{n+nb+bp}{self}\PYG{o}{.}\PYG{n}{directives} \PYG{o}{=} \PYG{n}{directives}

    \PYG{k}{return} \PYG{n+nb+bp}{self}



\PYG{k}{def} \PYG{n+nf}{clone}\PYG{p}{(}\PYG{n+nb+bp}{self}\PYG{p}{,} \PYG{n}{dline}\PYG{o}{=}\PYG{n+nb+bp}{None}\PYG{p}{)}\PYG{p}{:}

    \PYG{k}{if} \PYG{n}{dline} \PYG{o+ow}{is} \PYG{o+ow}{not} \PYG{n+nb+bp}{None}\PYG{p}{:}
        \PYG{k}{for} \PYG{n}{d} \PYG{o+ow}{in} \PYG{n+nb+bp}{self}\PYG{o}{.}\PYG{n}{directives}\PYG{p}{:}
            \PYG{n}{d}\PYG{o}{.}\PYG{n}{line} \PYG{o}{=} \PYG{n}{dline}

    \PYG{k}{return} \PYG{n}{Line}\PYG{p}{(}\PYG{n+nb+bp}{self}\PYG{o}{.}\PYG{n}{fname}\PYG{p}{,} \PYG{n+nb+bp}{self}\PYG{o}{.}\PYG{n}{index}\PYG{p}{,} \PYG{n+nb+bp}{self}\PYG{o}{.}\PYG{n}{text}\PYG{p}{,}
                \PYG{n+nb+bp}{self}\PYG{o}{.}\PYG{n}{directives}\PYG{p}{[}\PYG{p}{:}\PYG{p}{]}\PYG{p}{,} \PYG{n+nb+bp}{self}\PYG{o}{.}\PYG{n}{type}\PYG{p}{)}
\end{Verbatim}

\end{fulllineitems}

\index{like() (Line method)}

\begin{fulllineitems}
\phantomsection\label{antiweb:Line.like}\pysiglinewithargsret{\sphinxbfcode{like}}{\emph{text}}{}
Clones the Line with a different text.

\begin{Verbatim}[commandchars=\\\{\}]
\PYG{k}{def} \PYG{n+nf}{like}\PYG{p}{(}\PYG{n+nb+bp}{self}\PYG{p}{,} \PYG{n}{text}\PYG{p}{)}\PYG{p}{:}
    \PYG{k}{return} \PYG{n}{Line}\PYG{p}{(}\PYG{n+nb+bp}{self}\PYG{o}{.}\PYG{n}{fname}\PYG{p}{,} \PYG{n+nb+bp}{self}\PYG{o}{.}\PYG{n}{index}\PYG{p}{,} \PYG{n+nb+bp}{self}\PYG{o}{.}\PYG{n}{indented}\PYG{p}{(}\PYG{n}{text}\PYG{p}{)}\PYG{p}{)}
\end{Verbatim}

\end{fulllineitems}

\index{indented() (Line method)}

\begin{fulllineitems}
\phantomsection\label{antiweb:Line.indented}\pysiglinewithargsret{\sphinxbfcode{indented}}{\emph{text}}{}
Returns the text, with the same indentation as \sphinxcode{self}.

\begin{Verbatim}[commandchars=\\\{\}]
\PYG{k}{def} \PYG{n+nf}{indented}\PYG{p}{(}\PYG{n+nb+bp}{self}\PYG{p}{,} \PYG{n}{text}\PYG{p}{)}\PYG{p}{:}
    \PYG{k}{return} \PYG{n+nb+bp}{self}\PYG{o}{.}\PYG{n}{sindent} \PYG{o}{+} \PYG{n}{text}
\end{Verbatim}

\end{fulllineitems}

\index{change\_indent() (Line method)}

\begin{fulllineitems}
\phantomsection\label{antiweb:Line.change_indent}\pysiglinewithargsret{\sphinxbfcode{change\_indent}}{\emph{delta}}{}
Changes the lines indentation.

\begin{Verbatim}[commandchars=\\\{\},numbers=left,firstnumber=1,stepnumber=1]
\PYG{k}{def} \PYG{n+nf}{change\PYGZus{}indent}\PYG{p}{(}\PYG{n+nb+bp}{self}\PYG{p}{,} \PYG{n}{delta}\PYG{p}{)}\PYG{p}{:}
    \PYG{k}{if} \PYG{n}{delta} \PYG{o}{\PYGZlt{}} \PYG{l+m+mi}{0}\PYG{p}{:}
        \PYG{n}{delta} \PYG{o}{=} \PYG{n+nb}{min}\PYG{p}{(}\PYG{o}{\PYGZhy{}}\PYG{n}{delta}\PYG{p}{,} \PYG{n+nb+bp}{self}\PYG{o}{.}\PYG{n}{indent}\PYG{p}{)}
        \PYG{n+nb+bp}{self}\PYG{o}{.}\PYG{n}{text} \PYG{o}{=} \PYG{n+nb+bp}{self}\PYG{o}{.}\PYG{n}{text}\PYG{p}{[}\PYG{n}{delta}\PYG{p}{:}\PYG{p}{]}

    \PYG{k}{elif} \PYG{n}{delta} \PYG{o}{\PYGZgt{}} \PYG{l+m+mi}{0}\PYG{p}{:}
        \PYG{n+nb+bp}{self}\PYG{o}{.}\PYG{n}{text} \PYG{o}{=} \PYG{l+s+s2}{\PYGZdq{}}\PYG{l+s+s2}{ }\PYG{l+s+s2}{\PYGZdq{}}\PYG{o}{*}\PYG{n}{delta} \PYG{o}{+} \PYG{n+nb+bp}{self}\PYG{o}{.}\PYG{n}{text}

    \PYG{k}{return} \PYG{n+nb+bp}{self}
\end{Verbatim}

\end{fulllineitems}

\index{\_\_len\_\_() (Line method)}

\begin{fulllineitems}
\phantomsection\label{antiweb:Line.__len__}\pysiglinewithargsret{\sphinxbfcode{\_\_len\_\_}}{}{}
returns the length of the stripped {\hyperref[antiweb:Line.text]{\sphinxcrossref{\sphinxcode{text}}}}.

\begin{Verbatim}[commandchars=\\\{\}]
\PYG{k}{def} \PYG{n+nf}{\PYGZus{}\PYGZus{}len\PYGZus{}\PYGZus{}}\PYG{p}{(}\PYG{n+nb+bp}{self}\PYG{p}{)}\PYG{p}{:}
    \PYG{k}{return} \PYG{n+nb}{len}\PYG{p}{(}\PYG{n+nb+bp}{self}\PYG{o}{.}\PYG{n}{text}\PYG{o}{.}\PYG{n}{strip}\PYG{p}{(}\PYG{p}{)}\PYG{p}{)}
\end{Verbatim}

\end{fulllineitems}

\index{\_\_repr\_\_() (Line method)}

\begin{fulllineitems}
\phantomsection\label{antiweb:Line.__repr__}\pysiglinewithargsret{\sphinxbfcode{\_\_repr\_\_}}{}{}
returns a textual representation of the line.

\begin{Verbatim}[commandchars=\\\{\}]
\PYG{k}{def} \PYG{n+nf}{\PYGZus{}\PYGZus{}repr\PYGZus{}\PYGZus{}}\PYG{p}{(}\PYG{n+nb+bp}{self}\PYG{p}{)}\PYG{p}{:}
    \PYG{k}{return} \PYG{l+s+s2}{\PYGZdq{}}\PYG{l+s+s2}{Line(}\PYG{l+s+si}{\PYGZpc{}i}\PYG{l+s+s2}{, }\PYG{l+s+si}{\PYGZpc{}s}\PYG{l+s+s2}{, }\PYG{l+s+si}{\PYGZpc{}s}\PYG{l+s+s2}{)}\PYG{l+s+s2}{\PYGZdq{}} \PYG{o}{\PYGZpc{}} \PYG{p}{(}\PYG{n+nb+bp}{self}\PYG{o}{.}\PYG{n}{index}\PYG{p}{,} \PYG{n+nb+bp}{self}\PYG{o}{.}\PYG{n}{text}\PYG{p}{,} \PYG{n+nb}{str}\PYG{p}{(}\PYG{n+nb+bp}{self}\PYG{o}{.}\PYG{n}{directives}\PYG{p}{)}\PYG{p}{)}
\end{Verbatim}

\end{fulllineitems}


\end{fulllineitems}



\section{File Layout}
\label{antiweb:file-layout}
\begin{Verbatim}[commandchars=\\\{\},numbers=left,firstnumber=1,stepnumber=1]
\PYG{o}{\PYGZlt{}\PYGZlt{}}\PYG{n}{imports}\PYG{o}{\PYGZgt{}\PYGZgt{}}
\PYG{o}{\PYGZlt{}\PYGZlt{}}\PYG{n}{management}\PYG{o}{\PYGZgt{}\PYGZgt{}}
\PYG{o}{\PYGZlt{}\PYGZlt{}}\PYG{n}{directives}\PYG{o}{\PYGZgt{}\PYGZgt{}}
\PYG{o}{\PYGZlt{}\PYGZlt{}}\PYG{n}{readers}\PYG{o}{\PYGZgt{}\PYGZgt{}}


\PYG{o}{\PYGZlt{}\PYGZlt{}}\PYG{n}{Line}\PYG{o}{\PYGZgt{}\PYGZgt{}}
\PYG{o}{\PYGZlt{}\PYGZlt{}}\PYG{n}{document}\PYG{o}{\PYGZgt{}\PYGZgt{}}

\PYG{k}{def} \PYG{n+nf}{write\PYGZus{}static}\PYG{p}{(}\PYG{n}{input\PYGZus{}type}\PYG{p}{,} \PYG{n}{index\PYGZus{}rst}\PYG{p}{,} \PYG{n}{startblock}\PYG{p}{,} \PYG{n}{endblock}\PYG{p}{)}\PYG{p}{:}
    \PYG{n}{index\PYGZus{}static} \PYG{o}{=} \PYG{l+s+s2}{\PYGZdq{}}\PYG{l+s+s2}{Documentation}\PYG{l+s+se}{\PYGZbs{}n}\PYG{l+s+s2}{=======================}\PYG{l+s+se}{\PYGZbs{}n}\PYG{l+s+s2}{Contents:}\PYG{l+s+se}{\PYGZbs{}n}\PYG{l+s+se}{\PYGZbs{}n}\PYG{l+s+s2}{.. toctree::}\PYG{l+s+se}{\PYGZbs{}n}\PYG{l+s+s2}{   :maxdepth: 2}\PYG{l+s+se}{\PYGZbs{}n}\PYG{l+s+se}{\PYGZbs{}n}\PYG{l+s+s2}{   }\PYG{l+s+s2}{\PYGZdq{}} \PYG{o}{+} \PYG{n}{startblock} \PYG{o}{+}\PYG{l+s+s2}{\PYGZdq{}}\PYG{l+s+se}{\PYGZbs{}n}\PYG{l+s+s2}{   }\PYG{l+s+s2}{\PYGZdq{}} \PYG{o}{+} \PYG{n}{endblock}
    \PYG{n}{index\PYGZus{}out} \PYG{o}{=} \PYG{n+nb}{open}\PYG{p}{(}\PYG{n}{os}\PYG{o}{.}\PYG{n}{path}\PYG{o}{.}\PYG{n}{join}\PYG{p}{(}\PYG{n}{input\PYGZus{}type}\PYG{p}{,} \PYG{n}{index\PYGZus{}rst}\PYG{p}{)}\PYG{p}{,} \PYG{l+s+s2}{\PYGZdq{}}\PYG{l+s+s2}{w}\PYG{l+s+s2}{\PYGZdq{}}\PYG{p}{)}
    \PYG{n}{index\PYGZus{}out}\PYG{o}{.}\PYG{n}{write}\PYG{p}{(}\PYG{n}{index\PYGZus{}static}\PYG{p}{)}

\PYG{k}{def} \PYG{n+nf}{main}\PYG{p}{(}\PYG{p}{)}\PYG{p}{:}
    \PYG{n}{parser} \PYG{o}{=} \PYG{n}{OptionParser}\PYG{p}{(}\PYG{l+s+s2}{\PYGZdq{}}\PYG{l+s+s2}{usage: }\PYG{l+s+s2}{\PYGZpc{}}\PYG{l+s+s2}{prog [options] SOURCEFILE}\PYG{l+s+s2}{\PYGZdq{}}\PYG{p}{,}
                          \PYG{n}{description}\PYG{o}{=}\PYG{l+s+s2}{\PYGZdq{}}\PYG{l+s+s2}{Tangles a source code file to a rst file.}\PYG{l+s+s2}{\PYGZdq{}}\PYG{p}{,}
                          \PYG{n}{version}\PYG{o}{=}\PYG{l+s+s2}{\PYGZdq{}}\PYG{l+s+s2}{\PYGZpc{}}\PYG{l+s+s2}{prog }\PYG{l+s+s2}{\PYGZdq{}} \PYG{o}{+} \PYG{n}{\PYGZus{}\PYGZus{}version\PYGZus{}\PYGZus{}}\PYG{p}{)}

    \PYG{n}{parser}\PYG{o}{.}\PYG{n}{add\PYGZus{}option}\PYG{p}{(}\PYG{l+s+s2}{\PYGZdq{}}\PYG{l+s+s2}{\PYGZhy{}o}\PYG{l+s+s2}{\PYGZdq{}}\PYG{p}{,} \PYG{l+s+s2}{\PYGZdq{}}\PYG{l+s+s2}{\PYGZhy{}\PYGZhy{}output}\PYG{l+s+s2}{\PYGZdq{}}\PYG{p}{,} \PYG{n}{dest}\PYG{o}{=}\PYG{l+s+s2}{\PYGZdq{}}\PYG{l+s+s2}{output}\PYG{l+s+s2}{\PYGZdq{}}\PYG{p}{,} \PYG{n}{default}\PYG{o}{=}\PYG{l+s+s2}{\PYGZdq{}}\PYG{l+s+s2}{\PYGZdq{}}\PYG{p}{,}
                      \PYG{n+nb}{type}\PYG{o}{=}\PYG{l+s+s2}{\PYGZdq{}}\PYG{l+s+s2}{string}\PYG{l+s+s2}{\PYGZdq{}}\PYG{p}{,} \PYG{n}{help}\PYG{o}{=}\PYG{l+s+s2}{\PYGZdq{}}\PYG{l+s+s2}{The output filename}\PYG{l+s+s2}{\PYGZdq{}}\PYG{p}{)}

    \PYG{n}{parser}\PYG{o}{.}\PYG{n}{add\PYGZus{}option}\PYG{p}{(}\PYG{l+s+s2}{\PYGZdq{}}\PYG{l+s+s2}{\PYGZhy{}t}\PYG{l+s+s2}{\PYGZdq{}}\PYG{p}{,} \PYG{l+s+s2}{\PYGZdq{}}\PYG{l+s+s2}{\PYGZhy{}\PYGZhy{}token}\PYG{l+s+s2}{\PYGZdq{}}\PYG{p}{,} \PYG{n}{dest}\PYG{o}{=}\PYG{l+s+s2}{\PYGZdq{}}\PYG{l+s+s2}{token}\PYG{l+s+s2}{\PYGZdq{}}\PYG{p}{,} \PYG{n}{action}\PYG{o}{=}\PYG{l+s+s2}{\PYGZdq{}}\PYG{l+s+s2}{append}\PYG{l+s+s2}{\PYGZdq{}}\PYG{p}{,}
                      \PYG{n+nb}{type}\PYG{o}{=}\PYG{l+s+s2}{\PYGZdq{}}\PYG{l+s+s2}{string}\PYG{l+s+s2}{\PYGZdq{}}\PYG{p}{,} \PYG{n}{help}\PYG{o}{=}\PYG{l+s+s2}{\PYGZdq{}}\PYG{l+s+s2}{defines a token, usable by @if directives}\PYG{l+s+s2}{\PYGZdq{}}\PYG{p}{)}

    \PYG{n}{parser}\PYG{o}{.}\PYG{n}{add\PYGZus{}option}\PYG{p}{(}\PYG{l+s+s2}{\PYGZdq{}}\PYG{l+s+s2}{\PYGZhy{}w}\PYG{l+s+s2}{\PYGZdq{}}\PYG{p}{,} \PYG{l+s+s2}{\PYGZdq{}}\PYG{l+s+s2}{\PYGZhy{}\PYGZhy{}warnings}\PYG{l+s+s2}{\PYGZdq{}}\PYG{p}{,} \PYG{n}{dest}\PYG{o}{=}\PYG{l+s+s2}{\PYGZdq{}}\PYG{l+s+s2}{warnings}\PYG{l+s+s2}{\PYGZdq{}}\PYG{p}{,}
                      \PYG{n}{action}\PYG{o}{=}\PYG{l+s+s2}{\PYGZdq{}}\PYG{l+s+s2}{store\PYGZus{}false}\PYG{l+s+s2}{\PYGZdq{}}\PYG{p}{,} \PYG{n}{help}\PYG{o}{=}\PYG{l+s+s2}{\PYGZdq{}}\PYG{l+s+s2}{suppresses warnings}\PYG{l+s+s2}{\PYGZdq{}}\PYG{p}{)}

\PYG{k}{if} \PYG{n}{\PYGZus{}\PYGZus{}name\PYGZus{}\PYGZus{}} \PYG{o}{==} \PYG{l+s+s2}{\PYGZdq{}}\PYG{l+s+s2}{\PYGZus{}\PYGZus{}main\PYGZus{}\PYGZus{}}\PYG{l+s+s2}{\PYGZdq{}}\PYG{p}{:}
    \PYG{n}{main}\PYG{p}{(}\PYG{p}{)}
\end{Verbatim}


\subsection{\textless{}\textless{}imports\textgreater{}\textgreater{}}
\label{antiweb:imports}
\begin{Verbatim}[commandchars=\\\{\},numbers=left,firstnumber=1,stepnumber=1]
\PYG{k+kn}{from} \PYG{n+nn}{optparse} \PYG{k+kn}{import} \PYG{n}{OptionParser}
\PYG{k+kn}{import} \PYG{n+nn}{pygments.lexers} \PYG{k+kn}{as} \PYG{n+nn}{pm}
\PYG{k+kn}{from} \PYG{n+nn}{pygments.token} \PYG{k+kn}{import} \PYG{n}{Token}
\PYG{k+kn}{import} \PYG{n+nn}{bisect}
\PYG{k+kn}{import} \PYG{n+nn}{re}
\PYG{k+kn}{import} \PYG{n+nn}{logging}
\PYG{k+kn}{import} \PYG{n+nn}{sys}
\PYG{k+kn}{import} \PYG{n+nn}{os.path}
\PYG{k+kn}{import} \PYG{n+nn}{operator}
\PYG{k+kn}{import} \PYG{n+nn}{os}
\PYG{k+kn}{import} \PYG{n+nn}{collections}
\PYG{k+kn}{from} \PYG{n+nn}{sys} \PYG{k+kn}{import} \PYG{n}{platform} \PYG{k}{as} \PYG{n}{\PYGZus{}platform}
\end{Verbatim}


\subsection{\textless{}\textless{}management\textgreater{}\textgreater{}}
\label{antiweb:management}
\begin{Verbatim}[commandchars=\\\{\},numbers=left,firstnumber=1,stepnumber=1]
\PYG{n}{\PYGZus{}\PYGZus{}version\PYGZus{}\PYGZus{}} \PYG{o}{=} \PYG{l+s+s2}{\PYGZdq{}}\PYG{l+s+s2}{0.3.1}\PYG{l+s+s2}{\PYGZdq{}}

\PYG{n}{logger} \PYG{o}{=} \PYG{n}{logging}\PYG{o}{.}\PYG{n}{getLogger}\PYG{p}{(}\PYG{l+s+s1}{\PYGZsq{}}\PYG{l+s+s1}{antiweb}\PYG{l+s+s1}{\PYGZsq{}}\PYG{p}{)}

\PYG{k}{class} \PYG{n+nc}{WebError}\PYG{p}{(}\PYG{n+ne}{Exception}\PYG{p}{)}\PYG{p}{:}
    \PYG{k}{def} \PYG{n+nf}{\PYGZus{}\PYGZus{}init\PYGZus{}\PYGZus{}}\PYG{p}{(}\PYG{n+nb+bp}{self}\PYG{p}{,} \PYG{n}{error\PYGZus{}list}\PYG{p}{)}\PYG{p}{:}
        \PYG{n+nb+bp}{self}\PYG{o}{.}\PYG{n}{error\PYGZus{}list} \PYG{o}{=} \PYG{n}{error\PYGZus{}list}
\end{Verbatim}


\subsection{\textless{}\textless{}directives\textgreater{}\textgreater{}}
\label{antiweb:id3}
\begin{Verbatim}[commandchars=\\\{\},numbers=left,firstnumber=1,stepnumber=1]
\PYG{o}{\PYGZlt{}\PYGZlt{}}\PYG{n}{Directive}\PYG{o}{\PYGZgt{}\PYGZgt{}}
\PYG{o}{\PYGZlt{}\PYGZlt{}}\PYG{n}{NameDirective}\PYG{o}{\PYGZgt{}\PYGZgt{}}
\PYG{o}{\PYGZlt{}\PYGZlt{}}\PYG{n}{Start}\PYG{o}{\PYGZgt{}\PYGZgt{}}
\PYG{o}{\PYGZlt{}\PYGZlt{}}\PYG{n}{RStart}\PYG{o}{\PYGZgt{}\PYGZgt{}}
\PYG{o}{\PYGZlt{}\PYGZlt{}}\PYG{n}{CStart}\PYG{o}{\PYGZgt{}\PYGZgt{}}
\PYG{o}{\PYGZlt{}\PYGZlt{}}\PYG{n}{End}\PYG{o}{\PYGZgt{}\PYGZgt{}}
\PYG{o}{\PYGZlt{}\PYGZlt{}}\PYG{n}{Fi}\PYG{o}{\PYGZgt{}\PYGZgt{}}
\PYG{o}{\PYGZlt{}\PYGZlt{}}\PYG{n}{If}\PYG{o}{\PYGZgt{}\PYGZgt{}}
\PYG{o}{\PYGZlt{}\PYGZlt{}}\PYG{n}{Define}\PYG{o}{\PYGZgt{}\PYGZgt{}}
\PYG{o}{\PYGZlt{}\PYGZlt{}}\PYG{n}{Enifed}\PYG{o}{\PYGZgt{}\PYGZgt{}}
\PYG{o}{\PYGZlt{}\PYGZlt{}}\PYG{n}{Subst}\PYG{o}{\PYGZgt{}\PYGZgt{}}
\PYG{o}{\PYGZlt{}\PYGZlt{}}\PYG{n}{Include}\PYG{o}{\PYGZgt{}\PYGZgt{}}
\PYG{o}{\PYGZlt{}\PYGZlt{}}\PYG{n}{RInclude}\PYG{o}{\PYGZgt{}\PYGZgt{}}
\PYG{o}{\PYGZlt{}\PYGZlt{}}\PYG{n}{Edoc}\PYG{o}{\PYGZgt{}\PYGZgt{}}
\PYG{o}{\PYGZlt{}\PYGZlt{}}\PYG{n}{Code}\PYG{o}{\PYGZgt{}\PYGZgt{}}
\PYG{o}{\PYGZlt{}\PYGZlt{}}\PYG{n}{Ignore}\PYG{o}{\PYGZgt{}\PYGZgt{}}
\PYG{o}{\PYGZlt{}\PYGZlt{}}\PYG{n}{Indent}\PYG{o}{\PYGZgt{}\PYGZgt{}}

\PYG{n}{directives} \PYG{o}{=} \PYG{p}{\PYGZob{}}
    \PYG{l+s+s2}{\PYGZdq{}}\PYG{l+s+s2}{start}\PYG{l+s+s2}{\PYGZdq{}} \PYG{p}{:} \PYG{n}{Start}\PYG{p}{,}
    \PYG{l+s+s2}{\PYGZdq{}}\PYG{l+s+s2}{rstart}\PYG{l+s+s2}{\PYGZdq{}} \PYG{p}{:} \PYG{n}{RStart}\PYG{p}{,}
    \PYG{l+s+s2}{\PYGZdq{}}\PYG{l+s+s2}{cstart}\PYG{l+s+s2}{\PYGZdq{}} \PYG{p}{:} \PYG{n}{CStart}\PYG{p}{,}
    \PYG{l+s+s2}{\PYGZdq{}}\PYG{l+s+s2}{edoc}\PYG{l+s+s2}{\PYGZdq{}} \PYG{p}{:} \PYG{n}{Edoc}\PYG{p}{,}
    \PYG{l+s+s2}{\PYGZdq{}}\PYG{l+s+s2}{end}\PYG{l+s+s2}{\PYGZdq{}} \PYG{p}{:} \PYG{n}{End}\PYG{p}{,}
    \PYG{l+s+s2}{\PYGZdq{}}\PYG{l+s+s2}{include}\PYG{l+s+s2}{\PYGZdq{}} \PYG{p}{:} \PYG{n}{Include}\PYG{p}{,}
    \PYG{l+s+s2}{\PYGZdq{}}\PYG{l+s+s2}{code}\PYG{l+s+s2}{\PYGZdq{}} \PYG{p}{:} \PYG{n}{Code}\PYG{p}{,}
    \PYG{l+s+s2}{\PYGZdq{}}\PYG{l+s+s2}{ignore}\PYG{l+s+s2}{\PYGZdq{}} \PYG{p}{:} \PYG{n}{Ignore}\PYG{p}{,}
    \PYG{l+s+s2}{\PYGZdq{}}\PYG{l+s+s2}{indent}\PYG{l+s+s2}{\PYGZdq{}} \PYG{p}{:} \PYG{n}{Indent}\PYG{p}{,}
    \PYG{l+s+s2}{\PYGZdq{}}\PYG{l+s+s2}{if}\PYG{l+s+s2}{\PYGZdq{}} \PYG{p}{:} \PYG{n}{If}\PYG{p}{,}
    \PYG{l+s+s2}{\PYGZdq{}}\PYG{l+s+s2}{fi}\PYG{l+s+s2}{\PYGZdq{}} \PYG{p}{:} \PYG{n}{Fi}\PYG{p}{,}
    \PYG{l+s+s2}{\PYGZdq{}}\PYG{l+s+s2}{define}\PYG{l+s+s2}{\PYGZdq{}} \PYG{p}{:} \PYG{n}{Define}\PYG{p}{,}
    \PYG{l+s+s2}{\PYGZdq{}}\PYG{l+s+s2}{enifed}\PYG{l+s+s2}{\PYGZdq{}} \PYG{p}{:} \PYG{n}{Enifed}\PYG{p}{,}
    \PYG{l+s+s2}{\PYGZdq{}}\PYG{l+s+s2}{subst}\PYG{l+s+s2}{\PYGZdq{}} \PYG{p}{:} \PYG{n}{Subst}\PYG{p}{,}
    \PYG{l+s+s2}{\PYGZdq{}}\PYG{l+s+s2}{rinclude}\PYG{l+s+s2}{\PYGZdq{}} \PYG{p}{:} \PYG{n}{RInclude}\PYG{p}{,}
    \PYG{p}{\PYGZcb{}}
\end{Verbatim}


\subsection{\textless{}\textless{}readers\textgreater{}\textgreater{}}
\label{antiweb:id5}\label{antiweb:id4}
\begin{Verbatim}[commandchars=\\\{\}]
\PYG{o}{\PYGZlt{}\PYGZlt{}}\PYG{n}{Reader}\PYG{o}{\PYGZgt{}\PYGZgt{}}
\PYG{o}{\PYGZlt{}\PYGZlt{}}\PYG{n}{CReader}\PYG{o}{\PYGZgt{}\PYGZgt{}}
\PYG{o}{\PYGZlt{}\PYGZlt{}}\PYG{n}{PythonReader}\PYG{o}{\PYGZgt{}\PYGZgt{}}
\end{Verbatim}


\subsection{\textless{}\textless{}document\textgreater{}\textgreater{}}
\label{antiweb:id6}
\begin{Verbatim}[commandchars=\\\{\}]
\PYG{o}{\PYGZlt{}\PYGZlt{}}\PYG{n}{Document}\PYG{o}{\PYGZgt{}\PYGZgt{}}
\PYG{o}{\PYGZlt{}\PYGZlt{}}\PYG{n}{generate}\PYG{o}{\PYGZgt{}\PYGZgt{}}
\end{Verbatim}


\section{Multi-File Processing and Sphinx Support}
\label{antiweb:multi-file-processing-and-sphinx-support}
There are two new flags in antiweb:
\begin{itemize}
\item {} \begin{description}
\item[{The `'-r'' flag:}] \leavevmode\begin{itemize}
\item {} 
antiweb will search for all compatible files to process them

\end{itemize}

\end{description}

\item {} \begin{description}
\item[{The `'-i'' flag:}] \leavevmode\begin{itemize}
\item {} 
Sphinx' index.rst will be edited to contain all processed files (empty files will be ignored)

\end{itemize}

\end{description}

\end{itemize}

\begin{Verbatim}[commandchars=\\\{\},numbers=left,firstnumber=1,stepnumber=1]
    \PYG{n}{parser}\PYG{o}{.}\PYG{n}{add\PYGZus{}option}\PYG{p}{(}\PYG{l+s+s2}{\PYGZdq{}}\PYG{l+s+s2}{\PYGZhy{}r}\PYG{l+s+s2}{\PYGZdq{}}\PYG{p}{,} \PYG{l+s+s2}{\PYGZdq{}}\PYG{l+s+s2}{\PYGZhy{}\PYGZhy{}recursive}\PYG{l+s+s2}{\PYGZdq{}}\PYG{p}{,} \PYG{n}{dest}\PYG{o}{=}\PYG{l+s+s2}{\PYGZdq{}}\PYG{l+s+s2}{recursive}\PYG{l+s+s2}{\PYGZdq{}}\PYG{p}{,}
                      \PYG{n}{action}\PYG{o}{=}\PYG{l+s+s2}{\PYGZdq{}}\PYG{l+s+s2}{store\PYGZus{}true}\PYG{l+s+s2}{\PYGZdq{}}\PYG{p}{,} \PYG{n}{help}\PYG{o}{=}\PYG{l+s+s2}{\PYGZdq{}}\PYG{l+s+s2}{Process every file in given directory}\PYG{l+s+s2}{\PYGZdq{}}\PYG{p}{)}

    \PYG{n}{parser}\PYG{o}{.}\PYG{n}{add\PYGZus{}option}\PYG{p}{(}\PYG{l+s+s2}{\PYGZdq{}}\PYG{l+s+s2}{\PYGZhy{}i}\PYG{l+s+s2}{\PYGZdq{}}\PYG{p}{,} \PYG{l+s+s2}{\PYGZdq{}}\PYG{l+s+s2}{\PYGZhy{}\PYGZhy{}index}\PYG{l+s+s2}{\PYGZdq{}}\PYG{p}{,} \PYG{n}{dest}\PYG{o}{=}\PYG{l+s+s2}{\PYGZdq{}}\PYG{l+s+s2}{index}\PYG{l+s+s2}{\PYGZdq{}}\PYG{p}{,}
                      \PYG{n}{action}\PYG{o}{=}\PYG{l+s+s2}{\PYGZdq{}}\PYG{l+s+s2}{store\PYGZus{}true}\PYG{l+s+s2}{\PYGZdq{}}\PYG{p}{,} \PYG{n}{help}\PYG{o}{=}\PYG{l+s+s2}{\PYGZdq{}}\PYG{l+s+s2}{Automatically write file(s) to Sphinx}\PYG{l+s+s2}{\PYGZsq{}}\PYG{l+s+s2}{ index.rst}\PYG{l+s+s2}{\PYGZdq{}}\PYG{p}{)}


\PYG{n}{options}\PYG{p}{,} \PYG{n}{args} \PYG{o}{=} \PYG{n}{parser}\PYG{o}{.}\PYG{n}{parse\PYGZus{}args}\PYG{p}{(}\PYG{p}{)}

\PYG{n}{logger}\PYG{o}{.}\PYG{n}{addHandler}\PYG{p}{(}\PYG{n}{logging}\PYG{o}{.}\PYG{n}{StreamHandler}\PYG{p}{(}\PYG{p}{)}\PYG{p}{)}
\PYG{n}{logger}\PYG{o}{.}\PYG{n}{setLevel}\PYG{p}{(}\PYG{n}{logging}\PYG{o}{.}\PYG{n}{INFO}\PYG{p}{)}

\PYG{k}{if} \PYG{n}{options}\PYG{o}{.}\PYG{n}{warnings} \PYG{o+ow}{is} \PYG{n+nb+bp}{None}\PYG{p}{:}
    \PYG{n}{options}\PYG{o}{.}\PYG{n}{warnings} \PYG{o}{=} \PYG{n+nb+bp}{True}

\PYG{k}{if} \PYG{o+ow}{not} \PYG{n}{args}\PYG{p}{:}
    \PYG{n}{parser}\PYG{o}{.}\PYG{n}{print\PYGZus{}help}\PYG{p}{(}\PYG{p}{)}
    \PYG{n}{sys}\PYG{o}{.}\PYG{n}{exit}\PYG{p}{(}\PYG{l+m+mi}{0}\PYG{p}{)}

\PYG{n}{index\PYGZus{}rst} \PYG{o}{=} \PYG{l+s+s2}{\PYGZdq{}}\PYG{l+s+s2}{index.rst}\PYG{l+s+s2}{\PYGZdq{}}
\PYG{n}{replace\PYGZus{}text} \PYG{o}{=} \PYG{l+s+s2}{\PYGZdq{}}\PYG{l+s+s2}{\PYGZdq{}}
\PYG{n}{startblock} \PYG{o}{=} \PYG{l+s+s2}{\PYGZdq{}}\PYG{l+s+s2}{.. @start(generated)}\PYG{l+s+s2}{\PYGZdq{}}
\PYG{n}{endblock} \PYG{o}{=} \PYG{l+s+s2}{\PYGZdq{}}\PYG{l+s+s2}{.. @(generated)}\PYG{l+s+s2}{\PYGZdq{}}
\end{Verbatim}

The program will check if a -r flag was given and if so, save the current directory and change it to the given one

\begin{Verbatim}[commandchars=\\\{\},numbers=left,firstnumber=1,stepnumber=1]
\PYG{k}{if} \PYG{n}{options}\PYG{o}{.}\PYG{n}{recursive}\PYG{p}{:}
    \PYG{n}{previous\PYGZus{}dir} \PYG{o}{=} \PYG{n}{os}\PYG{o}{.}\PYG{n}{getcwd}\PYG{p}{(}\PYG{p}{)}
    \PYG{n}{os}\PYG{o}{.}\PYG{n}{chdir}\PYG{p}{(}\PYG{n}{args}\PYG{p}{[}\PYG{l+m+mi}{0}\PYG{p}{]}\PYG{p}{)}
    \PYG{k}{if} \PYG{n}{options}\PYG{o}{.}\PYG{n}{output}\PYG{p}{:}
        \PYG{n}{os}\PYG{o}{.}\PYG{n}{makedirs}\PYG{p}{(}\PYG{n}{os}\PYG{o}{.}\PYG{n}{path}\PYG{o}{.}\PYG{n}{join}\PYG{p}{(}\PYG{n}{args}\PYG{p}{[}\PYG{l+m+mi}{0}\PYG{p}{]}\PYG{p}{,} \PYG{n}{options}\PYG{o}{.}\PYG{n}{output}\PYG{p}{)}\PYG{p}{,} \PYG{n}{exist\PYGZus{}ok}\PYG{o}{=}\PYG{n+nb+bp}{True}\PYG{p}{)}
    \PYG{k}{if} \PYG{n}{options}\PYG{o}{.}\PYG{n}{index}\PYG{p}{:}
        \PYG{k}{if} \PYG{n}{options}\PYG{o}{.}\PYG{n}{output}\PYG{p}{:}
            \PYG{n}{in\PYGZus{}type} \PYG{o}{=} \PYG{n}{os}\PYG{o}{.}\PYG{n}{path}\PYG{o}{.}\PYG{n}{join}\PYG{p}{(}\PYG{n}{args}\PYG{p}{[}\PYG{l+m+mi}{0}\PYG{p}{]}\PYG{p}{,} \PYG{n}{options}\PYG{o}{.}\PYG{n}{output}\PYG{p}{)}
            \PYG{k}{if} \PYG{o+ow}{not} \PYG{n}{os}\PYG{o}{.}\PYG{n}{path}\PYG{o}{.}\PYG{n}{isfile}\PYG{p}{(}\PYG{n}{os}\PYG{o}{.}\PYG{n}{path}\PYG{o}{.}\PYG{n}{join}\PYG{p}{(}\PYG{n}{in\PYGZus{}type}\PYG{p}{,} \PYG{n}{index\PYGZus{}rst}\PYG{p}{)}\PYG{p}{)}\PYG{p}{:}
                \PYG{n}{write\PYGZus{}static}\PYG{p}{(}\PYG{n}{in\PYGZus{}type}\PYG{p}{,} \PYG{n}{index\PYGZus{}rst}\PYG{p}{,} \PYG{n}{startblock}\PYG{p}{,} \PYG{n}{endblock}\PYG{p}{)}

            \PYG{n}{content}\PYG{p}{,} \PYG{n}{startline} \PYG{o}{=} \PYG{n}{search\PYGZus{}for\PYGZus{}generate}\PYG{p}{(}\PYG{n}{options}\PYG{o}{.}\PYG{n}{output}\PYG{p}{,} \PYG{n}{index\PYGZus{}rst}\PYG{p}{,} \PYG{n}{startblock}\PYG{p}{,} \PYG{n}{endblock}\PYG{p}{)}

        \PYG{k}{else}\PYG{p}{:}
            \PYG{n}{in\PYGZus{}type} \PYG{o}{=} \PYG{n}{args}\PYG{p}{[}\PYG{l+m+mi}{0}\PYG{p}{]}
            \PYG{k}{if} \PYG{o+ow}{not} \PYG{n}{os}\PYG{o}{.}\PYG{n}{path}\PYG{o}{.}\PYG{n}{isfile}\PYG{p}{(}\PYG{n}{os}\PYG{o}{.}\PYG{n}{path}\PYG{o}{.}\PYG{n}{join}\PYG{p}{(}\PYG{n}{in\PYGZus{}type}\PYG{p}{,} \PYG{n}{index\PYGZus{}rst}\PYG{p}{)}\PYG{p}{)}\PYG{p}{:}
                \PYG{n}{write\PYGZus{}static}\PYG{p}{(}\PYG{n}{in\PYGZus{}type}\PYG{p}{,} \PYG{n}{index\PYGZus{}rst}\PYG{p}{,} \PYG{n}{startblock}\PYG{p}{,} \PYG{n}{endblock}\PYG{p}{)}

            \PYG{n}{content}\PYG{p}{,} \PYG{n}{startline} \PYG{o}{=} \PYG{n}{search\PYGZus{}for\PYGZus{}generate}\PYG{p}{(}\PYG{n}{os}\PYG{o}{.}\PYG{n}{getcwd}\PYG{p}{(}\PYG{p}{)}\PYG{p}{,} \PYG{n}{index\PYGZus{}rst}\PYG{p}{,} \PYG{n}{startblock}\PYG{p}{,} \PYG{n}{endblock}\PYG{p}{)}
\end{Verbatim}

The program lists all files in the directory and sub-directories to prepare them for the process

\begin{Verbatim}[commandchars=\\\{\}]
\PYG{k}{for} \PYG{n}{root}\PYG{p}{,} \PYG{n}{dirs}\PYG{p}{,} \PYG{n}{files} \PYG{o+ow}{in} \PYG{n}{os}\PYG{o}{.}\PYG{n}{walk}\PYG{p}{(}\PYG{n}{args}\PYG{p}{[}\PYG{l+m+mi}{0}\PYG{p}{]}\PYG{p}{,} \PYG{n}{topdown}\PYG{o}{=}\PYG{n+nb+bp}{False}\PYG{p}{)}\PYG{p}{:}
    \PYG{k}{for} \PYG{n}{filename} \PYG{o+ow}{in} \PYG{n}{files}\PYG{p}{:}
        \PYG{n}{fname} \PYG{o}{=} \PYG{n}{os}\PYG{o}{.}\PYG{n}{path}\PYG{o}{.}\PYG{n}{join}\PYG{p}{(}\PYG{n}{root}\PYG{p}{,} \PYG{n}{filename}\PYG{p}{)}
\end{Verbatim}

If the found file is no folder nor a rst file, it will be put into the process

\begin{Verbatim}[commandchars=\\\{\},numbers=left,firstnumber=1,stepnumber=1]
\PYG{n}{ext\PYGZus{}tuple} \PYG{o}{=} \PYG{p}{(}\PYG{l+s+s2}{\PYGZdq{}}\PYG{l+s+s2}{.cs}\PYG{l+s+s2}{\PYGZdq{}}\PYG{p}{,}\PYG{l+s+s2}{\PYGZdq{}}\PYG{l+s+s2}{.cpp}\PYG{l+s+s2}{\PYGZdq{}}\PYG{p}{,}\PYG{l+s+s2}{\PYGZdq{}}\PYG{l+s+s2}{.py}\PYG{l+s+s2}{\PYGZdq{}}\PYG{p}{,}\PYG{l+s+s2}{\PYGZdq{}}\PYG{l+s+s2}{.cc}\PYG{l+s+s2}{\PYGZdq{}}\PYG{p}{)}
\PYG{k}{if} \PYG{n}{os}\PYG{o}{.}\PYG{n}{path}\PYG{o}{.}\PYG{n}{isfile}\PYG{p}{(}\PYG{n}{fname}\PYG{p}{)} \PYG{o+ow}{and} \PYG{n}{fname}\PYG{o}{.}\PYG{n}{endswith}\PYG{p}{(}\PYG{n}{ext\PYGZus{}tuple}\PYG{p}{)}\PYG{p}{:}
    \PYG{k}{if} \PYG{n}{options}\PYG{o}{.}\PYG{n}{index}\PYG{p}{:}
        \PYG{n}{write}\PYG{p}{(}\PYG{n}{os}\PYG{o}{.}\PYG{n}{getcwd}\PYG{p}{(}\PYG{p}{)}\PYG{p}{,} \PYG{n}{fname}\PYG{p}{,} \PYG{n}{options}\PYG{o}{.}\PYG{n}{output}\PYG{p}{,} \PYG{n}{options}\PYG{o}{.}\PYG{n}{token}\PYG{p}{,} \PYG{n}{options}\PYG{o}{.}\PYG{n}{warnings}\PYG{p}{,} \PYG{n}{options}\PYG{o}{.}\PYG{n}{index}\PYG{p}{,} \PYG{n}{index\PYGZus{}rst}\PYG{p}{,} \PYG{n}{options}\PYG{o}{.}\PYG{n}{recursive}\PYG{p}{,} \PYG{n}{content}\PYG{p}{,} \PYG{n}{startblock}\PYG{p}{,} \PYG{n}{endblock}\PYG{p}{,} \PYG{n}{startline}\PYG{p}{)}
    \PYG{k}{else}\PYG{p}{:}
        \PYG{n}{write}\PYG{p}{(}\PYG{n}{os}\PYG{o}{.}\PYG{n}{getcwd}\PYG{p}{(}\PYG{p}{)}\PYG{p}{,} \PYG{n}{fname}\PYG{p}{,} \PYG{n}{options}\PYG{o}{.}\PYG{n}{output}\PYG{p}{,} \PYG{n}{options}\PYG{o}{.}\PYG{n}{token}\PYG{p}{,} \PYG{n}{options}\PYG{o}{.}\PYG{n}{warnings}\PYG{p}{,} \PYG{n}{options}\PYG{o}{.}\PYG{n}{index}\PYG{p}{,} \PYG{n}{index\PYGZus{}rst}\PYG{p}{,} \PYG{n}{options}\PYG{o}{.}\PYG{n}{recursive}\PYG{p}{,} \PYG{n+nb+bp}{None}\PYG{p}{,} \PYG{n}{startblock}\PYG{p}{,} \PYG{n}{endblock}\PYG{p}{,} \PYG{n+nb+bp}{None}\PYG{p}{)}
\end{Verbatim}

This else will take place when the -r flag is not given.

\begin{Verbatim}[commandchars=\\\{\},numbers=left,firstnumber=1,stepnumber=1]
\PYG{k}{else}\PYG{p}{:}
    \PYG{n}{os}\PYG{o}{.}\PYG{n}{chdir}\PYG{p}{(}\PYG{n}{os}\PYG{o}{.}\PYG{n}{path}\PYG{o}{.}\PYG{n}{split}\PYG{p}{(}\PYG{n}{args}\PYG{p}{[}\PYG{l+m+mi}{0}\PYG{p}{]}\PYG{p}{)}\PYG{p}{[}\PYG{l+m+mi}{0}\PYG{p}{]}\PYG{p}{)}
    \PYG{k}{if} \PYG{n}{options}\PYG{o}{.}\PYG{n}{index}\PYG{p}{:}
        \PYG{k}{if} \PYG{o+ow}{not} \PYG{n}{os}\PYG{o}{.}\PYG{n}{path}\PYG{o}{.}\PYG{n}{isfile}\PYG{p}{(}\PYG{n}{os}\PYG{o}{.}\PYG{n}{path}\PYG{o}{.}\PYG{n}{join}\PYG{p}{(}\PYG{n}{os}\PYG{o}{.}\PYG{n}{getcwd}\PYG{p}{(}\PYG{p}{)}\PYG{p}{,} \PYG{n}{index\PYGZus{}rst}\PYG{p}{)}\PYG{p}{)}\PYG{p}{:}
            \PYG{n}{write\PYGZus{}static}\PYG{p}{(}\PYG{n}{os}\PYG{o}{.}\PYG{n}{getcwd}\PYG{p}{(}\PYG{p}{)}\PYG{p}{,} \PYG{n}{index\PYGZus{}rst}\PYG{p}{,} \PYG{n}{startblock}\PYG{p}{,} \PYG{n}{endblock}\PYG{p}{)}
        \PYG{n}{content}\PYG{p}{,} \PYG{n}{startline} \PYG{o}{=} \PYG{n}{search\PYGZus{}for\PYGZus{}generate}\PYG{p}{(}\PYG{n+nb+bp}{None}\PYG{p}{,} \PYG{n}{index\PYGZus{}rst}\PYG{p}{,} \PYG{n}{startblock}\PYG{p}{,} \PYG{n}{endblock}\PYG{p}{)}
        \PYG{n}{write}\PYG{p}{(}\PYG{n}{os}\PYG{o}{.}\PYG{n}{getcwd}\PYG{p}{(}\PYG{p}{)}\PYG{p}{,} \PYG{n}{args}\PYG{p}{[}\PYG{l+m+mi}{0}\PYG{p}{]}\PYG{p}{,} \PYG{n}{options}\PYG{o}{.}\PYG{n}{output}\PYG{p}{,} \PYG{n}{options}\PYG{o}{.}\PYG{n}{token}\PYG{p}{,} \PYG{n}{options}\PYG{o}{.}\PYG{n}{warnings}\PYG{p}{,} \PYG{n}{options}\PYG{o}{.}\PYG{n}{index}\PYG{p}{,} \PYG{n}{index\PYGZus{}rst}\PYG{p}{,} \PYG{n}{options}\PYG{o}{.}\PYG{n}{recursive}\PYG{p}{,} \PYG{n}{content}\PYG{p}{,} \PYG{n}{startblock}\PYG{p}{,} \PYG{n}{endblock}\PYG{p}{,} \PYG{n}{startline}\PYG{p}{)}
    \PYG{k}{else}\PYG{p}{:}
        \PYG{n}{write}\PYG{p}{(}\PYG{n}{os}\PYG{o}{.}\PYG{n}{getcwd}\PYG{p}{(}\PYG{p}{)}\PYG{p}{,} \PYG{n}{args}\PYG{p}{[}\PYG{l+m+mi}{0}\PYG{p}{]}\PYG{p}{,} \PYG{n}{options}\PYG{o}{.}\PYG{n}{output}\PYG{p}{,} \PYG{n}{options}\PYG{o}{.}\PYG{n}{token}\PYG{p}{,} \PYG{n}{options}\PYG{o}{.}\PYG{n}{warnings}\PYG{p}{,} \PYG{n}{options}\PYG{o}{.}\PYG{n}{index}\PYG{p}{,} \PYG{n}{index\PYGZus{}rst}\PYG{p}{,} \PYG{n}{options}\PYG{o}{.}\PYG{n}{recursive}\PYG{p}{,} \PYG{n+nb+bp}{None}\PYG{p}{,} \PYG{n}{startblock}\PYG{p}{,} \PYG{n}{endblock}\PYG{p}{,} \PYG{n+nb+bp}{None}\PYG{p}{)}
\end{Verbatim}


\subsection{Writing the index.rst file}
\label{antiweb:writing-the-index-rst-file}
From the given file a .rst file will be created if it contains an antiweb \sphinxcode{start() directive}

\begin{Verbatim}[commandchars=\\\{\}]
\PYG{k}{def} \PYG{n+nf}{write}\PYG{p}{(}\PYG{n}{path}\PYG{p}{,} \PYG{n}{fname}\PYG{p}{,} \PYG{n}{output}\PYG{p}{,} \PYG{n}{token}\PYG{p}{,} \PYG{n}{warnings}\PYG{p}{,} \PYG{n}{index}\PYG{p}{,} \PYG{n}{index\PYGZus{}rst}\PYG{p}{,} \PYG{n}{recursive}\PYG{p}{,} \PYG{n}{content}\PYG{p}{,} \PYG{n}{startblock}\PYG{p}{,} \PYG{n}{endblock}\PYG{p}{,} \PYG{n}{startline}\PYG{p}{)}\PYG{p}{:}
\end{Verbatim}

When there is no output given there are two possibilities: recursive or not recursive. The file path gets split up and put together so it can be processed by `'process\_file'`

\begin{Verbatim}[commandchars=\\\{\},numbers=left,firstnumber=1,stepnumber=1]
\PYG{k}{if} \PYG{o+ow}{not} \PYG{n}{output}\PYG{p}{:}
    \PYG{k}{if} \PYG{n}{recursive}\PYG{p}{:}
        \PYG{n}{out\PYGZus{}file\PYGZus{}name} \PYG{o}{=} \PYG{n}{os}\PYG{o}{.}\PYG{n}{path}\PYG{o}{.}\PYG{n}{splitext}\PYG{p}{(}\PYG{n}{fname}\PYG{p}{)}\PYG{p}{[}\PYG{l+m+mi}{0}\PYG{p}{]} \PYG{o}{+} \PYG{l+s+s2}{\PYGZdq{}}\PYG{l+s+s2}{.rst}\PYG{l+s+s2}{\PYGZdq{}}
        \PYG{n}{out\PYGZus{}file} \PYG{o}{=} \PYG{n}{out\PYGZus{}file\PYGZus{}name}
        \PYG{n}{out\PYGZus{}file\PYGZus{}name} \PYG{o}{=} \PYG{n}{os}\PYG{o}{.}\PYG{n}{path}\PYG{o}{.}\PYG{n}{relpath}\PYG{p}{(}\PYG{n}{out\PYGZus{}file\PYGZus{}name}\PYG{p}{,} \PYG{n}{path}\PYG{p}{)}
    \PYG{k}{else}\PYG{p}{:}
        \PYG{n}{out\PYGZus{}file} \PYG{o}{=} \PYG{n}{os}\PYG{o}{.}\PYG{n}{path}\PYG{o}{.}\PYG{n}{splitext}\PYG{p}{(}\PYG{n}{fname}\PYG{p}{)}\PYG{p}{[}\PYG{l+m+mi}{0}\PYG{p}{]} \PYG{o}{+} \PYG{l+s+s2}{\PYGZdq{}}\PYG{l+s+s2}{.rst}\PYG{l+s+s2}{\PYGZdq{}}
        \PYG{n}{out\PYGZus{}file\PYGZus{}name} \PYG{o}{=} \PYG{n}{os}\PYG{o}{.}\PYG{n}{path}\PYG{o}{.}\PYG{n}{split}\PYG{p}{(}\PYG{n}{out\PYGZus{}file}\PYG{p}{)}\PYG{p}{[}\PYG{l+m+mi}{1}\PYG{p}{]}
\end{Verbatim}

There is an output given, so it can also be either recursive or not recursive. With the additional output parameter the file path gets split up and put together so it can be processed by `'process\_file'`. There is also a differentiation between Linux, Windows and OS X (because of the different paths in each operating system)

\begin{Verbatim}[commandchars=\\\{\},numbers=left,firstnumber=1,stepnumber=1]
\PYG{k}{else}\PYG{p}{:}
    \PYG{k}{if} \PYG{n}{recursive}\PYG{p}{:}
        \PYG{n}{rel\PYGZus{}path} \PYG{o}{=} \PYG{n}{os}\PYG{o}{.}\PYG{n}{path}\PYG{o}{.}\PYG{n}{relpath}\PYG{p}{(}\PYG{n}{fname}\PYG{p}{,} \PYG{n}{path}\PYG{p}{)}
        \PYG{n}{out\PYGZus{}file\PYGZus{}name} \PYG{o}{=} \PYG{n}{os}\PYG{o}{.}\PYG{n}{path}\PYG{o}{.}\PYG{n}{splitext}\PYG{p}{(}\PYG{n}{rel\PYGZus{}path}\PYG{p}{)}\PYG{p}{[}\PYG{l+m+mi}{0}\PYG{p}{]} \PYG{o}{+} \PYG{l+s+s2}{\PYGZdq{}}\PYG{l+s+s2}{.rst}\PYG{l+s+s2}{\PYGZdq{}}
        \PYG{k}{if} \PYG{n}{\PYGZus{}platform} \PYG{o}{==} \PYG{l+s+s2}{\PYGZdq{}}\PYG{l+s+s2}{linux}\PYG{l+s+s2}{\PYGZdq{}} \PYG{o+ow}{or} \PYG{n}{\PYGZus{}platform} \PYG{o}{==} \PYG{l+s+s2}{\PYGZdq{}}\PYG{l+s+s2}{linux2}\PYG{l+s+s2}{\PYGZdq{}}\PYG{p}{:}
            \PYG{n}{out\PYGZus{}file\PYGZus{}name} \PYG{o}{=} \PYG{n}{out\PYGZus{}file\PYGZus{}name}\PYG{o}{.}\PYG{n}{replace}\PYG{p}{(}\PYG{l+s+s2}{\PYGZdq{}}\PYG{l+s+s2}{/}\PYG{l+s+s2}{\PYGZdq{}}\PYG{p}{,}\PYG{l+s+s2}{\PYGZdq{}}\PYG{l+s+s2}{\PYGZus{}}\PYG{l+s+s2}{\PYGZdq{}}\PYG{p}{)}
        \PYG{k}{if} \PYG{n}{\PYGZus{}platform} \PYG{o}{==} \PYG{l+s+s2}{\PYGZdq{}}\PYG{l+s+s2}{win32}\PYG{l+s+s2}{\PYGZdq{}}\PYG{p}{:}
            \PYG{n}{out\PYGZus{}file\PYGZus{}name} \PYG{o}{=} \PYG{n}{out\PYGZus{}file\PYGZus{}name}\PYG{o}{.}\PYG{n}{replace}\PYG{p}{(}\PYG{l+s+s2}{\PYGZdq{}}\PYG{l+s+se}{\PYGZbs{}\PYGZbs{}}\PYG{l+s+s2}{\PYGZdq{}}\PYG{p}{,}\PYG{l+s+s2}{\PYGZdq{}}\PYG{l+s+s2}{\PYGZus{}}\PYG{l+s+s2}{\PYGZdq{}}\PYG{p}{)}
        \PYG{n}{out\PYGZus{}file} \PYG{o}{=} \PYG{n}{os}\PYG{o}{.}\PYG{n}{path}\PYG{o}{.}\PYG{n}{join}\PYG{p}{(}\PYG{n}{path}\PYG{p}{,} \PYG{n}{output}\PYG{p}{,} \PYG{n}{out\PYGZus{}file\PYGZus{}name}\PYG{p}{)}
    \PYG{k}{else}\PYG{p}{:}
        \PYG{n}{out\PYGZus{}file\PYGZus{}name} \PYG{o}{=} \PYG{n}{output} \PYG{o}{+} \PYG{l+s+s2}{\PYGZdq{}}\PYG{l+s+s2}{.rst}\PYG{l+s+s2}{\PYGZdq{}}
        \PYG{n}{out\PYGZus{}file} \PYG{o}{=} \PYG{n}{os}\PYG{o}{.}\PYG{n}{path}\PYG{o}{.}\PYG{n}{join}\PYG{p}{(}\PYG{n}{path}\PYG{p}{,} \PYG{n}{out\PYGZus{}file\PYGZus{}name}\PYG{p}{)}
\end{Verbatim}

The prepared file path gets pushed to `'process file'`. If the process is successful, `'could\_process'' is set to `'True'`.

\begin{Verbatim}[commandchars=\\\{\}]
\PYG{n}{could\PYGZus{}process} \PYG{o}{=} \PYG{n}{process\PYGZus{}file}\PYG{p}{(}\PYG{n}{fname}\PYG{p}{,} \PYG{n}{out\PYGZus{}file}\PYG{p}{,} \PYG{n}{token}\PYG{p}{,} \PYG{n}{warnings}\PYG{p}{)}
\end{Verbatim}

If the user added the -i flag, the file gets added to Sphinx' index.rst file. Between the \sphinxcode{start(generated)} and \sphinxcode{(generated)} directives is the space for automatic added files, you can manually add files below the @(generated) directive.

\begin{Verbatim}[commandchars=\\\{\},numbers=left,firstnumber=1,stepnumber=1]
\PYG{k}{if} \PYG{n}{index}\PYG{p}{:}
    \PYG{k}{if} \PYG{n}{could\PYGZus{}process}\PYG{p}{:}

        \PYG{k}{if} \PYG{n}{output} \PYG{o+ow}{and} \PYG{n}{recursive}\PYG{p}{:}
            \PYG{n}{replace\PYGZus{}in\PYGZus{}generated}\PYG{p}{(}\PYG{n}{startblock}\PYG{p}{,} \PYG{n}{endblock}\PYG{p}{,} \PYG{n}{out\PYGZus{}file\PYGZus{}name}\PYG{p}{,} \PYG{n}{path}\PYG{p}{,} \PYG{n}{output}\PYG{p}{,} \PYG{n}{index\PYGZus{}rst}\PYG{p}{,} \PYG{n}{content}\PYG{p}{,} \PYG{n}{startline}\PYG{p}{)}
        \PYG{k}{else}\PYG{p}{:}
            \PYG{n}{replace\PYGZus{}in\PYGZus{}generated}\PYG{p}{(}\PYG{n}{startblock}\PYG{p}{,} \PYG{n}{endblock}\PYG{p}{,} \PYG{n}{out\PYGZus{}file\PYGZus{}name}\PYG{p}{,} \PYG{n}{path}\PYG{p}{,} \PYG{n+nb+bp}{None}\PYG{p}{,} \PYG{n}{index\PYGZus{}rst}\PYG{p}{,} \PYG{n}{content}\PYG{p}{,} \PYG{n}{startline}\PYG{p}{)}
\end{Verbatim}


\subsection{process\_file}
\label{antiweb:process-file}
If no output name was declared, the input name will be given

\begin{Verbatim}[commandchars=\\\{\}]
\PYG{k}{def} \PYG{n+nf}{process\PYGZus{}file}\PYG{p}{(}\PYG{n}{in\PYGZus{}file}\PYG{p}{,} \PYG{n}{out\PYGZus{}file}\PYG{p}{,} \PYG{n}{token}\PYG{p}{,} \PYG{n}{warnings}\PYG{p}{)}\PYG{p}{:}
\end{Verbatim}

The output text will be written in the output file. If there is an output text, the function returns could\_write as True

\begin{Verbatim}[commandchars=\\\{\},numbers=left,firstnumber=1,stepnumber=1]
\PYG{n}{could\PYGZus{}write} \PYG{o}{=} \PYG{n+nb+bp}{False}
\PYG{k}{try}\PYG{p}{:}
    \PYG{n}{text\PYGZus{}output} \PYG{o}{=} \PYG{n}{generate}\PYG{p}{(}\PYG{n}{in\PYGZus{}file}\PYG{p}{,} \PYG{n}{token}\PYG{p}{,} \PYG{n}{warnings}\PYG{p}{)}

    \PYG{k}{if} \PYG{n}{text\PYGZus{}output}\PYG{p}{:}
        \PYG{k}{with} \PYG{n+nb}{open}\PYG{p}{(}\PYG{n}{out\PYGZus{}file}\PYG{p}{,} \PYG{l+s+s2}{\PYGZdq{}}\PYG{l+s+s2}{w}\PYG{l+s+s2}{\PYGZdq{}}\PYG{p}{)} \PYG{k}{as} \PYG{n}{f}\PYG{p}{:}
            \PYG{n}{f}\PYG{o}{.}\PYG{n}{write}\PYG{p}{(}\PYG{n}{text\PYGZus{}output}\PYG{p}{)}
        \PYG{n}{could\PYGZus{}write} \PYG{o}{=} \PYG{n+nb+bp}{True}
\PYG{k}{except} \PYG{n}{WebError} \PYG{k}{as} \PYG{n}{e}\PYG{p}{:}
    \PYG{n}{logger}\PYG{o}{.}\PYG{n}{error}\PYG{p}{(}\PYG{l+s+s2}{\PYGZdq{}}\PYG{l+s+se}{\PYGZbs{}n}\PYG{l+s+s2}{Errors:}\PYG{l+s+s2}{\PYGZdq{}}\PYG{p}{)}
    \PYG{k}{for} \PYG{n}{l}\PYG{p}{,} \PYG{n}{d} \PYG{o+ow}{in} \PYG{n}{e}\PYG{o}{.}\PYG{n}{error\PYGZus{}list}\PYG{p}{:}
        \PYG{n}{logger}\PYG{o}{.}\PYG{n}{error}\PYG{p}{(}\PYG{l+s+s2}{\PYGZdq{}}\PYG{l+s+s2}{  in line }\PYG{l+s+si}{\PYGZpc{}i}\PYG{l+s+s2}{(}\PYG{l+s+si}{\PYGZpc{}s}\PYG{l+s+s2}{): }\PYG{l+s+si}{\PYGZpc{}s}\PYG{l+s+s2}{\PYGZdq{}}\PYG{p}{,} \PYG{n}{l}\PYG{o}{.}\PYG{n}{index}\PYG{o}{+}\PYG{l+m+mi}{1}\PYG{p}{,} \PYG{n}{l}\PYG{o}{.}\PYG{n}{fname}\PYG{p}{,} \PYG{n}{d}\PYG{p}{)}
        \PYG{n}{logger}\PYG{o}{.}\PYG{n}{error}\PYG{p}{(}\PYG{l+s+s2}{\PYGZdq{}}\PYG{l+s+s2}{      }\PYG{l+s+si}{\PYGZpc{}s}\PYG{l+s+s2}{\PYGZdq{}}\PYG{p}{,} \PYG{n}{l}\PYG{o}{.}\PYG{n}{text}\PYG{p}{)}

\PYG{k}{return} \PYG{n}{could\PYGZus{}write}
\end{Verbatim}


\subsection{search\_for\_generated}
\label{antiweb:search-for-generated}
The line numbers of the \sphinxcode{start(generated)} and \sphinxcode{(generated)} directive are looked up and their content getting depleted

\begin{Verbatim}[commandchars=\\\{\},numbers=left,firstnumber=1,stepnumber=1]
\PYG{k}{def} \PYG{n+nf}{search\PYGZus{}for\PYGZus{}generate}\PYG{p}{(}\PYG{n}{output}\PYG{p}{,} \PYG{n}{index\PYGZus{}rst}\PYG{p}{,} \PYG{n}{startblock}\PYG{p}{,} \PYG{n}{endblock}\PYG{p}{)}\PYG{p}{:}

    \PYG{n}{startline} \PYG{o}{=} \PYG{n+nb+bp}{None}
    \PYG{n}{endline} \PYG{o}{=} \PYG{n+nb+bp}{None}
    \PYG{n}{content} \PYG{o}{=} \PYG{l+s+s2}{\PYGZdq{}}\PYG{l+s+s2}{\PYGZdq{}}
    \PYG{k}{if} \PYG{o+ow}{not} \PYG{n}{output}\PYG{p}{:}
        \PYG{n}{output} \PYG{o}{=} \PYG{l+s+s2}{\PYGZdq{}}\PYG{l+s+s2}{\PYGZdq{}}

    \PYG{k}{while} \PYG{n}{startline} \PYG{o}{==} \PYG{n+nb+bp}{None} \PYG{o+ow}{or} \PYG{n}{endline} \PYG{o}{==} \PYG{n+nb+bp}{None}\PYG{p}{:}
        \PYG{n}{path} \PYG{o}{=} \PYG{n}{os}\PYG{o}{.}\PYG{n}{path}\PYG{o}{.}\PYG{n}{join}\PYG{p}{(}\PYG{n}{os}\PYG{o}{.}\PYG{n}{getcwd}\PYG{p}{(}\PYG{p}{)}\PYG{p}{,} \PYG{n}{output}\PYG{p}{,} \PYG{n}{index\PYGZus{}rst}\PYG{p}{)}
        \PYG{k}{with} \PYG{n+nb}{open}\PYG{p}{(}\PYG{n}{path}\PYG{p}{,} \PYG{l+s+s2}{\PYGZdq{}}\PYG{l+s+s2}{r}\PYG{l+s+s2}{\PYGZdq{}}\PYG{p}{)} \PYG{k}{as} \PYG{n}{index\PYGZus{}file}\PYG{p}{:}
            \PYG{k}{for} \PYG{n}{num}\PYG{p}{,} \PYG{n}{line} \PYG{o+ow}{in} \PYG{n+nb}{enumerate}\PYG{p}{(}\PYG{n}{index\PYGZus{}file}\PYG{p}{)}\PYG{p}{:}
                \PYG{k}{if} \PYG{n}{startblock} \PYG{o+ow}{in} \PYG{n}{line}\PYG{p}{:}
                    \PYG{n}{startline} \PYG{o}{=} \PYG{n}{num}
                \PYG{k}{if} \PYG{n}{endblock} \PYG{o+ow}{in} \PYG{n}{line}\PYG{p}{:}
                    \PYG{n}{endline} \PYG{o}{=} \PYG{n}{num}

            \PYG{k}{if} \PYG{n}{startline}\PYG{o}{==} \PYG{n+nb+bp}{None} \PYG{o+ow}{or} \PYG{n}{endline}\PYG{o}{==} \PYG{n+nb+bp}{None}\PYG{p}{:}
                \PYG{n}{write\PYGZus{}static}\PYG{p}{(}\PYG{n}{os}\PYG{o}{.}\PYG{n}{path}\PYG{o}{.}\PYG{n}{join}\PYG{p}{(}\PYG{n}{os}\PYG{o}{.}\PYG{n}{getcwd}\PYG{p}{(}\PYG{p}{)}\PYG{p}{,} \PYG{n}{output}\PYG{p}{)}\PYG{p}{,} \PYG{n}{index\PYGZus{}rst}\PYG{p}{,} \PYG{n}{startblock}\PYG{p}{,} \PYG{n}{endblock}\PYG{p}{)}
            \PYG{k}{else}\PYG{p}{:}
                \PYG{n}{index\PYGZus{}file}\PYG{o}{.}\PYG{n}{seek}\PYG{p}{(}\PYG{l+m+mi}{0}\PYG{p}{,} \PYG{l+m+mi}{0}\PYG{p}{)}
                \PYG{n}{content} \PYG{o}{=} \PYG{n}{index\PYGZus{}file}\PYG{o}{.}\PYG{n}{readlines}\PYG{p}{(}\PYG{p}{)}
                \PYG{k}{del} \PYG{n}{content}\PYG{p}{[}\PYG{n}{startline}\PYG{o}{+}\PYG{l+m+mi}{1}\PYG{p}{:}\PYG{n}{endline}\PYG{p}{]}
    \PYG{k}{return} \PYG{p}{(}\PYG{n}{content}\PYG{p}{,} \PYG{n}{startline}\PYG{p}{)}
\end{Verbatim}


\subsection{replace\_in\_generated}
\label{antiweb:replace-in-generated}
The name of the generated files get added between the \sphinxcode{start(generated)} and \sphinxcode{(generated)} directives. Code before and after is left as is.

\begin{Verbatim}[commandchars=\\\{\},numbers=left,firstnumber=1,stepnumber=1]
\PYG{k}{def} \PYG{n+nf}{replace\PYGZus{}in\PYGZus{}generated}\PYG{p}{(}\PYG{n}{startblock}\PYG{p}{,} \PYG{n}{endblock}\PYG{p}{,} \PYG{n}{out\PYGZus{}file\PYGZus{}name}\PYG{p}{,} \PYG{n}{path}\PYG{p}{,} \PYG{n}{output}\PYG{p}{,} \PYG{n}{index\PYGZus{}rst}\PYG{p}{,} \PYG{n}{content}\PYG{p}{,} \PYG{n}{startline}\PYG{p}{)}\PYG{p}{:}

    \PYG{k}{if} \PYG{n}{startline}\PYG{p}{:}
        \PYG{n}{endline} \PYG{o}{=} \PYG{n}{startline}\PYG{o}{+}\PYG{l+m+mi}{1}

    \PYG{n}{index\PYGZus{}var} \PYG{o}{=} \PYG{n}{os}\PYG{o}{.}\PYG{n}{path}\PYG{o}{.}\PYG{n}{splitext}\PYG{p}{(}\PYG{n}{out\PYGZus{}file\PYGZus{}name}\PYG{p}{)}\PYG{p}{[}\PYG{l+m+mi}{0}\PYG{p}{]}
    \PYG{k}{if} \PYG{n}{startline} \PYG{o+ow}{and} \PYG{n}{endline}\PYG{p}{:}
        \PYG{n}{content}\PYG{o}{.}\PYG{n}{insert}\PYG{p}{(}\PYG{n}{endline}\PYG{p}{,} \PYG{l+s+s2}{\PYGZdq{}}\PYG{l+s+s2}{   }\PYG{l+s+s2}{\PYGZdq{}} \PYG{o}{+} \PYG{n}{index\PYGZus{}var} \PYG{o}{+} \PYG{l+s+s2}{\PYGZdq{}}\PYG{l+s+se}{\PYGZbs{}n}\PYG{l+s+s2}{\PYGZdq{}}\PYG{p}{)}
    \PYG{k}{if} \PYG{n}{output}\PYG{p}{:}
        \PYG{n}{index\PYGZus{}out} \PYG{o}{=} \PYG{n+nb}{open}\PYG{p}{(}\PYG{n}{os}\PYG{o}{.}\PYG{n}{path}\PYG{o}{.}\PYG{n}{join}\PYG{p}{(}\PYG{n}{path}\PYG{p}{,} \PYG{n}{output}\PYG{p}{,} \PYG{n}{index\PYGZus{}rst}\PYG{p}{)}\PYG{p}{,} \PYG{l+s+s2}{\PYGZdq{}}\PYG{l+s+s2}{w}\PYG{l+s+s2}{\PYGZdq{}}\PYG{p}{)}
    \PYG{k}{else}\PYG{p}{:}
        \PYG{n}{index\PYGZus{}out} \PYG{o}{=} \PYG{n+nb}{open}\PYG{p}{(}\PYG{n}{os}\PYG{o}{.}\PYG{n}{path}\PYG{o}{.}\PYG{n}{join}\PYG{p}{(}\PYG{n}{path}\PYG{p}{,} \PYG{n}{index\PYGZus{}rst}\PYG{p}{)}\PYG{p}{,} \PYG{l+s+s2}{\PYGZdq{}}\PYG{l+s+s2}{w}\PYG{l+s+s2}{\PYGZdq{}}\PYG{p}{)}
    \PYG{k}{for} \PYG{n}{item} \PYG{o+ow}{in} \PYG{n}{content}\PYG{p}{:}
        \PYG{n}{index\PYGZus{}out}\PYG{o}{.}\PYG{n}{write}\PYG{p}{(}\PYG{n}{item}\PYG{p}{)}
    \PYG{n}{index\PYGZus{}out}\PYG{o}{.}\PYG{n}{close}\PYG{p}{(}\PYG{p}{)}
\end{Verbatim}


\section{How to add new languages}
\label{antiweb:how-to-add-new-languages}
New languages are added by writing a new Reader class
and registering it in the readers dictionary (see readers).
A simple Reader example is provides by {\hyperref[antiweb:CReader]{\sphinxcrossref{\sphinxcode{CReader}}}}
a more advances reader is {\hyperref[antiweb:PythonReader]{\sphinxcrossref{\sphinxcode{PythonReader}}}}.


\chapter{antisphinx}
\label{antisphinx:antisphinx}\label{antisphinx::doc}
This sphinx extension modifies the syntax highlight mechanism to handle
\textless{}\textless{}textblock\textgreater{}\textgreater{} abbreviations in source code.
Additionally it makes them linking to the referring source block.

The primary technique is:
\begin{enumerate}
\item {} 
Extend the basic pygments language lexer with a new \emph{Heading} token.

\item {} 
Filter the html output of pygment lexing process: Replacing the
heading's \sphinxcode{\textless{}span\textgreater{}} tag  by an \sphinxcode{\textless{}a\textgreater{}} tag,  referencing the
a block.

\end{enumerate}


\section{File Layout}
\label{antisphinx:file-layout}
\begin{Verbatim}[commandchars=\\\{\}]
\PYG{o}{\PYGZlt{}\PYGZlt{}}\PYG{n}{imports}\PYG{o}{\PYGZgt{}\PYGZgt{}}
\PYG{o}{\PYGZlt{}\PYGZlt{}}\PYG{n}{export}\PYG{o}{\PYGZgt{}\PYGZgt{}}
\PYG{o}{\PYGZlt{}\PYGZlt{}}\PYG{n}{Lexers}\PYG{o}{\PYGZgt{}\PYGZgt{}}
\PYG{o}{\PYGZlt{}\PYGZlt{}}\PYG{n}{Filter} \PYG{n}{Output}\PYG{o}{\PYGZgt{}\PYGZgt{}}
\end{Verbatim}


\subsection{\textless{}\textless{}imports\textgreater{}\textgreater{}}
\label{antisphinx:imports}
\begin{Verbatim}[commandchars=\\\{\}]
\PYG{k+kn}{import} \PYG{n+nn}{sphinx.highlighting} \PYG{k+kn}{as} \PYG{n+nn}{shighlighting}
\PYG{k+kn}{import} \PYG{n+nn}{pygments.lexers} \PYG{k+kn}{as} \PYG{n+nn}{plexers}
\PYG{k+kn}{import} \PYG{n+nn}{pygments}
\PYG{k+kn}{import} \PYG{n+nn}{re}
\PYG{k+kn}{from} \PYG{n+nn}{pygments.token} \PYG{k+kn}{import} \PYG{n}{Token}
\end{Verbatim}


\section{\textless{}\textless{}exports\textgreater{}\textgreater{}}
\label{antisphinx:exports}
Das ist der Export Test Text
\begin{quote}

\begin{Verbatim}[commandchars=\\\{\}]
\PYG{n}{priority} \PYG{o}{=} \PYG{l+m+mi}{5}
\end{Verbatim}
\end{quote}


\subsection{\textless{}\textless{}Lexers\textgreater{}\textgreater{}}
\label{antisphinx:lexers}
\begin{Verbatim}[commandchars=\\\{\},numbers=left,firstnumber=1,stepnumber=1]
\PYG{k}{class} \PYG{n+nc}{CHeaderLexer}\PYG{p}{(}\PYG{n}{plexers}\PYG{o}{.}\PYG{n}{CLexer}\PYG{p}{)}\PYG{p}{:}
    \PYG{n}{tokens} \PYG{o}{=} \PYG{n}{plexers}\PYG{o}{.}\PYG{n}{CLexer}\PYG{o}{.}\PYG{n}{tokens}\PYG{o}{.}\PYG{n}{copy}\PYG{p}{(}\PYG{p}{)}
    \PYG{n}{tokens}\PYG{p}{[}\PYG{l+s+s2}{\PYGZdq{}}\PYG{l+s+s2}{whitespace}\PYG{l+s+s2}{\PYGZdq{}}\PYG{p}{]} \PYG{o}{=} \PYG{p}{[} \PYG{p}{(}\PYG{l+s+s1}{r\PYGZsq{}}\PYG{l+s+s1}{(?m)\PYGZca{}}\PYG{l+s+s1}{\PYGZbs{}}\PYG{l+s+s1}{s*\PYGZlt{}\PYGZlt{}.+\PYGZgt{}\PYGZgt{}}\PYG{l+s+s1}{\PYGZbs{}}\PYG{l+s+s1}{s*\PYGZdl{}}\PYG{l+s+s1}{\PYGZsq{}}\PYG{p}{,} \PYG{n}{Token}\PYG{o}{.}\PYG{n}{Generic}\PYG{o}{.}\PYG{n}{Heading}\PYG{p}{)}\PYG{p}{,} \PYG{p}{]}\PYGZbs{}
                           \PYG{o}{+} \PYG{n}{plexers}\PYG{o}{.}\PYG{n}{CLexer}\PYG{o}{.}\PYG{n}{tokens}\PYG{p}{[}\PYG{l+s+s2}{\PYGZdq{}}\PYG{l+s+s2}{whitespace}\PYG{l+s+s2}{\PYGZdq{}}\PYG{p}{]}

\PYG{n}{CHeaderLexer}\PYG{o}{.}\PYG{n}{\PYGZus{}tokens} \PYG{o}{=} \PYG{n}{CHeaderLexer}\PYG{o}{.}\PYG{n}{process\PYGZus{}tokendef}\PYG{p}{(}\PYG{l+s+s1}{\PYGZsq{}}\PYG{l+s+s1}{\PYGZsq{}}\PYG{p}{,} \PYG{n}{CHeaderLexer}\PYG{o}{.}\PYG{n}{tokens}\PYG{p}{)}


\PYG{k}{class} \PYG{n+nc}{PythonHeaderLexer}\PYG{p}{(}\PYG{n}{plexers}\PYG{o}{.}\PYG{n}{PythonLexer}\PYG{p}{)}\PYG{p}{:}
    \PYG{n}{tokens} \PYG{o}{=} \PYG{n}{plexers}\PYG{o}{.}\PYG{n}{PythonLexer}\PYG{o}{.}\PYG{n}{tokens}\PYG{o}{.}\PYG{n}{copy}\PYG{p}{(}\PYG{p}{)}
    \PYG{n}{tokens}\PYG{p}{[}\PYG{l+s+s2}{\PYGZdq{}}\PYG{l+s+s2}{root}\PYG{l+s+s2}{\PYGZdq{}}\PYG{p}{]} \PYG{o}{=} \PYG{p}{[} \PYG{p}{(}\PYG{l+s+s1}{r\PYGZsq{}}\PYG{l+s+s1}{\PYGZca{}}\PYG{l+s+s1}{\PYGZbs{}}\PYG{l+s+s1}{s*\PYGZlt{}\PYGZlt{}.+\PYGZgt{}\PYGZgt{}}\PYG{l+s+s1}{\PYGZbs{}}\PYG{l+s+s1}{s*\PYGZdl{}}\PYG{l+s+s1}{\PYGZsq{}}\PYG{p}{,} \PYG{n}{Token}\PYG{o}{.}\PYG{n}{Generic}\PYG{o}{.}\PYG{n}{Heading}\PYG{p}{)}\PYG{p}{,} \PYG{p}{]}\PYGZbs{}
                     \PYG{o}{+} \PYG{n}{plexers}\PYG{o}{.}\PYG{n}{PythonLexer}\PYG{o}{.}\PYG{n}{tokens}\PYG{p}{[}\PYG{l+s+s2}{\PYGZdq{}}\PYG{l+s+s2}{root}\PYG{l+s+s2}{\PYGZdq{}}\PYG{p}{]}

\PYG{n}{PythonHeaderLexer}\PYG{o}{.}\PYG{n}{\PYGZus{}tokens} \PYG{o}{=} \PYG{n}{PythonHeaderLexer}\PYG{o}{.}\PYG{n}{process\PYGZus{}tokendef}\PYG{p}{(}\PYG{l+s+s1}{\PYGZsq{}}\PYG{l+s+s1}{\PYGZsq{}}\PYG{p}{,} \PYG{n}{PythonHeaderLexer}\PYG{o}{.}\PYG{n}{tokens}\PYG{p}{)}

\PYG{c+c1}{\PYGZsh{}replace the sphinx lexers by the new Lexers}
\PYG{n}{shighlighting}\PYG{o}{.}\PYG{n}{lexers}\PYG{p}{[}\PYG{l+s+s2}{\PYGZdq{}}\PYG{l+s+s2}{c}\PYG{l+s+s2}{\PYGZdq{}}\PYG{p}{]} \PYG{o}{=} \PYG{n}{CHeaderLexer}\PYG{p}{(}\PYG{p}{)}
\PYG{n}{shighlighting}\PYG{o}{.}\PYG{n}{lexers}\PYG{p}{[}\PYG{l+s+s2}{\PYGZdq{}}\PYG{l+s+s2}{python}\PYG{l+s+s2}{\PYGZdq{}}\PYG{p}{]} \PYG{o}{=} \PYG{n}{PythonHeaderLexer}\PYG{p}{(}\PYG{p}{)}
\end{Verbatim}


\subsection{\textless{}\textless{}Filter Output\textgreater{}\textgreater{}}
\label{antisphinx:filter-output}
\begin{Verbatim}[commandchars=\\\{\},numbers=left,firstnumber=1,stepnumber=1]
\PYG{n}{re\PYGZus{}html\PYGZus{}heading} \PYG{o}{=} \PYG{n}{re}\PYG{o}{.}\PYG{n}{compile}\PYG{p}{(}\PYG{l+s+s1}{\PYGZsq{}}\PYG{l+s+s1}{\PYGZlt{}span class=}\PYG{l+s+s1}{\PYGZdq{}}\PYG{l+s+s1}{gh}\PYG{l+s+s1}{\PYGZdq{}}\PYG{l+s+s1}{\PYGZgt{}(.*?)\PYGZlt{}/span\PYGZgt{}}\PYG{l+s+s1}{\PYGZsq{}}\PYG{p}{)}

\PYG{k}{def} \PYG{n+nf}{highlight}\PYG{p}{(}\PYG{n}{code}\PYG{p}{,} \PYG{n}{lexer}\PYG{p}{,} \PYG{n}{formatter}\PYG{p}{,} \PYG{n}{outfile}\PYG{o}{=}\PYG{n+nb+bp}{None}\PYG{p}{)}\PYG{p}{:}
    \PYG{o}{\PYGZlt{}\PYGZlt{}}\PYG{n}{make} \PYG{n}{anchor}\PYG{o}{\PYGZgt{}\PYGZgt{}}

    \PYG{n}{output} \PYG{o}{=} \PYG{n}{pygments}\PYG{o}{.}\PYG{n}{highlight}\PYG{p}{(}\PYG{n}{code}\PYG{p}{,} \PYG{n}{lexer}\PYG{p}{,} \PYG{n}{formatter}\PYG{p}{,} \PYG{n}{outfile}\PYG{p}{)}
    \PYG{n}{output}\PYG{p}{,} \PYG{n}{noc} \PYG{o}{=} \PYG{n}{re\PYGZus{}html\PYGZus{}heading}\PYG{o}{.}\PYG{n}{subn}\PYG{p}{(}\PYG{n}{make\PYGZus{}anchor}\PYG{p}{,} \PYG{n}{output}\PYG{p}{)}
    \PYG{k}{return} \PYG{n}{output}

\PYG{c+c1}{\PYGZsh{}monkey path the original sphinx highlighting}
\PYG{n}{shighlighting}\PYG{o}{.}\PYG{n}{highlight} \PYG{o}{=} \PYG{n}{highlight}
\PYG{n}{shighlighting}\PYG{o}{.}\PYG{n}{parser} \PYG{o}{=} \PYG{n+nb+bp}{None}


\PYG{k}{def} \PYG{n+nf}{setup}\PYG{p}{(}\PYG{n}{app}\PYG{p}{)}\PYG{p}{:}
    \PYG{c+c1}{\PYGZsh{}is needed for sphinx extension mechanism}
    \PYG{k}{pass}
\end{Verbatim}
\phantomsection\label{antisphinx:make-anchor}
\textbf{\textless{}\textless{}make anchor\textgreater{}\textgreater{}}

\begin{Verbatim}[commandchars=\\\{\},numbers=left,firstnumber=1,stepnumber=1]
\PYG{k}{def} \PYG{n+nf}{make\PYGZus{}anchor}\PYG{p}{(}\PYG{n}{mo}\PYG{p}{)}\PYG{p}{:}
    \PYG{n}{indented\PYGZus{}name} \PYG{o}{=} \PYG{n}{mo}\PYG{o}{.}\PYG{n}{group}\PYG{p}{(}\PYG{l+m+mi}{1}\PYG{p}{)}
    \PYG{n}{indent} \PYG{o}{=} \PYG{n+nb}{len}\PYG{p}{(}\PYG{n}{indented\PYGZus{}name}\PYG{p}{)}\PYG{o}{\PYGZhy{}}\PYG{n+nb}{len}\PYG{p}{(}\PYG{n}{indented\PYGZus{}name}\PYG{o}{.}\PYG{n}{lstrip}\PYG{p}{(}\PYG{p}{)}\PYG{p}{)}
    \PYG{n}{name} \PYG{o}{=} \PYG{n}{indented\PYGZus{}name}\PYG{o}{.}\PYG{n}{strip}\PYG{p}{(}\PYG{p}{)}

    \PYG{c+c1}{\PYGZsh{}mangle the textblock name to satisfy the sphinx anchor names.}
    \PYG{n}{href} \PYG{o}{=} \PYG{n}{name}\PYG{o}{.}\PYG{n}{replace}\PYG{p}{(}\PYG{l+s+s2}{\PYGZdq{}}\PYG{l+s+s2}{\PYGZam{}lt;}\PYG{l+s+s2}{\PYGZdq{}}\PYG{p}{,} \PYG{l+s+s2}{\PYGZdq{}}\PYG{l+s+s2}{\PYGZdq{}}\PYG{p}{)}\PYG{o}{.}\PYG{n}{replace}\PYG{p}{(}\PYG{l+s+s2}{\PYGZdq{}}\PYG{l+s+s2}{\PYGZam{}gt;}\PYG{l+s+s2}{\PYGZdq{}}\PYG{p}{,} \PYG{l+s+s2}{\PYGZdq{}}\PYG{l+s+s2}{\PYGZdq{}}\PYG{p}{)}\PYGZbs{}
           \PYG{o}{.}\PYG{n}{replace}\PYG{p}{(}\PYG{l+s+s2}{\PYGZdq{}}\PYG{l+s+s2}{ }\PYG{l+s+s2}{\PYGZdq{}}\PYG{p}{,} \PYG{l+s+s2}{\PYGZdq{}}\PYG{l+s+s2}{\PYGZhy{}}\PYG{l+s+s2}{\PYGZdq{}}\PYG{p}{)}\PYG{o}{.}\PYG{n}{replace}\PYG{p}{(}\PYG{l+s+s2}{\PYGZdq{}}\PYG{l+s+s2}{:}\PYG{l+s+s2}{\PYGZdq{}}\PYG{p}{,} \PYG{l+s+s2}{\PYGZdq{}}\PYG{l+s+s2}{\PYGZhy{}}\PYG{l+s+s2}{\PYGZdq{}}\PYG{p}{)}\PYG{o}{.}\PYG{n}{replace}\PYG{p}{(}\PYG{l+s+s2}{\PYGZdq{}}\PYG{l+s+s2}{+}\PYG{l+s+s2}{\PYGZdq{}}\PYG{p}{,} \PYG{l+s+s2}{\PYGZdq{}}\PYG{l+s+s2}{\PYGZhy{}}\PYG{l+s+s2}{\PYGZdq{}}\PYG{p}{)}

    \PYG{k}{if} \PYG{l+s+s2}{\PYGZdq{}}\PYG{l+s+s2}{.}\PYG{l+s+s2}{\PYGZdq{}} \PYG{o+ow}{in} \PYG{n}{href}\PYG{p}{:}
        \PYG{n}{path} \PYG{o}{=} \PYG{n}{href}\PYG{o}{.}\PYG{n}{split}\PYG{p}{(}\PYG{l+s+s2}{\PYGZdq{}}\PYG{l+s+s2}{.}\PYG{l+s+s2}{\PYGZdq{}}\PYG{p}{)}
        \PYG{n}{href} \PYG{o}{=} \PYG{n}{path}\PYG{p}{[}\PYG{l+m+mi}{0}\PYG{p}{]} \PYG{o}{+} \PYG{l+s+s2}{\PYGZdq{}}\PYG{l+s+s2}{.}\PYG{l+s+s2}{\PYGZdq{}} \PYG{o}{+} \PYG{l+s+s2}{\PYGZdq{}}\PYG{l+s+s2}{.}\PYG{l+s+s2}{\PYGZdq{}}\PYG{o}{.}\PYG{n}{join}\PYG{p}{(}\PYG{n}{path}\PYG{p}{[}\PYG{l+m+mi}{1}\PYG{p}{:}\PYG{p}{]}\PYG{p}{)}\PYG{o}{.}\PYG{n}{lower}\PYG{p}{(}\PYG{p}{)}
    \PYG{k}{else}\PYG{p}{:}
        \PYG{n}{href} \PYG{o}{=} \PYG{n}{href}\PYG{o}{.}\PYG{n}{replace}\PYG{p}{(}\PYG{l+s+s2}{\PYGZdq{}}\PYG{l+s+s2}{\PYGZus{}}\PYG{l+s+s2}{\PYGZdq{}}\PYG{p}{,} \PYG{l+s+s2}{\PYGZdq{}}\PYG{l+s+s2}{\PYGZhy{}}\PYG{l+s+s2}{\PYGZdq{}}\PYG{p}{)}\PYG{o}{.}\PYG{n}{lower}\PYG{p}{(}\PYG{p}{)}

    \PYG{k}{if} \PYG{n}{href}\PYG{o}{.}\PYG{n}{startswith}\PYG{p}{(}\PYG{l+s+s2}{\PYGZdq{}}\PYG{l+s+s2}{\PYGZhy{}}\PYG{l+s+s2}{\PYGZdq{}}\PYG{p}{)}\PYG{p}{:}
        \PYG{n}{href} \PYG{o}{=} \PYG{n}{href}\PYG{p}{[}\PYG{l+m+mi}{1}\PYG{p}{:}\PYG{p}{]}

    \PYG{n}{phref} \PYG{o}{=} \PYG{n+nb+bp}{None}
    \PYG{k}{while} \PYG{n}{phref} \PYG{o}{!=} \PYG{n}{href}\PYG{p}{:}
        \PYG{n}{phref} \PYG{o}{=} \PYG{n}{href}
        \PYG{n}{href} \PYG{o}{=} \PYG{n}{href}\PYG{o}{.}\PYG{n}{replace}\PYG{p}{(}\PYG{l+s+s2}{\PYGZdq{}}\PYG{l+s+s2}{\PYGZhy{}\PYGZhy{}}\PYG{l+s+s2}{\PYGZdq{}}\PYG{p}{,} \PYG{l+s+s2}{\PYGZdq{}}\PYG{l+s+s2}{\PYGZhy{}}\PYG{l+s+s2}{\PYGZdq{}}\PYG{p}{)}

    \PYG{k}{return} \PYG{l+s+s1}{\PYGZsq{}}\PYG{l+s+s1}{\PYGZlt{}span class=}\PYG{l+s+s1}{\PYGZdq{}}\PYG{l+s+s1}{gh}\PYG{l+s+s1}{\PYGZdq{}}\PYG{l+s+s1}{\PYGZgt{}}\PYG{l+s+si}{\PYGZpc{}s}\PYG{l+s+s1}{\PYGZlt{}a href=}\PYG{l+s+s1}{\PYGZdq{}}\PYG{l+s+s1}{\PYGZsh{}}\PYG{l+s+si}{\PYGZpc{}s}\PYG{l+s+s1}{\PYGZdq{}}\PYG{l+s+s1}{\PYGZgt{}}\PYG{l+s+si}{\PYGZpc{}s}\PYG{l+s+s1}{\PYGZlt{}/a\PYGZgt{}\PYGZlt{}/span\PYGZgt{}}\PYG{l+s+s1}{\PYGZsq{}} \PYGZbs{}
           \PYG{o}{\PYGZpc{}} \PYG{p}{(}\PYG{n}{indented\PYGZus{}name}\PYG{p}{[}\PYG{p}{:}\PYG{n}{indent}\PYG{p}{]}\PYG{p}{,} \PYG{n}{href}\PYG{p}{,} \PYG{n}{name}\PYG{p}{)}
\end{Verbatim}


\chapter{Changelog}
\label{changelog::doc}\label{changelog:changelog}\begin{itemize}
\item {} \begin{description}
\item[{10.08.2015:}] \leavevmode
Added support for C\#

Added installation instruction for use with sphinx

\end{description}

\item {} \begin{description}
\item[{11.08.2015:}] \leavevmode
Added a Getting Started section for quick use

``Installation'' and ``Getting Started'' improvements

\end{description}

\item {} \begin{description}
\item[{12.08.2015:}] \leavevmode
Added the -r flag: Process every compatible file in given directory

Added the -i flag: All processed files will be included in Sphinx' index.rst

\end{description}

\item {} \begin{description}
\item[{13.08.2015:}] \leavevmode
Added all changes to the documentation

Copyright adaption

Introducing a new version 0.3

\end{description}

\item {} \begin{description}
\item[{18.08.2015}] \leavevmode
Using the -o flag with -r now indicates the path where to save the documentation

Fixed some bugs

Improved behaviour

\end{description}

\item {} \begin{description}
\item[{18.08.2015 - 08.08.2016}] \leavevmode
Many different improvements and bugfixes

\end{description}

\item {} \begin{description}
\item[{08.08.2016}] \leavevmode
Improved antiweb documentation

\end{description}

\end{itemize}


\chapter{Why another literate programming tool?}
\label{motivation::doc}\label{motivation:why-another-literate-programming-tool}
Over the years I tried out several literate programming tools (cweb, funnelweb, leo and many more).
But all these tools suffer from several problems which are caused by
the \emph{book} approach: \href{http://en.wikipedia.org/wiki/WEB}{web} was built to write primary a book and secondly
to get a program.  Donald Knuth is a professor who is used to publish papers,
from that point of view \href{http://en.wikipedia.org/wiki/WEB}{web} is the perfect tool.

For a software developer the primary goal is a working, optimized program.
The second goal is good documentation for the fellow programmers to be able to
extend and maintain the code.

Most of the time a software developer spends for the following two tasks
(the \emph{main development cycle}):
\begin{itemize}
\item {} 
Test and debug program.

\item {} 
Adapt the code and start again testing.

\end{itemize}

And while \href{http://en.wikipedia.org/wiki/WEB}{web} is a cool \emph{top down} design tool. It is quite annoying to use it
during the \emph{main development cycle}, because it needs several extra steps compared
to non literate programing:
\begin{description}
\item[{In non literate python programming the \emph{main development cycle} is very fast:}] \leavevmode\begin{itemize}
\item {} 
Start the program and test.

\item {} 
Find the bug.

\item {} 
Change the code.

\item {} 
Start the program again.

\end{itemize}

\item[{In literate python programming there are several more steps:}] \leavevmode\begin{itemize}
\item {} 
Start the program and test.

\item {} 
Find the bug in the tangled code.

\item {} 
Identify the buggy code in the web document.

\item {} 
Change the web document.

\item {} 
Tangle the new source code.

\item {} 
Start the program again.

\end{itemize}

\end{description}

The unnecessary extra steps slow down the productivity, which is probably
the cause that literate programming didn't become broadly accepted in
professional computer programming.

Another personal reason for my web dislike is: The tangled source code is not ``mine'',
but some canned text, which is polluted by sentinels. Most of my time during
the \emph{main development cycle} I was spending with ugly generated code.


\section{Why are document extractors like doxygen not sufficient?}
\label{motivation:why-are-document-extractors-like-doxygen-not-sufficient}
The results of doxygen, epidoc, etc. are like the tangle source code of
\href{http://en.wikipedia.org/wiki/WEB}{web}: It is a canned piece of text. But source code is a complex material.
Describing and explaining it cannot by done using a cookie cutter approach.


\section{Features of antiweb}
\label{motivation:features-of-antiweb}
Antiweb weaves restructured code from source code.
\begin{itemize}
\item {} 
The source is partitioned in text blocks that can
be freely arranged to generate the source.

\item {} 
Simple support for macros

\item {} 
Support for conditional weaving (i.e. base on the
audience you can generate different outputs)

\item {} 
Supported languages: python, C/C++

\item {} 
Python doc string are supported.

\end{itemize}



\renewcommand{\indexname}{Index}
\printindex
\end{document}
